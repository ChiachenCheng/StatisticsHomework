%!TEX program = xelatex
\documentclass[11pt]{ctexart}

    \usepackage[breakable]{tcolorbox}
    \usepackage{parskip} % Stop auto-indenting (to mimic markdown behaviour)
    
    \usepackage{iftex}
    \ifPDFTeX
    	\usepackage[T1]{fontenc}
    	\usepackage{mathpazo}
    \else
    	\usepackage{fontspec}
    \fi

    % Basic figure setup, for now with no caption control since it's done
    % automatically by Pandoc (which extracts ![](path) syntax from Markdown).
    \usepackage{graphicx}
    % Maintain compatibility with old templates. Remove in nbconvert 6.0
    \let\Oldincludegraphics\includegraphics
    % Ensure that by default, figures have no caption (until we provide a
    % proper Figure object with a Caption API and a way to capture that
    % in the conversion process - todo).
    \usepackage{caption}
    \DeclareCaptionFormat{nocaption}{}
    \captionsetup{format=nocaption,aboveskip=0pt,belowskip=0pt}

    \usepackage{float}
    \floatplacement{figure}{H} % forces figures to be placed at the correct location
    \usepackage{xcolor} % Allow colors to be defined
    \usepackage{enumerate} % Needed for markdown enumerations to work
    \usepackage{geometry} % Used to adjust the document margins
    \usepackage{amsmath} % Equations
    \usepackage{amssymb} % Equations
    \usepackage{textcomp} % defines textquotesingle
    % Hack from http://tex.stackexchange.com/a/47451/13684:
    \AtBeginDocument{%
        \def\PYZsq{\textquotesingle}% Upright quotes in Pygmentized code
    }
    \usepackage{upquote} % Upright quotes for verbatim code
    \usepackage{eurosym} % defines \euro
    \usepackage[mathletters]{ucs} % Extended unicode (utf-8) support
    \usepackage{fancyvrb} % verbatim replacement that allows latex
    \usepackage{grffile} % extends the file name processing of package graphics 
                         % to support a larger range
    \makeatletter % fix for old versions of grffile with XeLaTeX
    \@ifpackagelater{grffile}{2019/11/01}
    {
      % Do nothing on new versions
    }
    {
      \def\Gread@@xetex#1{%
        \IfFileExists{"\Gin@base".bb}%
        {\Gread@eps{\Gin@base.bb}}%
        {\Gread@@xetex@aux#1}%
      }
    }
    \makeatother
    \usepackage[Export]{adjustbox} % Used to constrain images to a maximum size
    \adjustboxset{max size={0.9\linewidth}{0.9\paperheight}}

    % The hyperref package gives us a pdf with properly built
    % internal navigation ('pdf bookmarks' for the table of contents,
    % internal cross-reference links, web links for URLs, etc.)
    \usepackage{hyperref}
    % The default LaTeX title has an obnoxious amount of whitespace. By default,
    % titling removes some of it. It also provides customization options.
    \usepackage{titling}
    \usepackage{longtable} % longtable support required by pandoc >1.10
    \usepackage{booktabs}  % table support for pandoc > 1.12.2
    \usepackage[inline]{enumitem} % IRkernel/repr support (it uses the enumerate* environment)
    \usepackage[normalem]{ulem} % ulem is needed to support strikethroughs (\sout)
                                % normalem makes italics be italics, not underlines
    \usepackage{mathrsfs}
    

    
    % Colors for the hyperref package
    \definecolor{urlcolor}{rgb}{0,.145,.698}
    \definecolor{linkcolor}{rgb}{.71,0.21,0.01}
    \definecolor{citecolor}{rgb}{.12,.54,.11}

    % ANSI colors
    \definecolor{ansi-black}{HTML}{3E424D}
    \definecolor{ansi-black-intense}{HTML}{282C36}
    \definecolor{ansi-red}{HTML}{E75C58}
    \definecolor{ansi-red-intense}{HTML}{B22B31}
    \definecolor{ansi-green}{HTML}{00A250}
    \definecolor{ansi-green-intense}{HTML}{007427}
    \definecolor{ansi-yellow}{HTML}{DDB62B}
    \definecolor{ansi-yellow-intense}{HTML}{B27D12}
    \definecolor{ansi-blue}{HTML}{208FFB}
    \definecolor{ansi-blue-intense}{HTML}{0065CA}
    \definecolor{ansi-magenta}{HTML}{D160C4}
    \definecolor{ansi-magenta-intense}{HTML}{A03196}
    \definecolor{ansi-cyan}{HTML}{60C6C8}
    \definecolor{ansi-cyan-intense}{HTML}{258F8F}
    \definecolor{ansi-white}{HTML}{C5C1B4}
    \definecolor{ansi-white-intense}{HTML}{A1A6B2}
    \definecolor{ansi-default-inverse-fg}{HTML}{FFFFFF}
    \definecolor{ansi-default-inverse-bg}{HTML}{000000}

    % common color for the border for error outputs.
    \definecolor{outerrorbackground}{HTML}{FFDFDF}

    % commands and environments needed by pandoc snippets
    % extracted from the output of `pandoc -s`
    \providecommand{\tightlist}{%
      \setlength{\itemsep}{0pt}\setlength{\parskip}{0pt}}
    \DefineVerbatimEnvironment{Highlighting}{Verbatim}{commandchars=\\\{\}}
    % Add ',fontsize=\small' for more characters per line
    \newenvironment{Shaded}{}{}
    \newcommand{\KeywordTok}[1]{\textcolor[rgb]{0.00,0.44,0.13}{\textbf{{#1}}}}
    \newcommand{\DataTypeTok}[1]{\textcolor[rgb]{0.56,0.13,0.00}{{#1}}}
    \newcommand{\DecValTok}[1]{\textcolor[rgb]{0.25,0.63,0.44}{{#1}}}
    \newcommand{\BaseNTok}[1]{\textcolor[rgb]{0.25,0.63,0.44}{{#1}}}
    \newcommand{\FloatTok}[1]{\textcolor[rgb]{0.25,0.63,0.44}{{#1}}}
    \newcommand{\CharTok}[1]{\textcolor[rgb]{0.25,0.44,0.63}{{#1}}}
    \newcommand{\StringTok}[1]{\textcolor[rgb]{0.25,0.44,0.63}{{#1}}}
    \newcommand{\CommentTok}[1]{\textcolor[rgb]{0.38,0.63,0.69}{\textit{{#1}}}}
    \newcommand{\OtherTok}[1]{\textcolor[rgb]{0.00,0.44,0.13}{{#1}}}
    \newcommand{\AlertTok}[1]{\textcolor[rgb]{1.00,0.00,0.00}{\textbf{{#1}}}}
    \newcommand{\FunctionTok}[1]{\textcolor[rgb]{0.02,0.16,0.49}{{#1}}}
    \newcommand{\RegionMarkerTok}[1]{{#1}}
    \newcommand{\ErrorTok}[1]{\textcolor[rgb]{1.00,0.00,0.00}{\textbf{{#1}}}}
    \newcommand{\NormalTok}[1]{{#1}}
    
    % Additional commands for more recent versions of Pandoc
    \newcommand{\ConstantTok}[1]{\textcolor[rgb]{0.53,0.00,0.00}{{#1}}}
    \newcommand{\SpecialCharTok}[1]{\textcolor[rgb]{0.25,0.44,0.63}{{#1}}}
    \newcommand{\VerbatimStringTok}[1]{\textcolor[rgb]{0.25,0.44,0.63}{{#1}}}
    \newcommand{\SpecialStringTok}[1]{\textcolor[rgb]{0.73,0.40,0.53}{{#1}}}
    \newcommand{\ImportTok}[1]{{#1}}
    \newcommand{\DocumentationTok}[1]{\textcolor[rgb]{0.73,0.13,0.13}{\textit{{#1}}}}
    \newcommand{\AnnotationTok}[1]{\textcolor[rgb]{0.38,0.63,0.69}{\textbf{\textit{{#1}}}}}
    \newcommand{\CommentVarTok}[1]{\textcolor[rgb]{0.38,0.63,0.69}{\textbf{\textit{{#1}}}}}
    \newcommand{\VariableTok}[1]{\textcolor[rgb]{0.10,0.09,0.49}{{#1}}}
    \newcommand{\ControlFlowTok}[1]{\textcolor[rgb]{0.00,0.44,0.13}{\textbf{{#1}}}}
    \newcommand{\OperatorTok}[1]{\textcolor[rgb]{0.40,0.40,0.40}{{#1}}}
    \newcommand{\BuiltInTok}[1]{{#1}}
    \newcommand{\ExtensionTok}[1]{{#1}}
    \newcommand{\PreprocessorTok}[1]{\textcolor[rgb]{0.74,0.48,0.00}{{#1}}}
    \newcommand{\AttributeTok}[1]{\textcolor[rgb]{0.49,0.56,0.16}{{#1}}}
    \newcommand{\InformationTok}[1]{\textcolor[rgb]{0.38,0.63,0.69}{\textbf{\textit{{#1}}}}}
    \newcommand{\WarningTok}[1]{\textcolor[rgb]{0.38,0.63,0.69}{\textbf{\textit{{#1}}}}}
    
    
    % Define a nice break command that doesn't care if a line doesn't already
    % exist.
    \def\br{\hspace*{\fill} \\* }
    % Math Jax compatibility definitions
    \def\gt{>}
    \def\lt{<}
    \let\Oldtex\TeX
    \let\Oldlatex\LaTeX
    \renewcommand{\TeX}{\textrm{\Oldtex}}
    \renewcommand{\LaTeX}{\textrm{\Oldlatex}}
    % Document parameters
    % Document title
    \title{第六周实验作业}
    
    
    
    
    
% Pygments definitions
\makeatletter
\def\PY@reset{\let\PY@it=\relax \let\PY@bf=\relax%
    \let\PY@ul=\relax \let\PY@tc=\relax%
    \let\PY@bc=\relax \let\PY@ff=\relax}
\def\PY@tok#1{\csname PY@tok@#1\endcsname}
\def\PY@toks#1+{\ifx\relax#1\empty\else%
    \PY@tok{#1}\expandafter\PY@toks\fi}
\def\PY@do#1{\PY@bc{\PY@tc{\PY@ul{%
    \PY@it{\PY@bf{\PY@ff{#1}}}}}}}
\def\PY#1#2{\PY@reset\PY@toks#1+\relax+\PY@do{#2}}

\expandafter\def\csname PY@tok@w\endcsname{\def\PY@tc##1{\textcolor[rgb]{0.73,0.73,0.73}{##1}}}
\expandafter\def\csname PY@tok@c\endcsname{\let\PY@it=\textit\def\PY@tc##1{\textcolor[rgb]{0.25,0.50,0.50}{##1}}}
\expandafter\def\csname PY@tok@cp\endcsname{\def\PY@tc##1{\textcolor[rgb]{0.74,0.48,0.00}{##1}}}
\expandafter\def\csname PY@tok@k\endcsname{\let\PY@bf=\textbf\def\PY@tc##1{\textcolor[rgb]{0.00,0.50,0.00}{##1}}}
\expandafter\def\csname PY@tok@kp\endcsname{\def\PY@tc##1{\textcolor[rgb]{0.00,0.50,0.00}{##1}}}
\expandafter\def\csname PY@tok@kt\endcsname{\def\PY@tc##1{\textcolor[rgb]{0.69,0.00,0.25}{##1}}}
\expandafter\def\csname PY@tok@o\endcsname{\def\PY@tc##1{\textcolor[rgb]{0.40,0.40,0.40}{##1}}}
\expandafter\def\csname PY@tok@ow\endcsname{\let\PY@bf=\textbf\def\PY@tc##1{\textcolor[rgb]{0.67,0.13,1.00}{##1}}}
\expandafter\def\csname PY@tok@nb\endcsname{\def\PY@tc##1{\textcolor[rgb]{0.00,0.50,0.00}{##1}}}
\expandafter\def\csname PY@tok@nf\endcsname{\def\PY@tc##1{\textcolor[rgb]{0.00,0.00,1.00}{##1}}}
\expandafter\def\csname PY@tok@nc\endcsname{\let\PY@bf=\textbf\def\PY@tc##1{\textcolor[rgb]{0.00,0.00,1.00}{##1}}}
\expandafter\def\csname PY@tok@nn\endcsname{\let\PY@bf=\textbf\def\PY@tc##1{\textcolor[rgb]{0.00,0.00,1.00}{##1}}}
\expandafter\def\csname PY@tok@ne\endcsname{\let\PY@bf=\textbf\def\PY@tc##1{\textcolor[rgb]{0.82,0.25,0.23}{##1}}}
\expandafter\def\csname PY@tok@nv\endcsname{\def\PY@tc##1{\textcolor[rgb]{0.10,0.09,0.49}{##1}}}
\expandafter\def\csname PY@tok@no\endcsname{\def\PY@tc##1{\textcolor[rgb]{0.53,0.00,0.00}{##1}}}
\expandafter\def\csname PY@tok@nl\endcsname{\def\PY@tc##1{\textcolor[rgb]{0.63,0.63,0.00}{##1}}}
\expandafter\def\csname PY@tok@ni\endcsname{\let\PY@bf=\textbf\def\PY@tc##1{\textcolor[rgb]{0.60,0.60,0.60}{##1}}}
\expandafter\def\csname PY@tok@na\endcsname{\def\PY@tc##1{\textcolor[rgb]{0.49,0.56,0.16}{##1}}}
\expandafter\def\csname PY@tok@nt\endcsname{\let\PY@bf=\textbf\def\PY@tc##1{\textcolor[rgb]{0.00,0.50,0.00}{##1}}}
\expandafter\def\csname PY@tok@nd\endcsname{\def\PY@tc##1{\textcolor[rgb]{0.67,0.13,1.00}{##1}}}
\expandafter\def\csname PY@tok@s\endcsname{\def\PY@tc##1{\textcolor[rgb]{0.73,0.13,0.13}{##1}}}
\expandafter\def\csname PY@tok@sd\endcsname{\let\PY@it=\textit\def\PY@tc##1{\textcolor[rgb]{0.73,0.13,0.13}{##1}}}
\expandafter\def\csname PY@tok@si\endcsname{\let\PY@bf=\textbf\def\PY@tc##1{\textcolor[rgb]{0.73,0.40,0.53}{##1}}}
\expandafter\def\csname PY@tok@se\endcsname{\let\PY@bf=\textbf\def\PY@tc##1{\textcolor[rgb]{0.73,0.40,0.13}{##1}}}
\expandafter\def\csname PY@tok@sr\endcsname{\def\PY@tc##1{\textcolor[rgb]{0.73,0.40,0.53}{##1}}}
\expandafter\def\csname PY@tok@ss\endcsname{\def\PY@tc##1{\textcolor[rgb]{0.10,0.09,0.49}{##1}}}
\expandafter\def\csname PY@tok@sx\endcsname{\def\PY@tc##1{\textcolor[rgb]{0.00,0.50,0.00}{##1}}}
\expandafter\def\csname PY@tok@m\endcsname{\def\PY@tc##1{\textcolor[rgb]{0.40,0.40,0.40}{##1}}}
\expandafter\def\csname PY@tok@gh\endcsname{\let\PY@bf=\textbf\def\PY@tc##1{\textcolor[rgb]{0.00,0.00,0.50}{##1}}}
\expandafter\def\csname PY@tok@gu\endcsname{\let\PY@bf=\textbf\def\PY@tc##1{\textcolor[rgb]{0.50,0.00,0.50}{##1}}}
\expandafter\def\csname PY@tok@gd\endcsname{\def\PY@tc##1{\textcolor[rgb]{0.63,0.00,0.00}{##1}}}
\expandafter\def\csname PY@tok@gi\endcsname{\def\PY@tc##1{\textcolor[rgb]{0.00,0.63,0.00}{##1}}}
\expandafter\def\csname PY@tok@gr\endcsname{\def\PY@tc##1{\textcolor[rgb]{1.00,0.00,0.00}{##1}}}
\expandafter\def\csname PY@tok@ge\endcsname{\let\PY@it=\textit}
\expandafter\def\csname PY@tok@gs\endcsname{\let\PY@bf=\textbf}
\expandafter\def\csname PY@tok@gp\endcsname{\let\PY@bf=\textbf\def\PY@tc##1{\textcolor[rgb]{0.00,0.00,0.50}{##1}}}
\expandafter\def\csname PY@tok@go\endcsname{\def\PY@tc##1{\textcolor[rgb]{0.53,0.53,0.53}{##1}}}
\expandafter\def\csname PY@tok@gt\endcsname{\def\PY@tc##1{\textcolor[rgb]{0.00,0.27,0.87}{##1}}}
\expandafter\def\csname PY@tok@err\endcsname{\def\PY@bc##1{\setlength{\fboxsep}{0pt}\fcolorbox[rgb]{1.00,0.00,0.00}{1,1,1}{\strut ##1}}}
\expandafter\def\csname PY@tok@kc\endcsname{\let\PY@bf=\textbf\def\PY@tc##1{\textcolor[rgb]{0.00,0.50,0.00}{##1}}}
\expandafter\def\csname PY@tok@kd\endcsname{\let\PY@bf=\textbf\def\PY@tc##1{\textcolor[rgb]{0.00,0.50,0.00}{##1}}}
\expandafter\def\csname PY@tok@kn\endcsname{\let\PY@bf=\textbf\def\PY@tc##1{\textcolor[rgb]{0.00,0.50,0.00}{##1}}}
\expandafter\def\csname PY@tok@kr\endcsname{\let\PY@bf=\textbf\def\PY@tc##1{\textcolor[rgb]{0.00,0.50,0.00}{##1}}}
\expandafter\def\csname PY@tok@bp\endcsname{\def\PY@tc##1{\textcolor[rgb]{0.00,0.50,0.00}{##1}}}
\expandafter\def\csname PY@tok@fm\endcsname{\def\PY@tc##1{\textcolor[rgb]{0.00,0.00,1.00}{##1}}}
\expandafter\def\csname PY@tok@vc\endcsname{\def\PY@tc##1{\textcolor[rgb]{0.10,0.09,0.49}{##1}}}
\expandafter\def\csname PY@tok@vg\endcsname{\def\PY@tc##1{\textcolor[rgb]{0.10,0.09,0.49}{##1}}}
\expandafter\def\csname PY@tok@vi\endcsname{\def\PY@tc##1{\textcolor[rgb]{0.10,0.09,0.49}{##1}}}
\expandafter\def\csname PY@tok@vm\endcsname{\def\PY@tc##1{\textcolor[rgb]{0.10,0.09,0.49}{##1}}}
\expandafter\def\csname PY@tok@sa\endcsname{\def\PY@tc##1{\textcolor[rgb]{0.73,0.13,0.13}{##1}}}
\expandafter\def\csname PY@tok@sb\endcsname{\def\PY@tc##1{\textcolor[rgb]{0.73,0.13,0.13}{##1}}}
\expandafter\def\csname PY@tok@sc\endcsname{\def\PY@tc##1{\textcolor[rgb]{0.73,0.13,0.13}{##1}}}
\expandafter\def\csname PY@tok@dl\endcsname{\def\PY@tc##1{\textcolor[rgb]{0.73,0.13,0.13}{##1}}}
\expandafter\def\csname PY@tok@s2\endcsname{\def\PY@tc##1{\textcolor[rgb]{0.73,0.13,0.13}{##1}}}
\expandafter\def\csname PY@tok@sh\endcsname{\def\PY@tc##1{\textcolor[rgb]{0.73,0.13,0.13}{##1}}}
\expandafter\def\csname PY@tok@s1\endcsname{\def\PY@tc##1{\textcolor[rgb]{0.73,0.13,0.13}{##1}}}
\expandafter\def\csname PY@tok@mb\endcsname{\def\PY@tc##1{\textcolor[rgb]{0.40,0.40,0.40}{##1}}}
\expandafter\def\csname PY@tok@mf\endcsname{\def\PY@tc##1{\textcolor[rgb]{0.40,0.40,0.40}{##1}}}
\expandafter\def\csname PY@tok@mh\endcsname{\def\PY@tc##1{\textcolor[rgb]{0.40,0.40,0.40}{##1}}}
\expandafter\def\csname PY@tok@mi\endcsname{\def\PY@tc##1{\textcolor[rgb]{0.40,0.40,0.40}{##1}}}
\expandafter\def\csname PY@tok@il\endcsname{\def\PY@tc##1{\textcolor[rgb]{0.40,0.40,0.40}{##1}}}
\expandafter\def\csname PY@tok@mo\endcsname{\def\PY@tc##1{\textcolor[rgb]{0.40,0.40,0.40}{##1}}}
\expandafter\def\csname PY@tok@ch\endcsname{\let\PY@it=\textit\def\PY@tc##1{\textcolor[rgb]{0.25,0.50,0.50}{##1}}}
\expandafter\def\csname PY@tok@cm\endcsname{\let\PY@it=\textit\def\PY@tc##1{\textcolor[rgb]{0.25,0.50,0.50}{##1}}}
\expandafter\def\csname PY@tok@cpf\endcsname{\let\PY@it=\textit\def\PY@tc##1{\textcolor[rgb]{0.25,0.50,0.50}{##1}}}
\expandafter\def\csname PY@tok@c1\endcsname{\let\PY@it=\textit\def\PY@tc##1{\textcolor[rgb]{0.25,0.50,0.50}{##1}}}
\expandafter\def\csname PY@tok@cs\endcsname{\let\PY@it=\textit\def\PY@tc##1{\textcolor[rgb]{0.25,0.50,0.50}{##1}}}

\def\PYZbs{\char`\\}
\def\PYZus{\char`\_}
\def\PYZob{\char`\{}
\def\PYZcb{\char`\}}
\def\PYZca{\char`\^}
\def\PYZam{\char`\&}
\def\PYZlt{\char`\<}
\def\PYZgt{\char`\>}
\def\PYZsh{\char`\#}
\def\PYZpc{\char`\%}
\def\PYZdl{\char`\$}
\def\PYZhy{\char`\-}
\def\PYZsq{\char`\'}
\def\PYZdq{\char`\"}
\def\PYZti{\char`\~}
% for compatibility with earlier versions
\def\PYZat{@}
\def\PYZlb{[}
\def\PYZrb{]}
\makeatother


    % For linebreaks inside Verbatim environment from package fancyvrb. 
    \makeatletter
        \newbox\Wrappedcontinuationbox 
        \newbox\Wrappedvisiblespacebox 
        \newcommand*\Wrappedvisiblespace {\textcolor{red}{\textvisiblespace}} 
        \newcommand*\Wrappedcontinuationsymbol {\textcolor{red}{\llap{\tiny$\m@th\hookrightarrow$}}} 
        \newcommand*\Wrappedcontinuationindent {3ex } 
        \newcommand*\Wrappedafterbreak {\kern\Wrappedcontinuationindent\copy\Wrappedcontinuationbox} 
        % Take advantage of the already applied Pygments mark-up to insert 
        % potential linebreaks for TeX processing. 
        %        {, <, #, %, $, ' and ": go to next line. 
        %        _, }, ^, &, >, - and ~: stay at end of broken line. 
        % Use of \textquotesingle for straight quote. 
        \newcommand*\Wrappedbreaksatspecials {% 
            \def\PYGZus{\discretionary{\char`\_}{\Wrappedafterbreak}{\char`\_}}% 
            \def\PYGZob{\discretionary{}{\Wrappedafterbreak\char`\{}{\char`\{}}% 
            \def\PYGZcb{\discretionary{\char`\}}{\Wrappedafterbreak}{\char`\}}}% 
            \def\PYGZca{\discretionary{\char`\^}{\Wrappedafterbreak}{\char`\^}}% 
            \def\PYGZam{\discretionary{\char`\&}{\Wrappedafterbreak}{\char`\&}}% 
            \def\PYGZlt{\discretionary{}{\Wrappedafterbreak\char`\<}{\char`\<}}% 
            \def\PYGZgt{\discretionary{\char`\>}{\Wrappedafterbreak}{\char`\>}}% 
            \def\PYGZsh{\discretionary{}{\Wrappedafterbreak\char`\#}{\char`\#}}% 
            \def\PYGZpc{\discretionary{}{\Wrappedafterbreak\char`\%}{\char`\%}}% 
            \def\PYGZdl{\discretionary{}{\Wrappedafterbreak\char`\$}{\char`\$}}% 
            \def\PYGZhy{\discretionary{\char`\-}{\Wrappedafterbreak}{\char`\-}}% 
            \def\PYGZsq{\discretionary{}{\Wrappedafterbreak\textquotesingle}{\textquotesingle}}% 
            \def\PYGZdq{\discretionary{}{\Wrappedafterbreak\char`\"}{\char`\"}}% 
            \def\PYGZti{\discretionary{\char`\~}{\Wrappedafterbreak}{\char`\~}}% 
        } 
        % Some characters . , ; ? ! / are not pygmentized. 
        % This macro makes them "active" and they will insert potential linebreaks 
        \newcommand*\Wrappedbreaksatpunct {% 
            \lccode`\~`\.\lowercase{\def~}{\discretionary{\hbox{\char`\.}}{\Wrappedafterbreak}{\hbox{\char`\.}}}% 
            \lccode`\~`\,\lowercase{\def~}{\discretionary{\hbox{\char`\,}}{\Wrappedafterbreak}{\hbox{\char`\,}}}% 
            \lccode`\~`\;\lowercase{\def~}{\discretionary{\hbox{\char`\;}}{\Wrappedafterbreak}{\hbox{\char`\;}}}% 
            \lccode`\~`\:\lowercase{\def~}{\discretionary{\hbox{\char`\:}}{\Wrappedafterbreak}{\hbox{\char`\:}}}% 
            \lccode`\~`\?\lowercase{\def~}{\discretionary{\hbox{\char`\?}}{\Wrappedafterbreak}{\hbox{\char`\?}}}% 
            \lccode`\~`\!\lowercase{\def~}{\discretionary{\hbox{\char`\!}}{\Wrappedafterbreak}{\hbox{\char`\!}}}% 
            \lccode`\~`\/\lowercase{\def~}{\discretionary{\hbox{\char`\/}}{\Wrappedafterbreak}{\hbox{\char`\/}}}% 
            \catcode`\.\active
            \catcode`\,\active 
            \catcode`\;\active
            \catcode`\:\active
            \catcode`\?\active
            \catcode`\!\active
            \catcode`\/\active 
            \lccode`\~`\~ 	
        }
    \makeatother

    \let\OriginalVerbatim=\Verbatim
    \makeatletter
    \renewcommand{\Verbatim}[1][1]{%
        %\parskip\z@skip
        \sbox\Wrappedcontinuationbox {\Wrappedcontinuationsymbol}%
        \sbox\Wrappedvisiblespacebox {\FV@SetupFont\Wrappedvisiblespace}%
        \def\FancyVerbFormatLine ##1{\hsize\linewidth
            \vtop{\raggedright\hyphenpenalty\z@\exhyphenpenalty\z@
                \doublehyphendemerits\z@\finalhyphendemerits\z@
                \strut ##1\strut}%
        }%
        % If the linebreak is at a space, the latter will be displayed as visible
        % space at end of first line, and a continuation symbol starts next line.
        % Stretch/shrink are however usually zero for typewriter font.
        \def\FV@Space {%
            \nobreak\hskip\z@ plus\fontdimen3\font minus\fontdimen4\font
            \discretionary{\copy\Wrappedvisiblespacebox}{\Wrappedafterbreak}
            {\kern\fontdimen2\font}%
        }%
        
        % Allow breaks at special characters using \PYG... macros.
        \Wrappedbreaksatspecials
        % Breaks at punctuation characters . , ; ? ! and / need catcode=\active 	
        \OriginalVerbatim[#1,codes*=\Wrappedbreaksatpunct]%
    }
    \makeatother

    % Exact colors from NB
    \definecolor{incolor}{HTML}{303F9F}
    \definecolor{outcolor}{HTML}{D84315}
    \definecolor{cellborder}{HTML}{CFCFCF}
    \definecolor{cellbackground}{HTML}{F7F7F7}
    
    % prompt
    \makeatletter
    \newcommand{\boxspacing}{\kern\kvtcb@left@rule\kern\kvtcb@boxsep}
    \makeatother
    \newcommand{\prompt}[4]{
        {\ttfamily\llap{{\color{#2}[#3]:\hspace{3pt}#4}}\vspace{-\baselineskip}}
    }
    

    
    % Prevent overflowing lines due to hard-to-break entities
    \sloppy 
    % Setup hyperref package
    \hypersetup{
      breaklinks=true,  % so long urls are correctly broken across lines
      colorlinks=true,
      urlcolor=urlcolor,
      linkcolor=linkcolor,
      citecolor=citecolor,
      }
    % Slightly bigger margins than the latex defaults
    
    \geometry{verbose,tmargin=1in,bmargin=1in,lmargin=1in,rmargin=1in}
    
    

\begin{document}
    
    \maketitle
    
    

    
    \hypertarget{week6-random-simulation}{%
\section{Week6 Random Simulation}\label{week6-random-simulation}}

\hypertarget{ux80ccux666fux63cfux8ff0}{%
\subsection{背景描述}\label{ux80ccux666fux63cfux8ff0}}

设计一个模拟:实现欠拟合和过拟合对多元线性回归模型的影响。

在模拟中,我们可以定义\(Bias^2_k,\ Var_k,\ MSE_k\)分别为第 k
个线性回归模型的偏差平方、方差和均方误差。

\hypertarget{ux6570ux636eux63cfux8ff0}{%
\subsection{数据描述}\label{ux6570ux636eux63cfux8ff0}}

参数设置如下:

\begin{enumerate}
\def\labelenumi{\roman{enumi}.}
\item
  样本量\(n=300\);
\item
  变量维度 \((p, p_1) = (20, 10)\);
\item
  自变量的波动 \(\sigma_x = 0.2\);
\item
  自变量的相依程度 \(\rho_x = 0\);
\item
  误差的波动 \(\sigma_y = 3\);
\item
  预测点的位置 \(\pmb{x}_0 = (1, \pmb{0.05}^′_{20})^′\);
\item
  重复次数 \(M = 5000\)
\end{enumerate}

    \hypertarget{ux95eeux9898}{%
\subsection{问题}\label{ux95eeux9898}}

\begin{enumerate}
\def\labelenumi{(\alph{enumi})}
\item
  在同一张图上采用三种颜色绘制\(Bias^2_k\)、\(Var_k\) 和 \(MSE_k\)
  的三条曲线。
\item
  标示出 MSE 最小所对应的自变量个数。
\end{enumerate}

\hypertarget{ux89e3ux51b3ux65b9ux6848}{%
\subsection{解决方案}\label{ux89e3ux51b3ux65b9ux6848}}

根据作业中给出的要求,编写python代码如下。首先加载相应的库函数。

    \begin{tcolorbox}[breakable, size=fbox, boxrule=1pt, pad at break*=1mm,colback=cellbackground, colframe=cellborder]
\prompt{In}{incolor}{1}{\boxspacing}
\begin{Verbatim}[commandchars=\\\{\}]
\PY{k+kn}{import} \PY{n+nn}{numpy} \PY{k}{as} \PY{n+nn}{np}
\PY{k+kn}{import} \PY{n+nn}{random}
\PY{k+kn}{import} \PY{n+nn}{matplotlib}\PY{n+nn}{.}\PY{n+nn}{pyplot} \PY{k}{as} \PY{n+nn}{plt}
\end{Verbatim}
\end{tcolorbox}

    然后定义重要的函数。其中,epsilon函數用於生成向量\(\pmb{\epsilon}=(\epsilon_1,\epsilon_2,...,\epsilon_n)\),其中\(\epsilon_i\)独立同分布于正态分布\(N(0,\sigma^2_y)\)。y\_real函數用於根據給定的beta和生成X并構造y向量,公式是\(\pmb{y}=\pmb{X}\pmb{\beta}+\pmb{\epsilon}\),其中beta由參數指定,X在函數内部生成。

    \begin{tcolorbox}[breakable, size=fbox, boxrule=1pt, pad at break*=1mm,colback=cellbackground, colframe=cellborder]
\prompt{In}{incolor}{2}{\boxspacing}
\begin{Verbatim}[commandchars=\\\{\}]
\PY{k}{def} \PY{n+nf}{epsilon}\PY{p}{(}\PY{n}{n}\PY{p}{)}\PY{p}{:} \PY{c+c1}{\PYZsh{} 生成epsilon向量}
    \PY{k}{return} \PY{n}{np}\PY{o}{.}\PY{n}{mat}\PY{p}{(}\PY{p}{[}\PY{n}{random}\PY{o}{.}\PY{n}{gauss}\PY{p}{(}\PY{l+m+mi}{0}\PY{p}{,}\PY{n}{sig\PYZus{}y}\PY{p}{)} \PY{k}{for} \PY{n}{i} \PY{o+ow}{in} \PY{n+nb}{range}\PY{p}{(}\PY{n}{n}\PY{p}{)}\PY{p}{]}\PY{p}{)}\PY{o}{.}\PY{n}{T}

\PY{k}{def} \PY{n+nf}{y\PYZus{}real}\PY{p}{(}\PY{n}{beta}\PY{p}{,} \PY{n}{n}\PY{p}{)}\PY{p}{:} \PY{c+c1}{\PYZsh{} 生成X和y }
    \PY{n}{x} \PY{o}{=} \PY{p}{[}\PY{p}{]}
    \PY{k}{for} \PY{n}{j} \PY{o+ow}{in} \PY{n+nb}{range}\PY{p}{(}\PY{n}{n}\PY{p}{)}\PY{p}{:}
        \PY{n}{x\PYZus{}l} \PY{o}{=} \PY{p}{[}\PY{n}{random}\PY{o}{.}\PY{n}{gauss}\PY{p}{(}\PY{l+m+mi}{0}\PY{p}{,}\PY{n}{sig\PYZus{}x}\PY{p}{)} \PY{k}{for} \PY{n}{i} \PY{o+ow}{in} \PY{n+nb}{range}\PY{p}{(}\PY{n}{p} \PY{o}{+} \PY{l+m+mi}{1}\PY{p}{)}\PY{p}{]}
        \PY{n}{x\PYZus{}l}\PY{p}{[}\PY{l+m+mi}{0}\PY{p}{]} \PY{o}{=} \PY{l+m+mf}{1.0} \PY{c+c1}{\PYZsh{} 生成矩阵X的每一行xi,第一列恒为1}
        \PY{n}{x}\PY{o}{.}\PY{n}{append}\PY{p}{(}\PY{n}{x\PYZus{}l}\PY{p}{)}
    \PY{n}{x} \PY{o}{=} \PY{n}{np}\PY{o}{.}\PY{n}{mat}\PY{p}{(}\PY{n}{x}\PY{p}{)}
    \PY{n}{y} \PY{o}{=} \PY{n}{np}\PY{o}{.}\PY{n}{matmul}\PY{p}{(}\PY{n}{x}\PY{p}{,} \PY{n}{beta}\PY{p}{)} \PY{o}{+} \PY{n}{epsilon}\PY{p}{(}\PY{n}{n}\PY{p}{)} \PY{c+c1}{\PYZsh{} y的计算公式}
    \PY{k}{return} \PY{n}{y}\PY{p}{,} \PY{n}{x}
\end{Verbatim}
\end{tcolorbox}

    然后设置合适的参数并进行初始化,其中\(\pmb{\beta}\)的前\(p_1\)个分量的值爲1,其餘的值爲0。

    \begin{tcolorbox}[breakable, size=fbox, boxrule=1pt, pad at break*=1mm,colback=cellbackground, colframe=cellborder]
\prompt{In}{incolor}{3}{\boxspacing}
\begin{Verbatim}[commandchars=\\\{\}]
\PY{n}{n} \PY{o}{=} \PY{l+m+mi}{300}
\PY{n}{p} \PY{o}{=} \PY{l+m+mi}{20}
\PY{n}{p1} \PY{o}{=} \PY{n+nb}{int}\PY{p}{(}\PY{n}{p} \PY{o}{*} \PY{l+m+mf}{0.5}\PY{p}{)}
\PY{n}{sig\PYZus{}x} \PY{o}{=} \PY{l+m+mf}{0.2}
\PY{n}{sig\PYZus{}y} \PY{o}{=} \PY{l+m+mi}{3}
\PY{n}{x0} \PY{o}{=} \PY{p}{[}\PY{l+m+mi}{1} \PY{k}{if} \PY{n}{i} \PY{o}{==} \PY{l+m+mi}{0} \PY{k}{else} \PY{l+m+mf}{0.05} \PY{k}{for} \PY{n}{i} \PY{o+ow}{in} \PY{n+nb}{range}\PY{p}{(}\PY{n}{p} \PY{o}{+} \PY{l+m+mi}{1}\PY{p}{)}\PY{p}{]}
\PY{n}{M} \PY{o}{=} \PY{l+m+mi}{5000} 
\PY{n}{beta} \PY{o}{=} \PY{n}{np}\PY{o}{.}\PY{n}{mat}\PY{p}{(}\PY{p}{[}\PY{l+m+mi}{1} \PY{k}{if} \PY{n}{i} \PY{o}{\PYZlt{}}\PY{o}{=} \PY{n}{p1} \PY{k}{else} \PY{l+m+mi}{0} \PY{k}{for} \PY{n}{i} \PY{o+ow}{in} \PY{n+nb}{range}\PY{p}{(}\PY{n}{p} \PY{o}{+} \PY{l+m+mi}{1}\PY{p}{)}\PY{p}{]}\PY{p}{)}\PY{o}{.}\PY{n}{T}
\PY{n}{y0\PYZus{}e} \PY{o}{=} \PY{n}{np}\PY{o}{.}\PY{n}{matmul}\PY{p}{(}\PY{n}{x0}\PY{p}{,} \PY{n}{beta}\PY{p}{)} \PY{c+c1}{\PYZsh{} 这个变量是在计算偏差平方和均方误差时所需的x0beta}
\PY{n}{bia} \PY{o}{=} \PY{p}{[}\PY{p}{[}\PY{p}{]} \PY{k}{for} \PY{n}{i} \PY{o+ow}{in} \PY{n+nb}{range}\PY{p}{(}\PY{n}{p} \PY{o}{+} \PY{l+m+mi}{1}\PY{p}{)}\PY{p}{]} \PY{c+c1}{\PYZsh{} 记录每一次的y0\PYZus{}hat\PYZhy{}x0beta}
\PY{n}{y0\PYZus{}hats} \PY{o}{=} \PY{p}{[}\PY{p}{[}\PY{p}{]} \PY{k}{for} \PY{n}{i} \PY{o+ow}{in} \PY{n+nb}{range}\PY{p}{(}\PY{n}{p} \PY{o}{+} \PY{l+m+mi}{1}\PY{p}{)}\PY{p}{]} \PY{c+c1}{\PYZsh{} 记录每一次的y0\PYZus{}hat}
\end{Verbatim}
\end{tcolorbox}

    对于每一轮,利用公式\(\hat{\pmb{\beta}}=(\pmb{X}'\pmb{X})^{-1}\pmb{X}'\pmb{y}\)計算出\(\hat{\pmb{\beta}}\)的值,然後利用公式\(\hat{y_0}^{(k)}=\pmb{x_0}'\hat{\pmb{\beta}}^{(k)}\)計算出y0的估計值,然後把它和\(\hat{y_0}^{(k)}-\pmb{x}_0'\pmb{\beta}\)記録到列表中。

    \begin{tcolorbox}[breakable, size=fbox, boxrule=1pt, pad at break*=1mm,colback=cellbackground, colframe=cellborder]
\prompt{In}{incolor}{4}{\boxspacing}
\begin{Verbatim}[commandchars=\\\{\}]
\PY{k}{for} \PY{n}{m} \PY{o+ow}{in} \PY{n+nb}{range}\PY{p}{(}\PY{n}{M}\PY{p}{)}\PY{p}{:}
    \PY{n}{y}\PY{p}{,} \PY{n}{x} \PY{o}{=} \PY{n}{y\PYZus{}real}\PY{p}{(}\PY{n}{beta}\PY{p}{,}  \PY{n}{n}\PY{p}{)}
    \PY{n}{re} \PY{o}{=} \PY{n}{np}\PY{o}{.}\PY{n}{matmul}\PY{p}{(}\PY{n}{x}\PY{o}{.}\PY{n}{T}\PY{p}{,}\PY{n}{x}\PY{p}{)}\PY{o}{.}\PY{n}{I}
    \PY{n}{beta\PYZus{}hat} \PY{o}{=} \PY{n}{np}\PY{o}{.}\PY{n}{matmul}\PY{p}{(}\PY{n}{np}\PY{o}{.}\PY{n}{matmul}\PY{p}{(}\PY{n}{re}\PY{p}{,}\PY{n}{x}\PY{o}{.}\PY{n}{T}\PY{p}{)}\PY{p}{,}\PY{n}{y}\PY{p}{)}
    \PY{n}{beta\PYZus{}hat\PYZus{}k} \PY{o}{=} \PY{n}{beta\PYZus{}hat}
    \PY{k}{for} \PY{n}{k} \PY{o+ow}{in} \PY{n+nb}{range}\PY{p}{(}\PY{n}{p} \PY{o}{+} \PY{l+m+mi}{1}\PY{p}{)}\PY{p}{:} \PY{c+c1}{\PYZsh{} 每一次}
        \PY{n}{y0\PYZus{}hat} \PY{o}{=} \PY{n}{np}\PY{o}{.}\PY{n}{matmul}\PY{p}{(}\PY{n}{x0}\PY{p}{,} \PY{n}{beta\PYZus{}hat\PYZus{}k}\PY{p}{)}
        \PY{n}{bi} \PY{o}{=} \PY{n+nb}{float}\PY{p}{(}\PY{n}{y0\PYZus{}hat} \PY{o}{\PYZhy{}} \PY{n}{y0\PYZus{}e}\PY{p}{)}
        \PY{n}{bia}\PY{p}{[}\PY{n}{p} \PY{o}{\PYZhy{}} \PY{n}{k}\PY{p}{]}\PY{o}{.}\PY{n}{append}\PY{p}{(}\PY{n}{bi}\PY{p}{)}
        \PY{n}{y0\PYZus{}hats}\PY{p}{[}\PY{n}{p} \PY{o}{\PYZhy{}} \PY{n}{k}\PY{p}{]}\PY{o}{.}\PY{n}{append}\PY{p}{(}\PY{n+nb}{float}\PY{p}{(}\PY{n}{y0\PYZus{}hat}\PY{p}{)}\PY{p}{)}
        \PY{n}{beta\PYZus{}hat\PYZus{}k}\PY{p}{[}\PY{n}{p} \PY{o}{\PYZhy{}} \PY{n}{k}\PY{p}{]} \PY{o}{=} \PY{l+m+mi}{0}
\end{Verbatim}
\end{tcolorbox}

    最后计算出\(Bias^2_k,\ Var_k,\ MSE_k\)并打印出MSE的值和图表。

    \begin{tcolorbox}[breakable, size=fbox, boxrule=1pt, pad at break*=1mm,colback=cellbackground, colframe=cellborder]
\prompt{In}{incolor}{5}{\boxspacing}
\begin{Verbatim}[commandchars=\\\{\}]
\PY{n}{bias} \PY{o}{=} \PY{p}{[}\PY{p}{]}
\PY{n}{mses} \PY{o}{=} \PY{p}{[}\PY{p}{]}
\PY{k}{for} \PY{n}{k} \PY{o+ow}{in} \PY{n+nb}{range}\PY{p}{(}\PY{n}{p} \PY{o}{+} \PY{l+m+mi}{1}\PY{p}{)}\PY{p}{:}
    \PY{n}{y0\PYZus{}hat\PYZus{}l} \PY{o}{=} \PY{n}{np}\PY{o}{.}\PY{n}{array}\PY{p}{(}\PY{n}{y0\PYZus{}hats}\PY{p}{[}\PY{n}{k}\PY{p}{]}\PY{p}{)}
    \PY{n}{bis} \PY{o}{=} \PY{n}{np}\PY{o}{.}\PY{n}{array}\PY{p}{(}\PY{n}{bia}\PY{p}{[}\PY{n}{k}\PY{p}{]}\PY{p}{)}
    \PY{n}{biask} \PY{o}{=} \PY{n+nb}{float}\PY{p}{(}\PY{n}{y0\PYZus{}hat\PYZus{}l}\PY{o}{.}\PY{n}{mean}\PY{p}{(}\PY{p}{)} \PY{o}{\PYZhy{}} \PY{n}{y0\PYZus{}e}\PY{p}{)}
    \PY{n}{bias}\PY{o}{.}\PY{n}{append}\PY{p}{(}\PY{n}{biask} \PY{o}{*} \PY{n}{biask}\PY{p}{)}
    \PY{n}{mse} \PY{o}{=} \PY{n}{bis} \PY{o}{*} \PY{n}{bis}
    \PY{n}{mses}\PY{o}{.}\PY{n}{append}\PY{p}{(}\PY{n}{mse}\PY{o}{.}\PY{n}{mean}\PY{p}{(}\PY{p}{)}\PY{p}{)}
\PY{n+nb}{print}\PY{p}{(}\PY{l+s+s2}{\PYZdq{}}\PY{l+s+s2}{MSE}\PY{l+s+s2}{\PYZdq{}}\PY{p}{)}
\PY{k}{for} \PY{n}{k} \PY{o+ow}{in} \PY{n+nb}{range}\PY{p}{(}\PY{n}{p} \PY{o}{+} \PY{l+m+mi}{1}\PY{p}{)}\PY{p}{:}
    \PY{n+nb}{print}\PY{p}{(}\PY{n}{k}\PY{p}{,} \PY{l+s+s1}{\PYZsq{}}\PY{l+s+s1}{:}\PY{l+s+s1}{\PYZsq{}}\PY{p}{,} \PY{n+nb}{round}\PY{p}{(}\PY{n}{mses}\PY{p}{[}\PY{n}{k}\PY{p}{]}\PY{p}{,} \PY{l+m+mi}{4}\PY{p}{)}\PY{p}{)}
\end{Verbatim}
\end{tcolorbox}

    \begin{Verbatim}[commandchars=\\\{\}]
MSE
0 : 0.2819
1 : 0.2364
2 : 0.1952
3 : 0.1597
4 : 0.1298
5 : 0.1048
6 : 0.0847
7 : 0.0695
8 : 0.059
9 : 0.0535
10 : 0.0526
11 : 0.0544
12 : 0.057
13 : 0.0592
14 : 0.0609
15 : 0.0629
16 : 0.0643
17 : 0.0666
18 : 0.0683
19 : 0.0697
20 : 0.0717
    \end{Verbatim}

    \begin{tcolorbox}[breakable, size=fbox, boxrule=1pt, pad at break*=1mm,colback=cellbackground, colframe=cellborder]
\prompt{In}{incolor}{6}{\boxspacing}
\begin{Verbatim}[commandchars=\\\{\}]
\PY{n}{T} \PY{o}{=} \PY{n}{np}\PY{o}{.}\PY{n}{array}\PY{p}{(}\PY{p}{[}\PY{n}{x} \PY{k}{for} \PY{n}{x} \PY{o+ow}{in} \PY{n+nb}{range}\PY{p}{(}\PY{n}{p}\PY{o}{+}\PY{l+m+mi}{1}\PY{p}{)}\PY{p}{]}\PY{p}{)}
\PY{n}{l1} \PY{o}{=} \PY{n}{np}\PY{o}{.}\PY{n}{array}\PY{p}{(}\PY{n}{bias}\PY{p}{)}
\PY{n}{l3} \PY{o}{=} \PY{n}{np}\PY{o}{.}\PY{n}{array}\PY{p}{(}\PY{n}{mses}\PY{p}{)}
\PY{n}{l2} \PY{o}{=} \PY{n}{l3} \PY{o}{\PYZhy{}} \PY{n}{l1}
\PY{n}{plt}\PY{o}{.}\PY{n}{plot}\PY{p}{(}\PY{n}{T}\PY{p}{,} \PY{n}{l1}\PY{p}{,} \PY{n}{label}\PY{o}{=}\PY{l+s+s2}{\PYZdq{}}\PY{l+s+s2}{Bias\PYZca{}2}\PY{l+s+s2}{\PYZdq{}}\PY{p}{)}
\PY{n}{plt}\PY{o}{.}\PY{n}{plot}\PY{p}{(}\PY{n}{T}\PY{p}{,} \PY{n}{l2}\PY{p}{,} \PY{n}{label}\PY{o}{=}\PY{l+s+s2}{\PYZdq{}}\PY{l+s+s2}{Var}\PY{l+s+s2}{\PYZdq{}}\PY{p}{)}
\PY{n}{plt}\PY{o}{.}\PY{n}{plot}\PY{p}{(}\PY{n}{T}\PY{p}{,} \PY{n}{l3}\PY{p}{,} \PY{n}{label}\PY{o}{=}\PY{l+s+s2}{\PYZdq{}}\PY{l+s+s2}{MSE}\PY{l+s+s2}{\PYZdq{}}\PY{p}{)}
\PY{n}{plt}\PY{o}{.}\PY{n}{legend}\PY{p}{(}\PY{p}{)}
\PY{n}{plt}\PY{o}{.}\PY{n}{show}\PY{p}{(}\PY{p}{)}
\end{Verbatim}
\end{tcolorbox}

    \begin{center}
    \adjustimage{max size={0.9\linewidth}{0.9\paperheight}}{output_11_0.png}
    \end{center}
    { \hspace*{\fill} \\}
    
    由此我们可以看出,随着k的不断增加,偏差平方非线性减小,在k=10处减小至接近0;方差不断增大,大致呈线性。均方误差先增大后减小,并在k=10处取得最小值。在k\textless10时,所选参数数量过少,为欠拟合,此时偏差较大,但方差较小。在k\textgreater10时,所选参数数量过多,为过拟合,此时基本没有偏差,但方差较大。在k=10处,所选参数和原有的参数恰好相等,均方误差最小,为拟合最好的点。

    此外,我还尝试了改变向量beta和x0。将向量beta和x0的每个分量改为每次由高斯分布随机生成,打印出的图表如下图。

    \begin{tcolorbox}[breakable, size=fbox, boxrule=1pt, pad at break*=1mm,colback=cellbackground, colframe=cellborder]
\prompt{In}{incolor}{7}{\boxspacing}
\begin{Verbatim}[commandchars=\\\{\}]
\PY{k}{def} \PY{n+nf}{getmean}\PY{p}{(}\PY{n}{l\PYZus{}all}\PY{p}{)}\PY{p}{:}
    \PY{n}{l\PYZus{}all} \PY{o}{=} \PY{n}{np}\PY{o}{.}\PY{n}{array}\PY{p}{(}\PY{n}{l\PYZus{}all}\PY{p}{)}\PY{o}{.}\PY{n}{T}
    \PY{n}{l} \PY{o}{=} \PY{p}{[}\PY{p}{]}
    \PY{k}{for} \PY{n}{k} \PY{o+ow}{in} \PY{n}{l\PYZus{}all}\PY{p}{:}
        \PY{n}{l}\PY{o}{.}\PY{n}{append}\PY{p}{(}\PY{n}{k}\PY{o}{.}\PY{n}{mean}\PY{p}{(}\PY{p}{)}\PY{p}{)}
    \PY{k}{return} \PY{n}{np}\PY{o}{.}\PY{n}{array}\PY{p}{(}\PY{n}{l}\PY{p}{)}

\PY{n}{n} \PY{o}{=} \PY{l+m+mi}{300}
\PY{n}{p} \PY{o}{=} \PY{l+m+mi}{20}
\PY{n}{p1} \PY{o}{=} \PY{n+nb}{int}\PY{p}{(}\PY{n}{p} \PY{o}{*} \PY{l+m+mf}{0.5}\PY{p}{)}
\PY{n}{sig\PYZus{}x} \PY{o}{=} \PY{l+m+mf}{0.2}
\PY{n}{sig\PYZus{}y} \PY{o}{=} \PY{l+m+mi}{3}
\PY{n}{M} \PY{o}{=} \PY{l+m+mi}{4000}
\PY{n}{M1} \PY{o}{=} \PY{l+m+mi}{100}
\PY{n}{l1\PYZus{}all} \PY{o}{=} \PY{p}{[}\PY{p}{]}
\PY{n}{l2\PYZus{}all} \PY{o}{=} \PY{p}{[}\PY{p}{]}
\PY{n}{l3\PYZus{}all} \PY{o}{=} \PY{p}{[}\PY{p}{]}
\PY{k}{for} \PY{n}{m1} \PY{o+ow}{in} \PY{n+nb}{range}\PY{p}{(}\PY{n}{M1}\PY{p}{)}\PY{p}{:}
    \PY{n}{x0} \PY{o}{=} \PY{p}{[}\PY{l+m+mi}{1} \PY{k}{if} \PY{n}{i} \PY{o}{==} \PY{l+m+mi}{0} \PY{k}{else} \PY{n}{random}\PY{o}{.}\PY{n}{gauss}\PY{p}{(}\PY{l+m+mf}{0.05}\PY{p}{,}\PY{l+m+mf}{0.05}\PY{p}{)} \PY{k}{for} \PY{n}{i} \PY{o+ow}{in} \PY{n+nb}{range}\PY{p}{(}\PY{n}{p} \PY{o}{+} \PY{l+m+mi}{1}\PY{p}{)}\PY{p}{]}
    \PY{n}{beta} \PY{o}{=} \PY{n}{np}\PY{o}{.}\PY{n}{mat}\PY{p}{(}\PY{p}{[}\PY{n}{random}\PY{o}{.}\PY{n}{gauss}\PY{p}{(}\PY{l+m+mi}{1}\PY{p}{,}\PY{l+m+mi}{1}\PY{p}{)} \PY{k}{if} \PY{n}{i} \PY{o}{\PYZlt{}}\PY{o}{=} \PY{n}{p1} \PY{k}{else} \PY{l+m+mi}{0} \PY{k}{for} \PY{n}{i} \PY{o+ow}{in} \PY{n+nb}{range}\PY{p}{(}\PY{n}{p} \PY{o}{+} \PY{l+m+mi}{1}\PY{p}{)}\PY{p}{]}\PY{p}{)}\PY{o}{.}\PY{n}{T}
    \PY{n}{y0\PYZus{}e} \PY{o}{=} \PY{n}{np}\PY{o}{.}\PY{n}{matmul}\PY{p}{(}\PY{n}{x0}\PY{p}{,} \PY{n}{beta}\PY{p}{)}
    \PY{n}{bia} \PY{o}{=} \PY{p}{[}\PY{p}{[}\PY{p}{]} \PY{k}{for} \PY{n}{i} \PY{o+ow}{in} \PY{n+nb}{range}\PY{p}{(}\PY{n}{p} \PY{o}{+} \PY{l+m+mi}{1}\PY{p}{)}\PY{p}{]}
    \PY{n}{y0\PYZus{}hats} \PY{o}{=} \PY{p}{[}\PY{p}{[}\PY{p}{]} \PY{k}{for} \PY{n}{i} \PY{o+ow}{in} \PY{n+nb}{range}\PY{p}{(}\PY{n}{p} \PY{o}{+} \PY{l+m+mi}{1}\PY{p}{)}\PY{p}{]}
    \PY{k}{for} \PY{n}{m} \PY{o+ow}{in} \PY{n+nb}{range}\PY{p}{(}\PY{n+nb}{int}\PY{p}{(}\PY{n}{M}\PY{o}{/}\PY{n}{M1}\PY{p}{)}\PY{p}{)}\PY{p}{:}
        \PY{n}{y}\PY{p}{,} \PY{n}{x} \PY{o}{=} \PY{n}{y\PYZus{}real}\PY{p}{(}\PY{n}{beta}\PY{p}{,}  \PY{n}{n}\PY{p}{)}
        \PY{n}{re} \PY{o}{=} \PY{n}{np}\PY{o}{.}\PY{n}{matmul}\PY{p}{(}\PY{n}{x}\PY{o}{.}\PY{n}{T}\PY{p}{,}\PY{n}{x}\PY{p}{)}\PY{o}{.}\PY{n}{I}
        \PY{n}{beta\PYZus{}hat} \PY{o}{=} \PY{n}{np}\PY{o}{.}\PY{n}{matmul}\PY{p}{(}\PY{n}{np}\PY{o}{.}\PY{n}{matmul}\PY{p}{(}\PY{n}{re}\PY{p}{,}\PY{n}{x}\PY{o}{.}\PY{n}{T}\PY{p}{)}\PY{p}{,}\PY{n}{y}\PY{p}{)}
        \PY{n}{beta\PYZus{}hat\PYZus{}k} \PY{o}{=} \PY{n}{beta\PYZus{}hat}
        \PY{k}{for} \PY{n}{k} \PY{o+ow}{in} \PY{n+nb}{range}\PY{p}{(}\PY{n}{p} \PY{o}{+} \PY{l+m+mi}{1}\PY{p}{)}\PY{p}{:}
            \PY{n}{y0\PYZus{}hat} \PY{o}{=} \PY{n}{np}\PY{o}{.}\PY{n}{matmul}\PY{p}{(}\PY{n}{x0}\PY{p}{,} \PY{n}{beta\PYZus{}hat\PYZus{}k}\PY{p}{)}
            \PY{n}{bi} \PY{o}{=} \PY{n+nb}{float}\PY{p}{(}\PY{n}{y0\PYZus{}hat} \PY{o}{\PYZhy{}} \PY{n}{y0\PYZus{}e}\PY{p}{)}
            \PY{n}{bia}\PY{p}{[}\PY{n}{p} \PY{o}{\PYZhy{}} \PY{n}{k}\PY{p}{]}\PY{o}{.}\PY{n}{append}\PY{p}{(}\PY{n}{bi}\PY{p}{)}
            \PY{n}{y0\PYZus{}hats}\PY{p}{[}\PY{n}{p} \PY{o}{\PYZhy{}} \PY{n}{k}\PY{p}{]}\PY{o}{.}\PY{n}{append}\PY{p}{(}\PY{n+nb}{float}\PY{p}{(}\PY{n}{y0\PYZus{}hat}\PY{p}{)}\PY{p}{)}
            \PY{n}{beta\PYZus{}hat\PYZus{}k}\PY{p}{[}\PY{n}{p} \PY{o}{\PYZhy{}} \PY{n}{k}\PY{p}{]} \PY{o}{=} \PY{l+m+mi}{0}

    \PY{n}{bias} \PY{o}{=} \PY{p}{[}\PY{p}{]}
    \PY{n}{mses} \PY{o}{=} \PY{p}{[}\PY{p}{]}
    \PY{k}{for} \PY{n}{k} \PY{o+ow}{in} \PY{n+nb}{range}\PY{p}{(}\PY{n}{p} \PY{o}{+} \PY{l+m+mi}{1}\PY{p}{)}\PY{p}{:}
        \PY{n}{y0\PYZus{}hat\PYZus{}l} \PY{o}{=} \PY{n}{np}\PY{o}{.}\PY{n}{array}\PY{p}{(}\PY{n}{y0\PYZus{}hats}\PY{p}{[}\PY{n}{k}\PY{p}{]}\PY{p}{)}
        \PY{n}{bis} \PY{o}{=} \PY{n}{np}\PY{o}{.}\PY{n}{array}\PY{p}{(}\PY{n}{bia}\PY{p}{[}\PY{n}{k}\PY{p}{]}\PY{p}{)}
        \PY{n}{biask} \PY{o}{=} \PY{n+nb}{float}\PY{p}{(}\PY{n}{y0\PYZus{}hat\PYZus{}l}\PY{o}{.}\PY{n}{mean}\PY{p}{(}\PY{p}{)} \PY{o}{\PYZhy{}} \PY{n}{y0\PYZus{}e}\PY{p}{)}
        \PY{n}{bias}\PY{o}{.}\PY{n}{append}\PY{p}{(}\PY{n}{biask} \PY{o}{*} \PY{n}{biask}\PY{p}{)}
        \PY{n}{mse} \PY{o}{=} \PY{n}{bis} \PY{o}{*} \PY{n}{bis}
        \PY{n}{mses}\PY{o}{.}\PY{n}{append}\PY{p}{(}\PY{n}{mse}\PY{o}{.}\PY{n}{mean}\PY{p}{(}\PY{p}{)}\PY{p}{)}

    \PY{n}{l1} \PY{o}{=} \PY{n}{np}\PY{o}{.}\PY{n}{array}\PY{p}{(}\PY{n}{bias}\PY{p}{)}
    \PY{n}{l3} \PY{o}{=} \PY{n}{np}\PY{o}{.}\PY{n}{array}\PY{p}{(}\PY{n}{mses}\PY{p}{)}
    \PY{n}{l2} \PY{o}{=} \PY{n}{l3} \PY{o}{\PYZhy{}} \PY{n}{l1}
    \PY{n}{l1\PYZus{}all}\PY{o}{.}\PY{n}{append}\PY{p}{(}\PY{n}{l1}\PY{p}{)}
    \PY{n}{l2\PYZus{}all}\PY{o}{.}\PY{n}{append}\PY{p}{(}\PY{n}{l2}\PY{p}{)}
    \PY{n}{l3\PYZus{}all}\PY{o}{.}\PY{n}{append}\PY{p}{(}\PY{n}{l3}\PY{p}{)}

\PY{n}{l1\PYZus{}f} \PY{o}{=} \PY{n}{getmean}\PY{p}{(}\PY{n}{l1\PYZus{}all}\PY{p}{)}
\PY{n}{l2\PYZus{}f} \PY{o}{=} \PY{n}{getmean}\PY{p}{(}\PY{n}{l2\PYZus{}all}\PY{p}{)}
\PY{n}{l3\PYZus{}f} \PY{o}{=} \PY{n}{getmean}\PY{p}{(}\PY{n}{l3\PYZus{}all}\PY{p}{)}
\PY{n}{T} \PY{o}{=} \PY{n}{np}\PY{o}{.}\PY{n}{array}\PY{p}{(}\PY{p}{[}\PY{n}{x} \PY{k}{for} \PY{n}{x} \PY{o+ow}{in} \PY{n+nb}{range}\PY{p}{(}\PY{n}{p} \PY{o}{+} \PY{l+m+mi}{1}\PY{p}{)}\PY{p}{]}\PY{p}{)}
\PY{n}{plt}\PY{o}{.}\PY{n}{plot}\PY{p}{(}\PY{n}{T}\PY{p}{,} \PY{n}{l1\PYZus{}f}\PY{p}{,} \PY{n}{label}\PY{o}{=}\PY{l+s+s2}{\PYZdq{}}\PY{l+s+s2}{Bias\PYZca{}2}\PY{l+s+s2}{\PYZdq{}}\PY{p}{)}
\PY{n}{plt}\PY{o}{.}\PY{n}{plot}\PY{p}{(}\PY{n}{T}\PY{p}{,} \PY{n}{l2\PYZus{}f}\PY{p}{,} \PY{n}{label}\PY{o}{=}\PY{l+s+s2}{\PYZdq{}}\PY{l+s+s2}{Var}\PY{l+s+s2}{\PYZdq{}}\PY{p}{)}
\PY{n}{plt}\PY{o}{.}\PY{n}{plot}\PY{p}{(}\PY{n}{T}\PY{p}{,} \PY{n}{l3\PYZus{}f}\PY{p}{,} \PY{n}{label}\PY{o}{=}\PY{l+s+s2}{\PYZdq{}}\PY{l+s+s2}{MSE}\PY{l+s+s2}{\PYZdq{}}\PY{p}{)}
\PY{n}{plt}\PY{o}{.}\PY{n}{legend}\PY{p}{(}\PY{p}{)}
\PY{n}{plt}\PY{o}{.}\PY{n}{show}\PY{p}{(}\PY{p}{)}
\end{Verbatim}
\end{tcolorbox}

    \begin{center}
    \adjustimage{max size={0.9\linewidth}{0.9\paperheight}}{output_14_0.png}
    \end{center}
    { \hspace*{\fill} \\}
    
    通过和前面图表的比较可以看出,一方面这种情况下得到的图表中的三条曲线不如beta和x0固定时得到的平滑。另一方面根据纵轴的显示,三种评价指标的值同等增大了一些。这可能是因为引入了更多的随机元素导致的。可以预测,如果进行足够多轮的随机模拟,也可以得到相对平滑的曲线。


    % Add a bibliography block to the postdoc
    
    
    
\end{document}
