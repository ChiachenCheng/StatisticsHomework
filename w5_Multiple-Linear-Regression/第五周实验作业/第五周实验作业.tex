%!TEX program = xelatex
\documentclass[11pt]{ctexart}

    \usepackage[breakable]{tcolorbox}
    \usepackage{parskip} % Stop auto-indenting (to mimic markdown behaviour)
    
    \usepackage{iftex}
    \ifPDFTeX
    	\usepackage[T1]{fontenc}
    	\usepackage{mathpazo}
    \else
    	\usepackage{fontspec}
    \fi

    % Basic figure setup, for now with no caption control since it's done
    % automatically by Pandoc (which extracts ![](path) syntax from Markdown).
    \usepackage{graphicx}
    % Maintain compatibility with old templates. Remove in nbconvert 6.0
    \let\Oldincludegraphics\includegraphics
    % Ensure that by default, figures have no caption (until we provide a
    % proper Figure object with a Caption API and a way to capture that
    % in the conversion process - todo).
    \usepackage{caption}
    \DeclareCaptionFormat{nocaption}{}
    \captionsetup{format=nocaption,aboveskip=0pt,belowskip=0pt}

    \usepackage{float}
    \floatplacement{figure}{H} % forces figures to be placed at the correct location
    \usepackage{xcolor} % Allow colors to be defined
    \usepackage{enumerate} % Needed for markdown enumerations to work
    \usepackage{geometry} % Used to adjust the document margins
    \usepackage{amsmath} % Equations
    \usepackage{amssymb} % Equations
    \usepackage{textcomp} % defines textquotesingle
    % Hack from http://tex.stackexchange.com/a/47451/13684:
    \AtBeginDocument{%
        \def\PYZsq{\textquotesingle}% Upright quotes in Pygmentized code
    }
    \usepackage{upquote} % Upright quotes for verbatim code
    \usepackage{eurosym} % defines \euro
    \usepackage[mathletters]{ucs} % Extended unicode (utf-8) support
    \usepackage{fancyvrb} % verbatim replacement that allows latex
    \usepackage{grffile} % extends the file name processing of package graphics 
                         % to support a larger range
    \makeatletter % fix for old versions of grffile with XeLaTeX
    \@ifpackagelater{grffile}{2019/11/01}
    {
      % Do nothing on new versions
    }
    {
      \def\Gread@@xetex#1{%
        \IfFileExists{"\Gin@base".bb}%
        {\Gread@eps{\Gin@base.bb}}%
        {\Gread@@xetex@aux#1}%
      }
    }
    \makeatother
    \usepackage[Export]{adjustbox} % Used to constrain images to a maximum size
    \adjustboxset{max size={0.9\linewidth}{0.9\paperheight}}

    % The hyperref package gives us a pdf with properly built
    % internal navigation ('pdf bookmarks' for the table of contents,
    % internal cross-reference links, web links for URLs, etc.)
    \usepackage{hyperref}
    % The default LaTeX title has an obnoxious amount of whitespace. By default,
    % titling removes some of it. It also provides customization options.
    \usepackage{titling}
    \usepackage{longtable} % longtable support required by pandoc >1.10
    \usepackage{booktabs}  % table support for pandoc > 1.12.2
    \usepackage[inline]{enumitem} % IRkernel/repr support (it uses the enumerate* environment)
    \usepackage[normalem]{ulem} % ulem is needed to support strikethroughs (\sout)
                                % normalem makes italics be italics, not underlines
    \usepackage{mathrsfs}
    

    
    % Colors for the hyperref package
    \definecolor{urlcolor}{rgb}{0,.145,.698}
    \definecolor{linkcolor}{rgb}{.71,0.21,0.01}
    \definecolor{citecolor}{rgb}{.12,.54,.11}

    % ANSI colors
    \definecolor{ansi-black}{HTML}{3E424D}
    \definecolor{ansi-black-intense}{HTML}{282C36}
    \definecolor{ansi-red}{HTML}{E75C58}
    \definecolor{ansi-red-intense}{HTML}{B22B31}
    \definecolor{ansi-green}{HTML}{00A250}
    \definecolor{ansi-green-intense}{HTML}{007427}
    \definecolor{ansi-yellow}{HTML}{DDB62B}
    \definecolor{ansi-yellow-intense}{HTML}{B27D12}
    \definecolor{ansi-blue}{HTML}{208FFB}
    \definecolor{ansi-blue-intense}{HTML}{0065CA}
    \definecolor{ansi-magenta}{HTML}{D160C4}
    \definecolor{ansi-magenta-intense}{HTML}{A03196}
    \definecolor{ansi-cyan}{HTML}{60C6C8}
    \definecolor{ansi-cyan-intense}{HTML}{258F8F}
    \definecolor{ansi-white}{HTML}{C5C1B4}
    \definecolor{ansi-white-intense}{HTML}{A1A6B2}
    \definecolor{ansi-default-inverse-fg}{HTML}{FFFFFF}
    \definecolor{ansi-default-inverse-bg}{HTML}{000000}

    % common color for the border for error outputs.
    \definecolor{outerrorbackground}{HTML}{FFDFDF}

    % commands and environments needed by pandoc snippets
    % extracted from the output of `pandoc -s`
    \providecommand{\tightlist}{%
      \setlength{\itemsep}{0pt}\setlength{\parskip}{0pt}}
    \DefineVerbatimEnvironment{Highlighting}{Verbatim}{commandchars=\\\{\}}
    % Add ',fontsize=\small' for more characters per line
    \newenvironment{Shaded}{}{}
    \newcommand{\KeywordTok}[1]{\textcolor[rgb]{0.00,0.44,0.13}{\textbf{{#1}}}}
    \newcommand{\DataTypeTok}[1]{\textcolor[rgb]{0.56,0.13,0.00}{{#1}}}
    \newcommand{\DecValTok}[1]{\textcolor[rgb]{0.25,0.63,0.44}{{#1}}}
    \newcommand{\BaseNTok}[1]{\textcolor[rgb]{0.25,0.63,0.44}{{#1}}}
    \newcommand{\FloatTok}[1]{\textcolor[rgb]{0.25,0.63,0.44}{{#1}}}
    \newcommand{\CharTok}[1]{\textcolor[rgb]{0.25,0.44,0.63}{{#1}}}
    \newcommand{\StringTok}[1]{\textcolor[rgb]{0.25,0.44,0.63}{{#1}}}
    \newcommand{\CommentTok}[1]{\textcolor[rgb]{0.38,0.63,0.69}{\textit{{#1}}}}
    \newcommand{\OtherTok}[1]{\textcolor[rgb]{0.00,0.44,0.13}{{#1}}}
    \newcommand{\AlertTok}[1]{\textcolor[rgb]{1.00,0.00,0.00}{\textbf{{#1}}}}
    \newcommand{\FunctionTok}[1]{\textcolor[rgb]{0.02,0.16,0.49}{{#1}}}
    \newcommand{\RegionMarkerTok}[1]{{#1}}
    \newcommand{\ErrorTok}[1]{\textcolor[rgb]{1.00,0.00,0.00}{\textbf{{#1}}}}
    \newcommand{\NormalTok}[1]{{#1}}
    
    % Additional commands for more recent versions of Pandoc
    \newcommand{\ConstantTok}[1]{\textcolor[rgb]{0.53,0.00,0.00}{{#1}}}
    \newcommand{\SpecialCharTok}[1]{\textcolor[rgb]{0.25,0.44,0.63}{{#1}}}
    \newcommand{\VerbatimStringTok}[1]{\textcolor[rgb]{0.25,0.44,0.63}{{#1}}}
    \newcommand{\SpecialStringTok}[1]{\textcolor[rgb]{0.73,0.40,0.53}{{#1}}}
    \newcommand{\ImportTok}[1]{{#1}}
    \newcommand{\DocumentationTok}[1]{\textcolor[rgb]{0.73,0.13,0.13}{\textit{{#1}}}}
    \newcommand{\AnnotationTok}[1]{\textcolor[rgb]{0.38,0.63,0.69}{\textbf{\textit{{#1}}}}}
    \newcommand{\CommentVarTok}[1]{\textcolor[rgb]{0.38,0.63,0.69}{\textbf{\textit{{#1}}}}}
    \newcommand{\VariableTok}[1]{\textcolor[rgb]{0.10,0.09,0.49}{{#1}}}
    \newcommand{\ControlFlowTok}[1]{\textcolor[rgb]{0.00,0.44,0.13}{\textbf{{#1}}}}
    \newcommand{\OperatorTok}[1]{\textcolor[rgb]{0.40,0.40,0.40}{{#1}}}
    \newcommand{\BuiltInTok}[1]{{#1}}
    \newcommand{\ExtensionTok}[1]{{#1}}
    \newcommand{\PreprocessorTok}[1]{\textcolor[rgb]{0.74,0.48,0.00}{{#1}}}
    \newcommand{\AttributeTok}[1]{\textcolor[rgb]{0.49,0.56,0.16}{{#1}}}
    \newcommand{\InformationTok}[1]{\textcolor[rgb]{0.38,0.63,0.69}{\textbf{\textit{{#1}}}}}
    \newcommand{\WarningTok}[1]{\textcolor[rgb]{0.38,0.63,0.69}{\textbf{\textit{{#1}}}}}
    
    
    % Define a nice break command that doesn't care if a line doesn't already
    % exist.
    \def\br{\hspace*{\fill} \\* }
    % Math Jax compatibility definitions
    \def\gt{>}
    \def\lt{<}
    \let\Oldtex\TeX
    \let\Oldlatex\LaTeX
    \renewcommand{\TeX}{\textrm{\Oldtex}}
    \renewcommand{\LaTeX}{\textrm{\Oldlatex}}
    % Document parameters
    % Document title
    \title{第五周实验作业}
    
    
    
    
    
% Pygments definitions
\makeatletter
\def\PY@reset{\let\PY@it=\relax \let\PY@bf=\relax%
    \let\PY@ul=\relax \let\PY@tc=\relax%
    \let\PY@bc=\relax \let\PY@ff=\relax}
\def\PY@tok#1{\csname PY@tok@#1\endcsname}
\def\PY@toks#1+{\ifx\relax#1\empty\else%
    \PY@tok{#1}\expandafter\PY@toks\fi}
\def\PY@do#1{\PY@bc{\PY@tc{\PY@ul{%
    \PY@it{\PY@bf{\PY@ff{#1}}}}}}}
\def\PY#1#2{\PY@reset\PY@toks#1+\relax+\PY@do{#2}}

\expandafter\def\csname PY@tok@w\endcsname{\def\PY@tc##1{\textcolor[rgb]{0.73,0.73,0.73}{##1}}}
\expandafter\def\csname PY@tok@c\endcsname{\let\PY@it=\textit\def\PY@tc##1{\textcolor[rgb]{0.25,0.50,0.50}{##1}}}
\expandafter\def\csname PY@tok@cp\endcsname{\def\PY@tc##1{\textcolor[rgb]{0.74,0.48,0.00}{##1}}}
\expandafter\def\csname PY@tok@k\endcsname{\let\PY@bf=\textbf\def\PY@tc##1{\textcolor[rgb]{0.00,0.50,0.00}{##1}}}
\expandafter\def\csname PY@tok@kp\endcsname{\def\PY@tc##1{\textcolor[rgb]{0.00,0.50,0.00}{##1}}}
\expandafter\def\csname PY@tok@kt\endcsname{\def\PY@tc##1{\textcolor[rgb]{0.69,0.00,0.25}{##1}}}
\expandafter\def\csname PY@tok@o\endcsname{\def\PY@tc##1{\textcolor[rgb]{0.40,0.40,0.40}{##1}}}
\expandafter\def\csname PY@tok@ow\endcsname{\let\PY@bf=\textbf\def\PY@tc##1{\textcolor[rgb]{0.67,0.13,1.00}{##1}}}
\expandafter\def\csname PY@tok@nb\endcsname{\def\PY@tc##1{\textcolor[rgb]{0.00,0.50,0.00}{##1}}}
\expandafter\def\csname PY@tok@nf\endcsname{\def\PY@tc##1{\textcolor[rgb]{0.00,0.00,1.00}{##1}}}
\expandafter\def\csname PY@tok@nc\endcsname{\let\PY@bf=\textbf\def\PY@tc##1{\textcolor[rgb]{0.00,0.00,1.00}{##1}}}
\expandafter\def\csname PY@tok@nn\endcsname{\let\PY@bf=\textbf\def\PY@tc##1{\textcolor[rgb]{0.00,0.00,1.00}{##1}}}
\expandafter\def\csname PY@tok@ne\endcsname{\let\PY@bf=\textbf\def\PY@tc##1{\textcolor[rgb]{0.82,0.25,0.23}{##1}}}
\expandafter\def\csname PY@tok@nv\endcsname{\def\PY@tc##1{\textcolor[rgb]{0.10,0.09,0.49}{##1}}}
\expandafter\def\csname PY@tok@no\endcsname{\def\PY@tc##1{\textcolor[rgb]{0.53,0.00,0.00}{##1}}}
\expandafter\def\csname PY@tok@nl\endcsname{\def\PY@tc##1{\textcolor[rgb]{0.63,0.63,0.00}{##1}}}
\expandafter\def\csname PY@tok@ni\endcsname{\let\PY@bf=\textbf\def\PY@tc##1{\textcolor[rgb]{0.60,0.60,0.60}{##1}}}
\expandafter\def\csname PY@tok@na\endcsname{\def\PY@tc##1{\textcolor[rgb]{0.49,0.56,0.16}{##1}}}
\expandafter\def\csname PY@tok@nt\endcsname{\let\PY@bf=\textbf\def\PY@tc##1{\textcolor[rgb]{0.00,0.50,0.00}{##1}}}
\expandafter\def\csname PY@tok@nd\endcsname{\def\PY@tc##1{\textcolor[rgb]{0.67,0.13,1.00}{##1}}}
\expandafter\def\csname PY@tok@s\endcsname{\def\PY@tc##1{\textcolor[rgb]{0.73,0.13,0.13}{##1}}}
\expandafter\def\csname PY@tok@sd\endcsname{\let\PY@it=\textit\def\PY@tc##1{\textcolor[rgb]{0.73,0.13,0.13}{##1}}}
\expandafter\def\csname PY@tok@si\endcsname{\let\PY@bf=\textbf\def\PY@tc##1{\textcolor[rgb]{0.73,0.40,0.53}{##1}}}
\expandafter\def\csname PY@tok@se\endcsname{\let\PY@bf=\textbf\def\PY@tc##1{\textcolor[rgb]{0.73,0.40,0.13}{##1}}}
\expandafter\def\csname PY@tok@sr\endcsname{\def\PY@tc##1{\textcolor[rgb]{0.73,0.40,0.53}{##1}}}
\expandafter\def\csname PY@tok@ss\endcsname{\def\PY@tc##1{\textcolor[rgb]{0.10,0.09,0.49}{##1}}}
\expandafter\def\csname PY@tok@sx\endcsname{\def\PY@tc##1{\textcolor[rgb]{0.00,0.50,0.00}{##1}}}
\expandafter\def\csname PY@tok@m\endcsname{\def\PY@tc##1{\textcolor[rgb]{0.40,0.40,0.40}{##1}}}
\expandafter\def\csname PY@tok@gh\endcsname{\let\PY@bf=\textbf\def\PY@tc##1{\textcolor[rgb]{0.00,0.00,0.50}{##1}}}
\expandafter\def\csname PY@tok@gu\endcsname{\let\PY@bf=\textbf\def\PY@tc##1{\textcolor[rgb]{0.50,0.00,0.50}{##1}}}
\expandafter\def\csname PY@tok@gd\endcsname{\def\PY@tc##1{\textcolor[rgb]{0.63,0.00,0.00}{##1}}}
\expandafter\def\csname PY@tok@gi\endcsname{\def\PY@tc##1{\textcolor[rgb]{0.00,0.63,0.00}{##1}}}
\expandafter\def\csname PY@tok@gr\endcsname{\def\PY@tc##1{\textcolor[rgb]{1.00,0.00,0.00}{##1}}}
\expandafter\def\csname PY@tok@ge\endcsname{\let\PY@it=\textit}
\expandafter\def\csname PY@tok@gs\endcsname{\let\PY@bf=\textbf}
\expandafter\def\csname PY@tok@gp\endcsname{\let\PY@bf=\textbf\def\PY@tc##1{\textcolor[rgb]{0.00,0.00,0.50}{##1}}}
\expandafter\def\csname PY@tok@go\endcsname{\def\PY@tc##1{\textcolor[rgb]{0.53,0.53,0.53}{##1}}}
\expandafter\def\csname PY@tok@gt\endcsname{\def\PY@tc##1{\textcolor[rgb]{0.00,0.27,0.87}{##1}}}
\expandafter\def\csname PY@tok@err\endcsname{\def\PY@bc##1{\setlength{\fboxsep}{0pt}\fcolorbox[rgb]{1.00,0.00,0.00}{1,1,1}{\strut ##1}}}
\expandafter\def\csname PY@tok@kc\endcsname{\let\PY@bf=\textbf\def\PY@tc##1{\textcolor[rgb]{0.00,0.50,0.00}{##1}}}
\expandafter\def\csname PY@tok@kd\endcsname{\let\PY@bf=\textbf\def\PY@tc##1{\textcolor[rgb]{0.00,0.50,0.00}{##1}}}
\expandafter\def\csname PY@tok@kn\endcsname{\let\PY@bf=\textbf\def\PY@tc##1{\textcolor[rgb]{0.00,0.50,0.00}{##1}}}
\expandafter\def\csname PY@tok@kr\endcsname{\let\PY@bf=\textbf\def\PY@tc##1{\textcolor[rgb]{0.00,0.50,0.00}{##1}}}
\expandafter\def\csname PY@tok@bp\endcsname{\def\PY@tc##1{\textcolor[rgb]{0.00,0.50,0.00}{##1}}}
\expandafter\def\csname PY@tok@fm\endcsname{\def\PY@tc##1{\textcolor[rgb]{0.00,0.00,1.00}{##1}}}
\expandafter\def\csname PY@tok@vc\endcsname{\def\PY@tc##1{\textcolor[rgb]{0.10,0.09,0.49}{##1}}}
\expandafter\def\csname PY@tok@vg\endcsname{\def\PY@tc##1{\textcolor[rgb]{0.10,0.09,0.49}{##1}}}
\expandafter\def\csname PY@tok@vi\endcsname{\def\PY@tc##1{\textcolor[rgb]{0.10,0.09,0.49}{##1}}}
\expandafter\def\csname PY@tok@vm\endcsname{\def\PY@tc##1{\textcolor[rgb]{0.10,0.09,0.49}{##1}}}
\expandafter\def\csname PY@tok@sa\endcsname{\def\PY@tc##1{\textcolor[rgb]{0.73,0.13,0.13}{##1}}}
\expandafter\def\csname PY@tok@sb\endcsname{\def\PY@tc##1{\textcolor[rgb]{0.73,0.13,0.13}{##1}}}
\expandafter\def\csname PY@tok@sc\endcsname{\def\PY@tc##1{\textcolor[rgb]{0.73,0.13,0.13}{##1}}}
\expandafter\def\csname PY@tok@dl\endcsname{\def\PY@tc##1{\textcolor[rgb]{0.73,0.13,0.13}{##1}}}
\expandafter\def\csname PY@tok@s2\endcsname{\def\PY@tc##1{\textcolor[rgb]{0.73,0.13,0.13}{##1}}}
\expandafter\def\csname PY@tok@sh\endcsname{\def\PY@tc##1{\textcolor[rgb]{0.73,0.13,0.13}{##1}}}
\expandafter\def\csname PY@tok@s1\endcsname{\def\PY@tc##1{\textcolor[rgb]{0.73,0.13,0.13}{##1}}}
\expandafter\def\csname PY@tok@mb\endcsname{\def\PY@tc##1{\textcolor[rgb]{0.40,0.40,0.40}{##1}}}
\expandafter\def\csname PY@tok@mf\endcsname{\def\PY@tc##1{\textcolor[rgb]{0.40,0.40,0.40}{##1}}}
\expandafter\def\csname PY@tok@mh\endcsname{\def\PY@tc##1{\textcolor[rgb]{0.40,0.40,0.40}{##1}}}
\expandafter\def\csname PY@tok@mi\endcsname{\def\PY@tc##1{\textcolor[rgb]{0.40,0.40,0.40}{##1}}}
\expandafter\def\csname PY@tok@il\endcsname{\def\PY@tc##1{\textcolor[rgb]{0.40,0.40,0.40}{##1}}}
\expandafter\def\csname PY@tok@mo\endcsname{\def\PY@tc##1{\textcolor[rgb]{0.40,0.40,0.40}{##1}}}
\expandafter\def\csname PY@tok@ch\endcsname{\let\PY@it=\textit\def\PY@tc##1{\textcolor[rgb]{0.25,0.50,0.50}{##1}}}
\expandafter\def\csname PY@tok@cm\endcsname{\let\PY@it=\textit\def\PY@tc##1{\textcolor[rgb]{0.25,0.50,0.50}{##1}}}
\expandafter\def\csname PY@tok@cpf\endcsname{\let\PY@it=\textit\def\PY@tc##1{\textcolor[rgb]{0.25,0.50,0.50}{##1}}}
\expandafter\def\csname PY@tok@c1\endcsname{\let\PY@it=\textit\def\PY@tc##1{\textcolor[rgb]{0.25,0.50,0.50}{##1}}}
\expandafter\def\csname PY@tok@cs\endcsname{\let\PY@it=\textit\def\PY@tc##1{\textcolor[rgb]{0.25,0.50,0.50}{##1}}}

\def\PYZbs{\char`\\}
\def\PYZus{\char`\_}
\def\PYZob{\char`\{}
\def\PYZcb{\char`\}}
\def\PYZca{\char`\^}
\def\PYZam{\char`\&}
\def\PYZlt{\char`\<}
\def\PYZgt{\char`\>}
\def\PYZsh{\char`\#}
\def\PYZpc{\char`\%}
\def\PYZdl{\char`\$}
\def\PYZhy{\char`\-}
\def\PYZsq{\char`\'}
\def\PYZdq{\char`\"}
\def\PYZti{\char`\~}
% for compatibility with earlier versions
\def\PYZat{@}
\def\PYZlb{[}
\def\PYZrb{]}
\makeatother


    % For linebreaks inside Verbatim environment from package fancyvrb. 
    \makeatletter
        \newbox\Wrappedcontinuationbox 
        \newbox\Wrappedvisiblespacebox 
        \newcommand*\Wrappedvisiblespace {\textcolor{red}{\textvisiblespace}} 
        \newcommand*\Wrappedcontinuationsymbol {\textcolor{red}{\llap{\tiny$\m@th\hookrightarrow$}}} 
        \newcommand*\Wrappedcontinuationindent {3ex } 
        \newcommand*\Wrappedafterbreak {\kern\Wrappedcontinuationindent\copy\Wrappedcontinuationbox} 
        % Take advantage of the already applied Pygments mark-up to insert 
        % potential linebreaks for TeX processing. 
        %        {, <, #, %, $, ' and ": go to next line. 
        %        _, }, ^, &, >, - and ~: stay at end of broken line. 
        % Use of \textquotesingle for straight quote. 
        \newcommand*\Wrappedbreaksatspecials {% 
            \def\PYGZus{\discretionary{\char`\_}{\Wrappedafterbreak}{\char`\_}}% 
            \def\PYGZob{\discretionary{}{\Wrappedafterbreak\char`\{}{\char`\{}}% 
            \def\PYGZcb{\discretionary{\char`\}}{\Wrappedafterbreak}{\char`\}}}% 
            \def\PYGZca{\discretionary{\char`\^}{\Wrappedafterbreak}{\char`\^}}% 
            \def\PYGZam{\discretionary{\char`\&}{\Wrappedafterbreak}{\char`\&}}% 
            \def\PYGZlt{\discretionary{}{\Wrappedafterbreak\char`\<}{\char`\<}}% 
            \def\PYGZgt{\discretionary{\char`\>}{\Wrappedafterbreak}{\char`\>}}% 
            \def\PYGZsh{\discretionary{}{\Wrappedafterbreak\char`\#}{\char`\#}}% 
            \def\PYGZpc{\discretionary{}{\Wrappedafterbreak\char`\%}{\char`\%}}% 
            \def\PYGZdl{\discretionary{}{\Wrappedafterbreak\char`\$}{\char`\$}}% 
            \def\PYGZhy{\discretionary{\char`\-}{\Wrappedafterbreak}{\char`\-}}% 
            \def\PYGZsq{\discretionary{}{\Wrappedafterbreak\textquotesingle}{\textquotesingle}}% 
            \def\PYGZdq{\discretionary{}{\Wrappedafterbreak\char`\"}{\char`\"}}% 
            \def\PYGZti{\discretionary{\char`\~}{\Wrappedafterbreak}{\char`\~}}% 
        } 
        % Some characters . , ; ? ! / are not pygmentized. 
        % This macro makes them "active" and they will insert potential linebreaks 
        \newcommand*\Wrappedbreaksatpunct {% 
            \lccode`\~`\.\lowercase{\def~}{\discretionary{\hbox{\char`\.}}{\Wrappedafterbreak}{\hbox{\char`\.}}}% 
            \lccode`\~`\,\lowercase{\def~}{\discretionary{\hbox{\char`\,}}{\Wrappedafterbreak}{\hbox{\char`\,}}}% 
            \lccode`\~`\;\lowercase{\def~}{\discretionary{\hbox{\char`\;}}{\Wrappedafterbreak}{\hbox{\char`\;}}}% 
            \lccode`\~`\:\lowercase{\def~}{\discretionary{\hbox{\char`\:}}{\Wrappedafterbreak}{\hbox{\char`\:}}}% 
            \lccode`\~`\?\lowercase{\def~}{\discretionary{\hbox{\char`\?}}{\Wrappedafterbreak}{\hbox{\char`\?}}}% 
            \lccode`\~`\!\lowercase{\def~}{\discretionary{\hbox{\char`\!}}{\Wrappedafterbreak}{\hbox{\char`\!}}}% 
            \lccode`\~`\/\lowercase{\def~}{\discretionary{\hbox{\char`\/}}{\Wrappedafterbreak}{\hbox{\char`\/}}}% 
            \catcode`\.\active
            \catcode`\,\active 
            \catcode`\;\active
            \catcode`\:\active
            \catcode`\?\active
            \catcode`\!\active
            \catcode`\/\active 
            \lccode`\~`\~ 	
        }
    \makeatother

    \let\OriginalVerbatim=\Verbatim
    \makeatletter
    \renewcommand{\Verbatim}[1][1]{%
        %\parskip\z@skip
        \sbox\Wrappedcontinuationbox {\Wrappedcontinuationsymbol}%
        \sbox\Wrappedvisiblespacebox {\FV@SetupFont\Wrappedvisiblespace}%
        \def\FancyVerbFormatLine ##1{\hsize\linewidth
            \vtop{\raggedright\hyphenpenalty\z@\exhyphenpenalty\z@
                \doublehyphendemerits\z@\finalhyphendemerits\z@
                \strut ##1\strut}%
        }%
        % If the linebreak is at a space, the latter will be displayed as visible
        % space at end of first line, and a continuation symbol starts next line.
        % Stretch/shrink are however usually zero for typewriter font.
        \def\FV@Space {%
            \nobreak\hskip\z@ plus\fontdimen3\font minus\fontdimen4\font
            \discretionary{\copy\Wrappedvisiblespacebox}{\Wrappedafterbreak}
            {\kern\fontdimen2\font}%
        }%
        
        % Allow breaks at special characters using \PYG... macros.
        \Wrappedbreaksatspecials
        % Breaks at punctuation characters . , ; ? ! and / need catcode=\active 	
        \OriginalVerbatim[#1,codes*=\Wrappedbreaksatpunct]%
    }
    \makeatother

    % Exact colors from NB
    \definecolor{incolor}{HTML}{303F9F}
    \definecolor{outcolor}{HTML}{D84315}
    \definecolor{cellborder}{HTML}{CFCFCF}
    \definecolor{cellbackground}{HTML}{F7F7F7}
    
    % prompt
    \makeatletter
    \newcommand{\boxspacing}{\kern\kvtcb@left@rule\kern\kvtcb@boxsep}
    \makeatother
    \newcommand{\prompt}[4]{
        {\ttfamily\llap{{\color{#2}[#3]:\hspace{3pt}#4}}\vspace{-\baselineskip}}
    }
    

    
    % Prevent overflowing lines due to hard-to-break entities
    \sloppy 
    % Setup hyperref package
    \hypersetup{
      breaklinks=true,  % so long urls are correctly broken across lines
      colorlinks=true,
      urlcolor=urlcolor,
      linkcolor=linkcolor,
      citecolor=citecolor,
      }
    % Slightly bigger margins than the latex defaults
    
    \geometry{verbose,tmargin=1in,bmargin=1in,lmargin=1in,rmargin=1in}
    
    

\begin{document}
    
    \maketitle
    
    

    
    \hypertarget{week5-multiple-linear-regression}{%
\section{Week5 Multiple Linear
Regression}\label{week5-multiple-linear-regression}}

\hypertarget{ux80ccux666fux63cfux8ff0}{%
\subsection{背景描述}\label{ux80ccux666fux63cfux8ff0}}

汽车发动机在测功机上产生的\textbf{制动马力}被认为是\textbf{发动机转速}(每分钟转数,rpm)、\textbf{燃料的道路辛烷值}和\textbf{发动机压缩值}的函数,我们在实验室里进行实验,研究它们的函数关系。

\hypertarget{ux6570ux636eux63cfux8ff0}{%
\subsection{数据描述}\label{ux6570ux636eux63cfux8ff0}}

\begin{longtable}{llll}
\toprule
变量名 & 变量含义 & 变量类型 & 变量取值范围 \\
\midrule
\endhead
(自变量)rpm & 发动机转速 & continuous variable & \(\mathbb{R}^+\) \\
(自变量)Road\_Octane\_Number & 道路辛烷值 & continuous variable &
\(\mathbb{R}^+\) \\
(自变量)Compression & 压缩值 & continuous variable &
\(\mathbb{R}^+\) \\
(因变量)Brake\_Horsepower & 制动马力 & continuous variable &
\(\mathbb{R}^+\) \\
\bottomrule
\end{longtable}

    \begin{tcolorbox}[breakable, size=fbox, boxrule=1pt, pad at break*=1mm,colback=cellbackground, colframe=cellborder]
\prompt{In}{incolor}{1}{\boxspacing}
\begin{Verbatim}[commandchars=\\\{\}]
\PY{k+kn}{import} \PY{n+nn}{pandas} \PY{k}{as} \PY{n+nn}{pd}
\PY{n+nb}{print}\PY{p}{(}\PY{l+s+s1}{\PYZsq{}}\PY{l+s+s1}{Data: }\PY{l+s+se}{\PYZbs{}n}\PY{l+s+s1}{\PYZsq{}}\PY{p}{,} \PY{n}{pd}\PY{o}{.}\PY{n}{read\PYZus{}csv}\PY{p}{(}\PY{l+s+s1}{\PYZsq{}}\PY{l+s+s1}{Project5.csv}\PY{l+s+s1}{\PYZsq{}}\PY{p}{)}\PY{o}{.}\PY{n}{values}\PY{p}{)}
\end{Verbatim}
\end{tcolorbox}

    \begin{Verbatim}[commandchars=\\\{\}]
Data:
 [[2000   90  100  225]
 [1800   94   95  212]
 [2400   88  110  229]
 [1900   91   96  222]
 [1600   86  100  219]
 [2500   96  110  278]
 [3000   94   98  246]
 [3200   90  100  237]
 [2800   88  105  233]
 [3400   86   97  224]
 [1800   90  100  223]
 [2500   89  104  230]]
    \end{Verbatim}

    \hypertarget{ux95eeux9898}{%
\subsection{问题}\label{ux95eeux9898}}

注:这里使用 \(\alpha\)=0.05 的显著性水平:

\begin{enumerate}
\def\labelenumi{\arabic{enumi}.}
\tightlist
\item
  请用统计模型描述制动马力和上述 3 个自变量之间的函数关系.
\item
  请问Q1中所建立的模型是否合理.
\item
  当发动机转速为3000转/min,道路辛烷值为90,发动机压缩值为100时,估计制动马力值的合理区间范围.
\end{enumerate}

\hypertarget{ux89e3ux51b3ux65b9ux6848}{%
\subsection{解决方案}\label{ux89e3ux51b3ux65b9ux6848}}

    \textbf{Q1:}\\
使用多元线性回归的方法,令发动机转速为\(X_1\),道路辛烷值为\(X_2\),压缩值为\(X_3\),制动马力为\(Y\)。则线性模型为:\(Y = \beta_0 + \beta_1X_1 + \beta_2X_2 + \beta_3X_3 +\epsilon\),并假定随机误差项符合正态分布。根据以上数据,可以求得\(\hat{\beta_1},\hat{\beta_2},\hat{\beta_3}\)及线性回归方程如下。

    \begin{tcolorbox}[breakable, size=fbox, boxrule=1pt, pad at break*=1mm,colback=cellbackground, colframe=cellborder]
\prompt{In}{incolor}{2}{\boxspacing}
\begin{Verbatim}[commandchars=\\\{\}]
\PY{c+c1}{\PYZsh{} Import standard packages}
\PY{k+kn}{import} \PY{n+nn}{numpy} \PY{k}{as} \PY{n+nn}{np}
\PY{k+kn}{import} \PY{n+nn}{pandas} \PY{k}{as} \PY{n+nn}{pd}
\PY{k+kn}{import} \PY{n+nn}{scipy}\PY{n+nn}{.}\PY{n+nn}{stats} \PY{k}{as} \PY{n+nn}{stats}
\PY{k+kn}{import} \PY{n+nn}{matplotlib}\PY{n+nn}{.}\PY{n+nn}{pyplot} \PY{k}{as} \PY{n+nn}{plt}
\PY{k+kn}{import} \PY{n+nn}{math}

\PY{c+c1}{\PYZsh{} Import additional packages}
\PY{k+kn}{from} \PY{n+nn}{statsmodels}\PY{n+nn}{.}\PY{n+nn}{formula}\PY{n+nn}{.}\PY{n+nn}{api} \PY{k+kn}{import} \PY{n}{ols}
\PY{k+kn}{from} \PY{n+nn}{statsmodels}\PY{n+nn}{.}\PY{n+nn}{stats}\PY{n+nn}{.}\PY{n+nn}{anova} \PY{k+kn}{import} \PY{n}{anova\PYZus{}lm}
\PY{k+kn}{from} \PY{n+nn}{scipy}\PY{n+nn}{.}\PY{n+nn}{stats} \PY{k+kn}{import} \PY{n}{f}
\PY{k+kn}{from} \PY{n+nn}{scipy}\PY{n+nn}{.}\PY{n+nn}{stats} \PY{k+kn}{import} \PY{n}{t}

\PY{n}{alpha} \PY{o}{=} \PY{l+m+mf}{0.05}
\PY{n}{p} \PY{o}{=} \PY{l+m+mi}{3}
\PY{n}{n} \PY{o}{=} \PY{l+m+mi}{12}

\PY{n}{x} \PY{o}{=} \PY{n}{pd}\PY{o}{.}\PY{n}{read\PYZus{}csv}\PY{p}{(}\PY{l+s+s1}{\PYZsq{}}\PY{l+s+s1}{Project5.csv}\PY{l+s+s1}{\PYZsq{}}\PY{p}{)}
\PY{n}{x}\PY{o}{.}\PY{n}{insert}\PY{p}{(}\PY{l+m+mi}{0}\PY{p}{,} \PY{l+s+s1}{\PYZsq{}}\PY{l+s+s1}{intercept}\PY{l+s+s1}{\PYZsq{}}\PY{p}{,} \PY{n}{np}\PY{o}{.}\PY{n}{ones}\PY{p}{(}\PY{n+nb}{len}\PY{p}{(}\PY{n}{x}\PY{p}{)}\PY{p}{)}\PY{p}{)} 
\PY{n}{data} \PY{o}{=} \PY{n}{x}\PY{o}{.}\PY{n}{values} \PY{o}{*} \PY{l+m+mi}{1}
\PY{n}{df} \PY{o}{=} \PY{n}{pd}\PY{o}{.}\PY{n}{DataFrame}\PY{p}{(}\PY{n}{data}\PY{p}{,} \PY{n}{columns} \PY{o}{=} \PY{p}{[}\PY{l+s+s1}{\PYZsq{}}\PY{l+s+s1}{intercept}\PY{l+s+s1}{\PYZsq{}}\PY{p}{,} \PY{l+s+s1}{\PYZsq{}}\PY{l+s+s1}{P1}\PY{l+s+s1}{\PYZsq{}}\PY{p}{,} \PY{l+s+s1}{\PYZsq{}}\PY{l+s+s1}{P2}\PY{l+s+s1}{\PYZsq{}}\PY{p}{,} \PY{l+s+s1}{\PYZsq{}}\PY{l+s+s1}{P3}\PY{l+s+s1}{\PYZsq{}}\PY{p}{,} \PY{l+s+s1}{\PYZsq{}}\PY{l+s+s1}{F}\PY{l+s+s1}{\PYZsq{}}\PY{p}{]}\PY{p}{)}
\PY{n+nb}{print}\PY{p}{(}\PY{n}{df}\PY{p}{)}

\PY{c+c1}{\PYZsh{} Do the multiple linear regression}
\PY{n}{model} \PY{o}{=} \PY{n}{ols}\PY{p}{(}\PY{l+s+s1}{\PYZsq{}}\PY{l+s+s1}{F \PYZti{} P1 + P2 + P3}\PY{l+s+s1}{\PYZsq{}}\PY{p}{,} \PY{n}{df}\PY{p}{)}\PY{o}{.}\PY{n}{fit}\PY{p}{(}\PY{p}{)}
\PY{n}{beta} \PY{o}{=} \PY{n}{model}\PY{o}{.}\PY{n}{params}
\PY{n+nb}{print}\PY{p}{(}\PY{l+s+s1}{\PYZsq{}}\PY{l+s+s1}{参数估计值: }\PY{l+s+se}{\PYZbs{}n}\PY{l+s+s1}{\PYZsq{}}\PY{p}{,} \PY{n+nb}{round}\PY{p}{(}\PY{n}{beta}\PY{p}{,} \PY{l+m+mi}{4}\PY{p}{)}\PY{p}{)}
\PY{n}{X} \PY{o}{=} \PY{n}{data}\PY{p}{[}\PY{p}{:}\PY{p}{,} \PY{l+m+mi}{0} \PY{p}{:} \PY{n}{p} \PY{o}{+} \PY{l+m+mi}{1}\PY{p}{]}
\PY{n}{Y} \PY{o}{=} \PY{n}{data}\PY{p}{[}\PY{p}{:}\PY{p}{,} \PY{o}{\PYZhy{}}\PY{l+m+mi}{1}\PY{p}{]}
\PY{n}{Y\PYZus{}hat} \PY{o}{=} \PY{n}{model}\PY{o}{.}\PY{n}{fittedvalues}
\PY{n}{model}\PY{o}{.}\PY{n}{summary}\PY{p}{(}\PY{p}{)}
\end{Verbatim}
\end{tcolorbox}

    \begin{Verbatim}[commandchars=\\\{\}]
    intercept      P1    P2     P3      F
0         1.0  2000.0  90.0  100.0  225.0
1         1.0  1800.0  94.0   95.0  212.0
2         1.0  2400.0  88.0  110.0  229.0
3         1.0  1900.0  91.0   96.0  222.0
4         1.0  1600.0  86.0  100.0  219.0
5         1.0  2500.0  96.0  110.0  278.0
6         1.0  3000.0  94.0   98.0  246.0
7         1.0  3200.0  90.0  100.0  237.0
8         1.0  2800.0  88.0  105.0  233.0
9         1.0  3400.0  86.0   97.0  224.0
10        1.0  1800.0  90.0  100.0  223.0
11        1.0  2500.0  89.0  104.0  230.0
参数估计值:
 Intercept   -266.0312
P1             0.0107
P2             3.1348
P3             1.8674
dtype: float64
    \end{Verbatim}

    \begin{Verbatim}[commandchars=\\\{\}]
/Library/Frameworks/Python.framework/Versions/3.6/lib/python3.6/site-
packages/scipy/stats/stats.py:1604: UserWarning: kurtosistest only valid for
n>=20 {\ldots} continuing anyway, n=12
  "anyway, n=\%i" \% int(n))
    \end{Verbatim}

            \begin{tcolorbox}[breakable, size=fbox, boxrule=.5pt, pad at break*=1mm, opacityfill=0]
\prompt{Out}{outcolor}{2}{\boxspacing}
\begin{Verbatim}[commandchars=\\\{\}]
<class 'statsmodels.iolib.summary.Summary'>
"""
                            OLS Regression Results
==============================================================================
Dep. Variable:                      F   R-squared:                       0.807
Model:                            OLS   Adj. R-squared:                  0.734
Method:                 Least Squares   F-statistic:                     11.12
Date:                Wed, 31 Mar 2021   Prob (F-statistic):            0.00317
Time:                        16:12:47   Log-Likelihood:                -40.708
No. Observations:                  12   AIC:                             89.42
Df Residuals:                       8   BIC:                             91.36
Df Model:                           3
Covariance Type:            nonrobust
==============================================================================
                 coef    std err          t      P>|t|      [0.025      0.975]
------------------------------------------------------------------------------
Intercept   -266.0312     92.674     -2.871      0.021    -479.737     -52.325
P1             0.0107      0.004      2.390      0.044       0.000       0.021
P2             3.1348      0.844      3.712      0.006       1.188       5.082
P3             1.8674      0.535      3.494      0.008       0.635       3.100
==============================================================================
Omnibus:                        0.392   Durbin-Watson:                   1.043
Prob(Omnibus):                  0.822   Jarque-Bera (JB):                0.230
Skew:                          -0.282   Prob(JB):                        0.891
Kurtosis:                       2.625   Cond. No.                     9.03e+04
==============================================================================

Notes:
[1] Standard Errors assume that the covariance matrix of the errors is correctly
specified.
[2] The condition number is large, 9.03e+04. This might indicate that there are
strong multicollinearity or other numerical problems.
"""
\end{Verbatim}
\end{tcolorbox}
        
    \begin{tcolorbox}[breakable, size=fbox, boxrule=1pt, pad at break*=1mm,colback=cellbackground, colframe=cellborder]
\prompt{In}{incolor}{3}{\boxspacing}
\begin{Verbatim}[commandchars=\\\{\}]
\PY{c+c1}{\PYZsh{} 输出多元线性回归方程}
\PY{n+nb}{print}\PY{p}{(}\PY{l+s+s1}{\PYZsq{}}\PY{l+s+s1}{Y\PYZus{}hat = }\PY{l+s+si}{\PYZpc{}.2f}\PY{l+s+s1}{ + (}\PY{l+s+si}{\PYZpc{}.2f}\PY{l+s+s1}{ * X1) + (}\PY{l+s+si}{\PYZpc{}.2f}\PY{l+s+s1}{ * X2) + (}\PY{l+s+si}{\PYZpc{}.2f}\PY{l+s+s1}{ * X3)}\PY{l+s+s1}{\PYZsq{}} \PY{o}{\PYZpc{}} \PY{p}{(}\PY{n}{beta}\PY{p}{[}\PY{l+m+mi}{0}\PY{p}{]}\PY{p}{,} \PY{n}{beta}\PY{p}{[}\PY{l+m+mi}{1}\PY{p}{]}\PY{p}{,} \PY{n}{beta}\PY{p}{[}\PY{l+m+mi}{2}\PY{p}{]}\PY{p}{,} \PY{n}{beta}\PY{p}{[}\PY{l+m+mi}{3}\PY{p}{]}\PY{p}{)}\PY{p}{)}
\end{Verbatim}
\end{tcolorbox}

    \begin{Verbatim}[commandchars=\\\{\}]
Y\_hat = -266.03 + (0.01 * X1) + (3.13 * X2) + (1.87 * X3)
    \end{Verbatim}

    \textbf{将数据中心化之后,求经验回归方程:}

    \begin{tcolorbox}[breakable, size=fbox, boxrule=1pt, pad at break*=1mm,colback=cellbackground, colframe=cellborder]
\prompt{In}{incolor}{4}{\boxspacing}
\begin{Verbatim}[commandchars=\\\{\}]
\PY{c+c1}{\PYZsh{} 求均值}
\PY{n}{X\PYZus{}mean} \PY{o}{=} \PY{p}{[}\PY{p}{]}
\PY{k}{for} \PY{n}{k} \PY{o+ow}{in} \PY{n+nb}{range}\PY{p}{(}\PY{n}{p} \PY{o}{+} \PY{l+m+mi}{1}\PY{p}{)}\PY{p}{:}
    \PY{n}{X\PYZus{}mean}\PY{o}{.}\PY{n}{append}\PY{p}{(}\PY{n}{np}\PY{o}{.}\PY{n}{mean}\PY{p}{(}\PY{n}{data}\PY{p}{[}\PY{p}{:}\PY{p}{,} \PY{n}{k}\PY{p}{]}\PY{p}{)}\PY{p}{)}  \PY{c+c1}{\PYZsh{} 自变量 x 的均值}
\PY{n}{Y\PYZus{}mean} \PY{o}{=} \PY{n}{np}\PY{o}{.}\PY{n}{mean}\PY{p}{(}\PY{n}{data}\PY{p}{[}\PY{p}{:}\PY{p}{,} \PY{o}{\PYZhy{}}\PY{l+m+mi}{1}\PY{p}{]}\PY{p}{)}  \PY{c+c1}{\PYZsh{} 因变量 y 的均值}

\PY{c+c1}{\PYZsh{} 数据中心化}
\PY{n}{X\PYZus{}cent} \PY{o}{=} \PY{n}{X} \PY{o}{\PYZhy{}} \PY{n}{X\PYZus{}mean}
\PY{n}{Y\PYZus{}cent} \PY{o}{=} \PY{n}{Y} \PY{o}{\PYZhy{}} \PY{n}{Y\PYZus{}mean}

\PY{c+c1}{\PYZsh{} Do the multiple linear regression}
\PY{n}{df} \PY{o}{=} \PY{n}{pd}\PY{o}{.}\PY{n}{DataFrame}\PY{p}{(}\PY{n}{X\PYZus{}cent}\PY{p}{,} \PY{n}{columns} \PY{o}{=} \PY{p}{[}\PY{l+s+s1}{\PYZsq{}}\PY{l+s+s1}{intercept\PYZus{}cent}\PY{l+s+s1}{\PYZsq{}}\PY{p}{,} \PY{l+s+s1}{\PYZsq{}}\PY{l+s+s1}{P1\PYZus{}cent}\PY{l+s+s1}{\PYZsq{}}\PY{p}{,} \PY{l+s+s1}{\PYZsq{}}\PY{l+s+s1}{P2\PYZus{}cent}\PY{l+s+s1}{\PYZsq{}}\PY{p}{,} \PY{l+s+s1}{\PYZsq{}}\PY{l+s+s1}{P3\PYZus{}cent}\PY{l+s+s1}{\PYZsq{}}\PY{p}{]}\PY{p}{)}
\PY{n}{df}\PY{p}{[}\PY{l+s+s1}{\PYZsq{}}\PY{l+s+s1}{F\PYZus{}cent}\PY{l+s+s1}{\PYZsq{}}\PY{p}{]} \PY{o}{=} \PY{n}{Y\PYZus{}cent}
\PY{n}{model\PYZus{}cent} \PY{o}{=} \PY{n}{ols}\PY{p}{(}\PY{l+s+s1}{\PYZsq{}}\PY{l+s+s1}{F\PYZus{}cent \PYZti{} P1\PYZus{}cent + P2\PYZus{}cent + P3\PYZus{}cent}\PY{l+s+s1}{\PYZsq{}}\PY{p}{,} \PY{n}{df}\PY{p}{)}\PY{o}{.}\PY{n}{fit}\PY{p}{(}\PY{p}{)}
\PY{n}{beta\PYZus{}cent} \PY{o}{=} \PY{n}{model\PYZus{}cent}\PY{o}{.}\PY{n}{params}
\PY{n+nb}{print}\PY{p}{(}\PY{l+s+s1}{\PYZsq{}}\PY{l+s+s1}{参数估计值: }\PY{l+s+se}{\PYZbs{}n}\PY{l+s+s1}{\PYZsq{}}\PY{p}{,} \PY{n+nb}{round}\PY{p}{(}\PY{n}{beta\PYZus{}cent}\PY{p}{,} \PY{l+m+mi}{4}\PY{p}{)}\PY{p}{)}
\PY{n}{Y\PYZus{}hat\PYZus{}cent} \PY{o}{=} \PY{n}{model\PYZus{}cent}\PY{o}{.}\PY{n}{fittedvalues}
\PY{n}{model\PYZus{}cent}\PY{o}{.}\PY{n}{summary}\PY{p}{(}\PY{p}{)}
\end{Verbatim}
\end{tcolorbox}

    \begin{Verbatim}[commandchars=\\\{\}]
参数估计值:
 Intercept   -0.0000
P1\_cent      0.0107
P2\_cent      3.1348
P3\_cent      1.8674
dtype: float64
    \end{Verbatim}

    \begin{Verbatim}[commandchars=\\\{\}]
/Library/Frameworks/Python.framework/Versions/3.6/lib/python3.6/site-
packages/scipy/stats/stats.py:1604: UserWarning: kurtosistest only valid for
n>=20 {\ldots} continuing anyway, n=12
  "anyway, n=\%i" \% int(n))
    \end{Verbatim}

            \begin{tcolorbox}[breakable, size=fbox, boxrule=.5pt, pad at break*=1mm, opacityfill=0]
\prompt{Out}{outcolor}{4}{\boxspacing}
\begin{Verbatim}[commandchars=\\\{\}]
<class 'statsmodels.iolib.summary.Summary'>
"""
                            OLS Regression Results
==============================================================================
Dep. Variable:                 F\_cent   R-squared:                       0.807
Model:                            OLS   Adj. R-squared:                  0.734
Method:                 Least Squares   F-statistic:                     11.12
Date:                Wed, 31 Mar 2021   Prob (F-statistic):            0.00317
Time:                        16:12:47   Log-Likelihood:                -40.708
No. Observations:                  12   AIC:                             89.42
Df Residuals:                       8   BIC:                             91.36
Df Model:                           3
Covariance Type:            nonrobust
==============================================================================
                 coef    std err          t      P>|t|      [0.025      0.975]
------------------------------------------------------------------------------
Intercept   -2.22e-16      2.544  -8.73e-17      1.000      -5.866       5.866
P1\_cent        0.0107      0.004      2.390      0.044       0.000       0.021
P2\_cent        3.1348      0.844      3.712      0.006       1.188       5.082
P3\_cent        1.8674      0.535      3.494      0.008       0.635       3.100
==============================================================================
Omnibus:                        0.392   Durbin-Watson:                   1.043
Prob(Omnibus):                  0.822   Jarque-Bera (JB):                0.230
Skew:                          -0.282   Prob(JB):                        0.891
Kurtosis:                       2.625   Cond. No.                         574.
==============================================================================

Notes:
[1] Standard Errors assume that the covariance matrix of the errors is correctly
specified.
"""
\end{Verbatim}
\end{tcolorbox}
        
    \begin{tcolorbox}[breakable, size=fbox, boxrule=1pt, pad at break*=1mm,colback=cellbackground, colframe=cellborder]
\prompt{In}{incolor}{5}{\boxspacing}
\begin{Verbatim}[commandchars=\\\{\}]
\PY{c+c1}{\PYZsh{} 输出中心化后的多元线性回归方程}
\PY{n+nb}{print}\PY{p}{(}\PY{l+s+s1}{\PYZsq{}}\PY{l+s+s1}{Y\PYZus{}hat\PYZus{}cent = }\PY{l+s+si}{\PYZpc{}.2f}\PY{l+s+s1}{ + (}\PY{l+s+si}{\PYZpc{}.2f}\PY{l+s+s1}{ * X1\PYZus{}cent) + (}\PY{l+s+si}{\PYZpc{}.2f}\PY{l+s+s1}{ * X2\PYZus{}cent) + (}\PY{l+s+si}{\PYZpc{}.2f}\PY{l+s+s1}{ * X3\PYZus{}cent)}\PY{l+s+s1}{\PYZsq{}} \PY{o}{\PYZpc{}} \PY{p}{(}\PY{n}{beta\PYZus{}cent}\PY{p}{[}\PY{l+m+mi}{0}\PY{p}{]}\PY{p}{,} \PY{n}{beta\PYZus{}cent}\PY{p}{[}\PY{l+m+mi}{1}\PY{p}{]}\PY{p}{,} \PY{n}{beta\PYZus{}cent}\PY{p}{[}\PY{l+m+mi}{2}\PY{p}{]}\PY{p}{,} \PY{n}{beta\PYZus{}cent}\PY{p}{[}\PY{l+m+mi}{3}\PY{p}{]}\PY{p}{)}\PY{p}{)}
\end{Verbatim}
\end{tcolorbox}

    \begin{Verbatim}[commandchars=\\\{\}]
Y\_hat\_cent = -0.00 + (0.01 * X1\_cent) + (3.13 * X2\_cent) + (1.87 * X3\_cent)
    \end{Verbatim}

    通过对比数据中心化前后后的结果,我们可以发现,对数据进行中心化之后回归常数变为
0,而回归系数值未改变。

    \textbf{Q2:}

要验证模型是否合理,可以使用多元线性模型的显著性检验,可决系数的检验,以及残差检验等方法。

    \begin{tcolorbox}[breakable, size=fbox, boxrule=1pt, pad at break*=1mm,colback=cellbackground, colframe=cellborder]
\prompt{In}{incolor}{6}{\boxspacing}
\begin{Verbatim}[commandchars=\\\{\}]
\PY{c+c1}{\PYZsh{} 求解相关项}
\PY{n}{SSE} \PY{o}{=} \PY{n+nb}{sum}\PY{p}{(}\PY{p}{(}\PY{n}{Y} \PY{o}{\PYZhy{}} \PY{n}{Y\PYZus{}hat}\PY{p}{)} \PY{o}{*}\PY{o}{*} \PY{l+m+mi}{2}\PY{p}{)}
\PY{n}{SST} \PY{o}{=} \PY{n+nb}{sum}\PY{p}{(}\PY{p}{(}\PY{n}{Y} \PY{o}{\PYZhy{}} \PY{n}{Y\PYZus{}mean}\PY{p}{)} \PY{o}{*}\PY{o}{*} \PY{l+m+mi}{2}\PY{p}{)}
\PY{n}{SSR} \PY{o}{=} \PY{n+nb}{sum}\PY{p}{(}\PY{p}{(}\PY{n}{Y\PYZus{}hat} \PY{o}{\PYZhy{}} \PY{n}{Y\PYZus{}mean}\PY{p}{)} \PY{o}{*}\PY{o}{*} \PY{l+m+mi}{2}\PY{p}{)}

\PY{n}{sigma2} \PY{o}{=} \PY{n}{SSE} \PY{o}{/} \PY{p}{(}\PY{n}{n} \PY{o}{\PYZhy{}} \PY{n}{p} \PY{o}{\PYZhy{}} \PY{l+m+mi}{1}\PY{p}{)}  \PY{c+c1}{\PYZsh{} 方差}
\PY{n}{sigma} \PY{o}{=} \PY{n}{np}\PY{o}{.}\PY{n}{sqrt}\PY{p}{(}\PY{n}{sigma2}\PY{p}{)}  \PY{c+c1}{\PYZsh{} 标准差}

\PY{n}{c} \PY{o}{=} \PY{n}{np}\PY{o}{.}\PY{n}{dot}\PY{p}{(}\PY{n}{X}\PY{o}{.}\PY{n}{T}\PY{p}{,} \PY{n}{X}\PY{p}{)}
\PY{n}{C} \PY{o}{=} \PY{n}{np}\PY{o}{.}\PY{n}{linalg}\PY{o}{.}\PY{n}{inv}\PY{p}{(}\PY{n}{c}\PY{p}{)}  \PY{c+c1}{\PYZsh{} 求逆}
\PY{c+c1}{\PYZsh{} print(C)}
\end{Verbatim}
\end{tcolorbox}

    \textbf{多元线性模型的显著性检验------F 检验:}\\
检验假设:\(H_0: \beta_1=\beta_2=\beta_3=0\) vs
\(H_1: \exists \beta_i \neq 0,i=1,2,3\)

    \begin{tcolorbox}[breakable, size=fbox, boxrule=1pt, pad at break*=1mm,colback=cellbackground, colframe=cellborder]
\prompt{In}{incolor}{7}{\boxspacing}
\begin{Verbatim}[commandchars=\\\{\}]
\PY{c+c1}{\PYZsh{} 计算F0}
\PY{n}{F0} \PY{o}{=} \PY{p}{(}\PY{n}{SSR} \PY{o}{/} \PY{n}{p}\PY{p}{)} \PY{o}{/} \PY{p}{(}\PY{n}{SSE} \PY{o}{/} \PY{p}{(}\PY{n}{n} \PY{o}{\PYZhy{}} \PY{n}{p} \PY{o}{\PYZhy{}} \PY{l+m+mi}{1}\PY{p}{)}\PY{p}{)}
\PY{c+c1}{\PYZsh{} F0 = model.fvalue}
\PY{n+nb}{print}\PY{p}{(}\PY{l+s+s1}{\PYZsq{}}\PY{l+s+s1}{F0: }\PY{l+s+si}{\PYZpc{}.2f}\PY{l+s+s1}{\PYZsq{}} \PY{o}{\PYZpc{}} \PY{n}{F0}\PY{p}{)}
\PY{n}{F} \PY{o}{=} \PY{n+nb}{round}\PY{p}{(}\PY{n}{f}\PY{o}{.}\PY{n}{ppf}\PY{p}{(}\PY{l+m+mf}{0.95}\PY{p}{,} \PY{n}{dfn} \PY{o}{=} \PY{n}{p}\PY{p}{,} \PY{n}{dfd} \PY{o}{=} \PY{n}{n} \PY{o}{\PYZhy{}} \PY{n}{p} \PY{o}{\PYZhy{}} \PY{l+m+mi}{1}\PY{p}{)}\PY{p}{,} \PY{l+m+mi}{2}\PY{p}{)}

\PY{c+c1}{\PYZsh{} 法1:}
\PY{n}{pVal1} \PY{o}{=} \PY{n}{f}\PY{o}{.}\PY{n}{sf}\PY{p}{(}\PY{n}{F0}\PY{p}{,} \PY{n}{p}\PY{p}{,} \PY{n}{n} \PY{o}{\PYZhy{}} \PY{n}{p} \PY{o}{\PYZhy{}} \PY{l+m+mi}{1}\PY{p}{)}
\PY{c+c1}{\PYZsh{} pVal1 = model.f\PYZus{}pvalue}
\PY{n+nb}{print}\PY{p}{(}\PY{l+s+s1}{\PYZsq{}}\PY{l+s+s1}{pVal1: }\PY{l+s+si}{\PYZpc{}.2f}\PY{l+s+s1}{\PYZsq{}} \PY{o}{\PYZpc{}} \PY{n}{pVal1}\PY{p}{)}
\PY{k}{if} \PY{n}{pVal1} \PY{o}{\PYZlt{}} \PY{n}{alpha}\PY{p}{:}
    \PY{n+nb}{print}\PY{p}{(}\PY{l+s+s1}{\PYZsq{}}\PY{l+s+se}{\PYZbs{}n}\PY{l+s+s1}{Since p\PYZhy{}value \PYZlt{} 0.05, reject H0.}\PY{l+s+s1}{\PYZsq{}}\PY{p}{)}
\PY{k}{else}\PY{p}{:}
    \PY{n+nb}{print}\PY{p}{(}\PY{l+s+s1}{\PYZsq{}}\PY{l+s+se}{\PYZbs{}n}\PY{l+s+s1}{Accept H0.}\PY{l+s+s1}{\PYZsq{}}\PY{p}{)} 

\PY{c+c1}{\PYZsh{} 法2:}
\PY{k}{if} \PY{n}{F0} \PY{o}{\PYZgt{}} \PY{n}{F}\PY{p}{:}
    \PY{n+nb}{print}\PY{p}{(}\PY{l+s+s1}{\PYZsq{}}\PY{l+s+s1}{Since F0 \PYZgt{} F(0.95, 3, 8) = }\PY{l+s+si}{\PYZpc{}.2f}\PY{l+s+s1}{, reject H0.}\PY{l+s+s1}{\PYZsq{}} \PY{o}{\PYZpc{}} \PY{n}{F}\PY{p}{)}
\PY{k}{else}\PY{p}{:}
    \PY{n+nb}{print}\PY{p}{(}\PY{l+s+s1}{\PYZsq{}}\PY{l+s+s1}{Accept H0.}\PY{l+s+s1}{\PYZsq{}}\PY{p}{)} 
\end{Verbatim}
\end{tcolorbox}

    \begin{Verbatim}[commandchars=\\\{\}]
F0: 11.12
pVal1: 0.00

Since p-value < 0.05, reject H0.
Since F0 > F(0.95, 3, 8) = 4.07, reject H0.
    \end{Verbatim}

    由 F 检验结果可知,该回归方程是显著的。

    \textbf{多元线性模型的显著性检验------t 检验:}\\
检验假设:\(H_{0j}: \beta_j=0\) vs \(H_{1j}: \beta_j \neq 0,j=1,2,3\)

    \begin{tcolorbox}[breakable, size=fbox, boxrule=1pt, pad at break*=1mm,colback=cellbackground, colframe=cellborder]
\prompt{In}{incolor}{8}{\boxspacing}
\begin{Verbatim}[commandchars=\\\{\}]
\PY{c+c1}{\PYZsh{} t检验}
\PY{n}{t0} \PY{o}{=} \PY{p}{[}\PY{p}{]}
\PY{k}{for} \PY{n}{i} \PY{o+ow}{in} \PY{n+nb}{range}\PY{p}{(}\PY{n}{p} \PY{o}{+} \PY{l+m+mi}{1}\PY{p}{)}\PY{p}{:}   
    \PY{n}{t0}\PY{o}{.}\PY{n}{append}\PY{p}{(}\PY{n}{beta}\PY{p}{[}\PY{n}{i}\PY{p}{]} \PY{o}{/} \PY{p}{(}\PY{n}{np}\PY{o}{.}\PY{n}{sqrt}\PY{p}{(}\PY{n}{C}\PY{p}{[}\PY{n}{i}\PY{p}{]}\PY{p}{[}\PY{n}{i}\PY{p}{]} \PY{o}{*} \PY{n}{sigma2}\PY{p}{)}\PY{p}{)}\PY{p}{)}  \PY{c+c1}{\PYZsh{} 求t值}
\PY{c+c1}{\PYZsh{} t0 = model.tvalues}
\PY{n+nb}{print}\PY{p}{(}\PY{l+s+s1}{\PYZsq{}}\PY{l+s+s1}{t0值为:}\PY{l+s+s1}{\PYZsq{}}\PY{p}{,} \PY{n}{np}\PY{o}{.}\PY{n}{round}\PY{p}{(}\PY{n}{t0}\PY{p}{,} \PY{l+m+mi}{4}\PY{p}{)}\PY{p}{)}
\PY{n}{tVal} \PY{o}{=} \PY{n}{t}\PY{o}{.}\PY{n}{ppf}\PY{p}{(}\PY{l+m+mi}{1} \PY{o}{\PYZhy{}} \PY{n}{alpha} \PY{o}{/} \PY{l+m+mi}{2}\PY{p}{,} \PY{n}{n} \PY{o}{\PYZhy{}} \PY{n}{p} \PY{o}{\PYZhy{}} \PY{l+m+mi}{1}\PY{p}{)}
\PY{n+nb}{print}\PY{p}{(}\PY{l+s+s1}{\PYZsq{}}\PY{l+s+s1}{t的临界值为:}\PY{l+s+si}{\PYZpc{}.4f}\PY{l+s+s1}{\PYZsq{}} \PY{o}{\PYZpc{}} \PY{n}{tVal}\PY{p}{)}
\PY{n}{pVal2} \PY{o}{=} \PY{p}{[}\PY{p}{]}
\PY{k}{for} \PY{n}{i} \PY{o+ow}{in} \PY{n+nb}{range}\PY{p}{(}\PY{n}{p} \PY{o}{+} \PY{l+m+mi}{1}\PY{p}{)}\PY{p}{:}
    \PY{n}{P} \PY{o}{=} \PY{n}{t}\PY{o}{.}\PY{n}{sf}\PY{p}{(}\PY{n+nb}{abs}\PY{p}{(}\PY{n}{t0}\PY{p}{[}\PY{n}{i}\PY{p}{]}\PY{p}{)}\PY{p}{,} \PY{n}{n} \PY{o}{\PYZhy{}} \PY{n}{p} \PY{o}{\PYZhy{}} \PY{l+m+mi}{1}\PY{p}{)}
    \PY{n}{pVal2}\PY{o}{.}\PY{n}{append}\PY{p}{(}\PY{n}{P}\PY{p}{)}  \PY{c+c1}{\PYZsh{} 已知临界值求p值}
\PY{c+c1}{\PYZsh{} pVal2 = model.pvalues / 2}
\PY{n+nb}{print}\PY{p}{(}\PY{l+s+s1}{\PYZsq{}}\PY{l+s+s1}{P值为:}\PY{l+s+s1}{\PYZsq{}}\PY{p}{,} \PY{n}{np}\PY{o}{.}\PY{n}{round}\PY{p}{(}\PY{n}{pVal2}\PY{p}{,} \PY{l+m+mi}{4}\PY{p}{)}\PY{p}{)}

\PY{n+nb}{print}\PY{p}{(}\PY{l+s+s1}{\PYZsq{}}\PY{l+s+se}{\PYZbs{}n}\PY{l+s+s1}{\PYZsq{}}\PY{p}{)}
\PY{c+c1}{\PYZsh{} 法1:}
\PY{k}{for} \PY{n}{i} \PY{o+ow}{in} \PY{n+nb}{range}\PY{p}{(}\PY{n}{p}\PY{p}{)}\PY{p}{:}
    \PY{k}{if} \PY{n}{pVal2}\PY{p}{[}\PY{n}{i} \PY{o}{+} \PY{l+m+mi}{1}\PY{p}{]} \PY{o}{\PYZlt{}} \PY{n}{alpha}\PY{p}{:}
        \PY{n+nb}{print} \PY{p}{(}\PY{l+s+s1}{\PYZsq{}}\PY{l+s+s1}{Since p}\PY{l+s+si}{\PYZpc{}d}\PY{l+s+s1}{\PYZhy{}value \PYZlt{} 0.05, reject H0}\PY{l+s+si}{\PYZpc{}d}\PY{l+s+s1}{.}\PY{l+s+s1}{\PYZsq{}} \PY{o}{\PYZpc{}} \PY{p}{(}\PY{n}{i} \PY{o}{+} \PY{l+m+mi}{1}\PY{p}{,} \PY{n}{i} \PY{o}{+} \PY{l+m+mi}{1}\PY{p}{)}\PY{p}{)}
    \PY{k}{else}\PY{p}{:}
        \PY{n+nb}{print}\PY{p}{(}\PY{l+s+s1}{\PYZsq{}}\PY{l+s+s1}{Accept H0}\PY{l+s+si}{\PYZpc{}d}\PY{l+s+s1}{.}\PY{l+s+s1}{\PYZsq{}} \PY{o}{\PYZpc{}} \PY{p}{(}\PY{n}{i} \PY{o}{+} \PY{l+m+mi}{1}\PY{p}{)}\PY{p}{)}
\PY{n+nb}{print}\PY{p}{(}\PY{l+s+s1}{\PYZsq{}}\PY{l+s+se}{\PYZbs{}n}\PY{l+s+s1}{\PYZsq{}}\PY{p}{)}   

\PY{c+c1}{\PYZsh{} 法2:}
\PY{k}{for} \PY{n}{i} \PY{o+ow}{in} \PY{n+nb}{range}\PY{p}{(}\PY{n}{p}\PY{p}{)}\PY{p}{:}
    \PY{k}{if} \PY{n+nb}{abs}\PY{p}{(}\PY{n}{t0}\PY{p}{[}\PY{n}{i} \PY{o}{+} \PY{l+m+mi}{1}\PY{p}{]}\PY{p}{)} \PY{o}{\PYZgt{}} \PY{n}{tVal}\PY{p}{:}
        \PY{n+nb}{print}\PY{p}{(}\PY{l+s+s1}{\PYZsq{}}\PY{l+s+s1}{Since t0}\PY{l+s+si}{\PYZpc{}d}\PY{l+s+s1}{ \PYZgt{} t(0.975, 8) = }\PY{l+s+si}{\PYZpc{}.4f}\PY{l+s+s1}{, reject H0}\PY{l+s+si}{\PYZpc{}d}\PY{l+s+s1}{\PYZsq{}} \PY{o}{\PYZpc{}} \PY{p}{(}\PY{n}{i} \PY{o}{+} \PY{l+m+mi}{1}\PY{p}{,} \PY{n}{tVal}\PY{p}{,} \PY{n}{i} \PY{o}{+} \PY{l+m+mi}{1}\PY{p}{)}\PY{p}{)}
    \PY{k}{else}\PY{p}{:}
        \PY{n+nb}{print}\PY{p}{(}\PY{l+s+s1}{\PYZsq{}}\PY{l+s+s1}{Accept H0}\PY{l+s+si}{\PYZpc{}d}\PY{l+s+s1}{.}\PY{l+s+s1}{\PYZsq{}} \PY{o}{\PYZpc{}} \PY{p}{(}\PY{n}{i} \PY{o}{+} \PY{l+m+mi}{1}\PY{p}{)}\PY{p}{)} 
\end{Verbatim}
\end{tcolorbox}

    \begin{Verbatim}[commandchars=\\\{\}]
t0值为: [-2.8706  2.3896  3.7123  3.4936]
t的临界值为:2.3060
P值为: [0.0104 0.0219 0.003  0.0041]


Since p1-value < 0.05, reject H01.
Since p2-value < 0.05, reject H02.
Since p3-value < 0.05, reject H03.


Since t01 > t(0.975, 8) = 2.3060, reject H01
Since t02 > t(0.975, 8) = 2.3060, reject H02
Since t03 > t(0.975, 8) = 2.3060, reject H03
    \end{Verbatim}

    由 t 检验结果可知,各回归系数都是显著的。

    \textbf{复相关系数(可决系数)的检验:}

    \begin{tcolorbox}[breakable, size=fbox, boxrule=1pt, pad at break*=1mm,colback=cellbackground, colframe=cellborder]
\prompt{In}{incolor}{9}{\boxspacing}
\begin{Verbatim}[commandchars=\\\{\}]
\PY{c+c1}{\PYZsh{} 可决系数}
\PY{n}{R2} \PY{o}{=} \PY{n}{SSR} \PY{o}{/} \PY{n}{SST}
\PY{n+nb}{print}\PY{p}{(}\PY{l+s+s1}{\PYZsq{}}\PY{l+s+s1}{可决系数:}\PY{l+s+si}{\PYZpc{}.4f}\PY{l+s+s1}{\PYZsq{}} \PY{o}{\PYZpc{}} \PY{n}{R2}\PY{p}{)}

\PY{c+c1}{\PYZsh{} 调整可决系数}
\PY{n}{R2c} \PY{o}{=} \PY{l+m+mi}{1} \PY{o}{\PYZhy{}} \PY{p}{(}\PY{n}{SSE}\PY{o}{/}\PY{p}{(}\PY{n}{n}\PY{o}{\PYZhy{}}\PY{n}{p}\PY{o}{\PYZhy{}}\PY{l+m+mi}{1}\PY{p}{)}\PY{p}{)} \PY{o}{/} \PY{p}{(}\PY{n}{SST}\PY{o}{/}\PY{p}{(}\PY{n}{n}\PY{o}{\PYZhy{}}\PY{l+m+mi}{1}\PY{p}{)}\PY{p}{)}
\PY{n+nb}{print}\PY{p}{(}\PY{l+s+s1}{\PYZsq{}}\PY{l+s+s1}{调整可决系数:}\PY{l+s+si}{\PYZpc{}.4f}\PY{l+s+s1}{\PYZsq{}} \PY{o}{\PYZpc{}} \PY{n}{R2c}\PY{p}{)}
\end{Verbatim}
\end{tcolorbox}

    \begin{Verbatim}[commandchars=\\\{\}]
可决系数:0.8065
调整可决系数:0.7340
    \end{Verbatim}

    由于复相关系数(可决系数)值接近1,可以说明整体上 \(X_1,X_2\) 与 \(Y\)
呈线性关系。

    综上所述,Q1中所建的多元线性回归模型是合理的。

    \textbf{残差分析0: 计算制动马力的残差}

    \begin{tcolorbox}[breakable, size=fbox, boxrule=1pt, pad at break*=1mm,colback=cellbackground, colframe=cellborder]
\prompt{In}{incolor}{10}{\boxspacing}
\begin{Verbatim}[commandchars=\\\{\}]
\PY{c+c1}{\PYZsh{} 计算制动马力的残差}
\PY{n}{data\PYZus{}res} \PY{o}{=} \PY{n}{data} \PY{o}{*} \PY{l+m+mf}{1.0}  \PY{c+c1}{\PYZsh{} 乘1.0 不然会损失精度}
\PY{k}{for} \PY{n}{i} \PY{o+ow}{in} \PY{n+nb}{range}\PY{p}{(}\PY{n}{n}\PY{p}{)}\PY{p}{:}
    \PY{n}{data\PYZus{}res}\PY{p}{[}\PY{p}{:}\PY{p}{,} \PY{n}{p} \PY{o}{+} \PY{l+m+mi}{1}\PY{p}{]} \PY{o}{=} \PY{n}{Y} \PY{o}{\PYZhy{}} \PY{n}{Y\PYZus{}hat}
\PY{n}{df} \PY{o}{=} \PY{n}{pd}\PY{o}{.}\PY{n}{DataFrame}\PY{p}{(}\PY{n}{data\PYZus{}res}\PY{p}{,} \PY{n}{columns} \PY{o}{=} \PY{p}{[}\PY{l+s+s1}{\PYZsq{}}\PY{l+s+s1}{intercept}\PY{l+s+s1}{\PYZsq{}}\PY{p}{,} \PY{l+s+s1}{\PYZsq{}}\PY{l+s+s1}{P1}\PY{l+s+s1}{\PYZsq{}}\PY{p}{,} \PY{l+s+s1}{\PYZsq{}}\PY{l+s+s1}{P2}\PY{l+s+s1}{\PYZsq{}}\PY{p}{,} \PY{l+s+s1}{\PYZsq{}}\PY{l+s+s1}{P3}\PY{l+s+s1}{\PYZsq{}}\PY{p}{,} \PY{l+s+s1}{\PYZsq{}}\PY{l+s+s1}{F\PYZus{}res}\PY{l+s+s1}{\PYZsq{}}\PY{p}{]}\PY{p}{)}
\PY{n}{res} \PY{o}{=} \PY{n}{data\PYZus{}res}\PY{p}{[}\PY{p}{:}\PY{p}{,} \PY{n}{p} \PY{o}{+} \PY{l+m+mi}{1}\PY{p}{]}
\PY{c+c1}{\PYZsh{} res = model.resid}
\PY{n+nb}{print}\PY{p}{(}\PY{n}{df}\PY{o}{.}\PY{n}{head}\PY{p}{(}\PY{p}{)}\PY{p}{)}
\end{Verbatim}
\end{tcolorbox}

    \begin{Verbatim}[commandchars=\\\{\}]
   intercept      P1    P2     P3      F\_res
0        1.0  2000.0  90.0  100.0   0.731289
1        1.0  1800.0  94.0   95.0 -13.328247
2        1.0  2400.0  88.0  110.0 -11.958476
3        1.0  1900.0  91.0   96.0   3.137442
4        1.0  1600.0  86.0  100.0  11.555798
    \end{Verbatim}

    \textbf{残差分析1: 残差的正态概率图}\\
使用残差正态概率图可验证残差呈正态分布的假设。残差的正态概率图应该大致为一条直线。

    \begin{tcolorbox}[breakable, size=fbox, boxrule=1pt, pad at break*=1mm,colback=cellbackground, colframe=cellborder]
\prompt{In}{incolor}{11}{\boxspacing}
\begin{Verbatim}[commandchars=\\\{\}]
\PY{c+c1}{\PYZsh{} 残差的正态概率图}
\PY{n}{osm}\PY{p}{,} \PY{n}{osr} \PY{o}{=} \PY{n}{stats}\PY{o}{.}\PY{n}{probplot}\PY{p}{(}\PY{n}{res}\PY{p}{,} \PY{n}{dist} \PY{o}{=} \PY{l+s+s1}{\PYZsq{}}\PY{l+s+s1}{norm}\PY{l+s+s1}{\PYZsq{}}\PY{p}{,} \PY{n}{plot} \PY{o}{=} \PY{n}{plt}\PY{p}{)}
\PY{n}{x} \PY{o}{=} \PY{n}{osm}\PY{p}{[}\PY{l+m+mi}{0}\PY{p}{]}\PY{p}{[}\PY{l+m+mi}{0}\PY{p}{]}
\PY{n}{y} \PY{o}{=} \PY{n}{osm}\PY{p}{[}\PY{l+m+mi}{1}\PY{p}{]}\PY{p}{[}\PY{l+m+mi}{0}\PY{p}{]}
\PY{n}{plt}\PY{o}{.}\PY{n}{text}\PY{p}{(}\PY{n}{x}\PY{p}{,} \PY{n}{y}\PY{p}{,} \PY{l+s+s1}{\PYZsq{}}\PY{l+s+si}{\PYZpc{}.2f}\PY{l+s+s1}{\PYZsq{}} \PY{o}{\PYZpc{}} \PY{n+nb}{float}\PY{p}{(}\PY{n}{y}\PY{p}{)}\PY{p}{,} \PY{n}{ha}\PY{o}{=}\PY{l+s+s1}{\PYZsq{}}\PY{l+s+s1}{center}\PY{l+s+s1}{\PYZsq{}}\PY{p}{,} \PY{n}{va}\PY{o}{=} \PY{l+s+s1}{\PYZsq{}}\PY{l+s+s1}{bottom}\PY{l+s+s1}{\PYZsq{}}\PY{p}{,} \PY{n}{fontsize}\PY{o}{=}\PY{l+m+mi}{9}\PY{p}{)}
\PY{n}{plt}\PY{o}{.}\PY{n}{grid}\PY{p}{(}\PY{p}{)}
\PY{n}{plt}\PY{o}{.}\PY{n}{show}\PY{p}{(}\PY{p}{)}
\end{Verbatim}
\end{tcolorbox}

    \begin{center}
    \adjustimage{max size={0.9\linewidth}{0.9\paperheight}}{output_25_0.png}
    \end{center}
    { \hspace*{\fill} \\}
    
    \begin{tcolorbox}[breakable, size=fbox, boxrule=1pt, pad at break*=1mm,colback=cellbackground, colframe=cellborder]
\prompt{In}{incolor}{12}{\boxspacing}
\begin{Verbatim}[commandchars=\\\{\}]
\PY{c+c1}{\PYZsh{} 异常值检验}
\PY{n}{MSE} \PY{o}{=} \PY{n}{SSE} \PY{o}{/} \PY{p}{(}\PY{n}{n} \PY{o}{\PYZhy{}} \PY{n}{p} \PY{o}{\PYZhy{}} \PY{l+m+mi}{1}\PY{p}{)}
\PY{c+c1}{\PYZsh{} MSE = model.mse\PYZus{}resid}
\PY{n}{d} \PY{o}{=} \PY{n}{np}\PY{o}{.}\PY{n}{abs}\PY{p}{(}\PY{n}{y}\PY{p}{)} \PY{o}{/} \PY{n}{np}\PY{o}{.}\PY{n}{sqrt}\PY{p}{(}\PY{n}{MSE}\PY{p}{)}
\PY{k}{if} \PY{n}{d} \PY{o}{\PYZlt{}} \PY{l+m+mi}{3}\PY{p}{:}
    \PY{n+nb}{print}\PY{p}{(}\PY{l+s+s1}{\PYZsq{}}\PY{l+s+s1}{残差}\PY{l+s+s1}{\PYZsq{}}\PY{p}{,} \PY{n+nb}{round}\PY{p}{(}\PY{n}{y}\PY{p}{,} \PY{l+m+mi}{2}\PY{p}{)}\PY{p}{,} \PY{l+s+s1}{\PYZsq{}}\PY{l+s+s1}{不是异常值.}\PY{l+s+s1}{\PYZsq{}}\PY{p}{)}
\PY{k}{else}\PY{p}{:}
    \PY{n+nb}{print}\PY{p}{(}\PY{l+s+s1}{\PYZsq{}}\PY{l+s+s1}{残差}\PY{l+s+s1}{\PYZsq{}}\PY{p}{,} \PY{n+nb}{round}\PY{p}{(}\PY{n}{y}\PY{p}{,} \PY{l+m+mi}{2}\PY{p}{)}\PY{p}{,} \PY{l+s+s1}{\PYZsq{}}\PY{l+s+s1}{是异常值}\PY{l+s+s1}{\PYZsq{}}\PY{p}{)}
\end{Verbatim}
\end{tcolorbox}

    \begin{Verbatim}[commandchars=\\\{\}]
残差 -13.33 不是异常值.
    \end{Verbatim}

    从残差的正态概率图可以看出,这些点大致在一条直线附近,说明数据近似服从正态分布。

    \textbf{残差分析2: 残差与拟合值的关系图}\\
使用残差与拟合值图可验证``残差随机分布''和``具有常量方差''的假设。理想情况下,点应当在
0
的两端随机分布(模型的拟合点应该平均散布在被拟合值点附近),点中无可辨识的模式。\\
下表中的模式可能表示该模型不满足模型假设:

    \begin{longtable}[]{@{}cc@{}}
\toprule
模式 & 模式的含义 \\
\midrule
\endhead
残差相对拟合值呈扇形或不均匀分散 & 异方差 \\
曲线 & 缺少高阶项 \\
远离 0 的点 & 异常值 \\
在 X 方向远离其他点的点 & 有影响的点 \\
\bottomrule
\end{longtable}

    \begin{tcolorbox}[breakable, size=fbox, boxrule=1pt, pad at break*=1mm,colback=cellbackground, colframe=cellborder]
\prompt{In}{incolor}{13}{\boxspacing}
\begin{Verbatim}[commandchars=\\\{\}]
\PY{c+c1}{\PYZsh{} 残差与拟合值的关系图}
\PY{n}{plt}\PY{o}{.}\PY{n}{scatter}\PY{p}{(}\PY{n}{Y\PYZus{}hat}\PY{p}{,} \PY{n}{res}\PY{p}{,} \PY{n}{c} \PY{o}{=} \PY{l+s+s1}{\PYZsq{}}\PY{l+s+s1}{red}\PY{l+s+s1}{\PYZsq{}}\PY{p}{)}
\PY{n}{plt}\PY{o}{.}\PY{n}{title}\PY{p}{(}\PY{l+s+s1}{\PYZsq{}}\PY{l+s+s1}{Plot of residuals versus y\PYZus{}i}\PY{l+s+s1}{\PYZsq{}}\PY{p}{)}
\PY{n}{plt}\PY{o}{.}\PY{n}{xlabel}\PY{p}{(}\PY{l+s+s1}{\PYZsq{}}\PY{l+s+s1}{y\PYZus{}i}\PY{l+s+s1}{\PYZsq{}}\PY{p}{)}
\PY{n}{plt}\PY{o}{.}\PY{n}{ylabel}\PY{p}{(}\PY{l+s+s1}{\PYZsq{}}\PY{l+s+s1}{e\PYZus{}i}\PY{l+s+s1}{\PYZsq{}}\PY{p}{)}
\end{Verbatim}
\end{tcolorbox}

            \begin{tcolorbox}[breakable, size=fbox, boxrule=.5pt, pad at break*=1mm, opacityfill=0]
\prompt{Out}{outcolor}{13}{\boxspacing}
\begin{Verbatim}[commandchars=\\\{\}]
Text(0, 0.5, 'e\_i')
\end{Verbatim}
\end{tcolorbox}
        
    \begin{center}
    \adjustimage{max size={0.9\linewidth}{0.9\paperheight}}{output_30_1.png}
    \end{center}
    { \hspace*{\fill} \\}
    
    从残差与拟合值的关系图中,无法拒绝方差齐性的假设。所以不能认为模型不合理,即认为模型合理。

    【补充】方差分析可以看成特殊情况下的回归分析,方差分析就是解释变量全为分组的dummy(哑变量)的回归分析。\\
参考网址:https://zhuanlan.zhihu.com/p/129043279\\
方差齐性检验的本质:样本以及总体的方差分布是常数,和自变量或因变量没关系。\\
线性回归里面我们一般用残差图来检验方差齐性,画散点图是为了弄清因变量和残差之间有没有关系。这里并不是说不同指标的方差不同,因为这里的自变量不像ANOVA中是可以人为控制的(分组-取值),这里的自变量是\textbf{观察}到的,如果我们没法取到每个
x 值所有对应的 y
值,就不能说对不同的指标方差不同,只是说观测值的方差不同(也只是一种探测);我们这里只是用残差图去估计因变量的方差齐性(比较主观,看残差分布是否与拟合值没有明显的趋势关系)。\\
当然我们并不能通过残差的分布来证明因变量的方差齐性,我们只是提出了``方差齐性''的假设,然后画出残差图,它杂乱无章,不相关,看起来是独立的,所以我们不能推翻这个假设,就认为观测值的方差是齐性的。\\
参考网址:https://blog.csdn.net/weixin\_33448555/article/details/112650361

    \textbf{Q3:}

本题需要使用估计与预测的方法确定预测得到的制动马力。可以使用\(E(y_0)\)的估计或\(y_0\)的预测区间作为制动马力的大致区间。

    \textbf{关于 \(E(y_0)\) 的估计:}

    \begin{tcolorbox}[breakable, size=fbox, boxrule=1pt, pad at break*=1mm,colback=cellbackground, colframe=cellborder]
\prompt{In}{incolor}{14}{\boxspacing}
\begin{Verbatim}[commandchars=\\\{\}]
\PY{c+c1}{\PYZsh{} 给定 x\PYZus{}0,求 E(y\PYZus{}0) 的估计值}
\PY{k}{def} \PY{n+nf}{confidence\PYZus{}interval}\PY{p}{(}\PY{n}{x0}\PY{p}{)}\PY{p}{:}
    \PY{n}{x0} \PY{o}{=} \PY{n}{np}\PY{o}{.}\PY{n}{array}\PY{p}{(}\PY{n}{x0}\PY{p}{)}
    \PY{n}{Y0} \PY{o}{=} \PY{n}{np}\PY{o}{.}\PY{n}{dot}\PY{p}{(}\PY{n}{x0}\PY{o}{.}\PY{n}{T}\PY{p}{,} \PY{n}{beta}\PY{p}{)}
    \PY{n}{delta0} \PY{o}{=} \PY{n}{tVal} \PY{o}{*} \PY{n}{sigma} \PY{o}{*} \PY{n}{np}\PY{o}{.}\PY{n}{sqrt}\PY{p}{(}\PY{n}{x0}\PY{o}{.}\PY{n}{T} \PY{o}{@} \PY{n}{C} \PY{o}{@} \PY{n}{x0}\PY{p}{)}
    \PY{n}{Y0\PYZus{}int} \PY{o}{=} \PY{p}{[}\PY{n}{Y0} \PY{o}{\PYZhy{}} \PY{n}{delta0}\PY{p}{,} \PY{n}{Y0} \PY{o}{+} \PY{n}{delta0}\PY{p}{]}
    \PY{k}{return} \PY{n}{Y0\PYZus{}int}

\PY{n}{x0} \PY{o}{=} \PY{p}{[}\PY{l+m+mi}{1}\PY{p}{]}
\PY{k}{for} \PY{n}{i} \PY{o+ow}{in} \PY{n+nb}{range}\PY{p}{(}\PY{n}{p}\PY{p}{)}\PY{p}{:}
    \PY{n}{x0}\PY{o}{.}\PY{n}{append}\PY{p}{(}\PY{n+nb}{int}\PY{p}{(}\PY{n+nb}{input}\PY{p}{(}\PY{p}{)}\PY{p}{)}\PY{p}{)}
\PY{n+nb}{print}\PY{p}{(}\PY{l+s+s1}{\PYZsq{}}\PY{l+s+s1}{给定 x = }\PY{l+s+s1}{\PYZsq{}}\PY{p}{,} \PY{n}{x0}\PY{p}{,} \PY{l+s+s1}{\PYZsq{}}\PY{l+s+s1}{, E(y\PYZus{}0) 的置信区间:}\PY{l+s+s1}{\PYZsq{}}\PY{p}{,} \PY{n}{np}\PY{o}{.}\PY{n}{round}\PY{p}{(}\PY{n}{confidence\PYZus{}interval}\PY{p}{(}\PY{n}{x0}\PY{p}{)}\PY{p}{,} \PY{l+m+mi}{4}\PY{p}{)}\PY{p}{)}
\end{Verbatim}
\end{tcolorbox}

    \begin{Verbatim}[commandchars=\\\{\}]
3000
90
100
给定 x =  [1, 3000, 90, 100] , E(y\_0) 的置信区间: [226.2457 243.7181]
    \end{Verbatim}

    \textbf{关于 \(y_0\) 的预测:}

    \begin{tcolorbox}[breakable, size=fbox, boxrule=1pt, pad at break*=1mm,colback=cellbackground, colframe=cellborder]
\prompt{In}{incolor}{15}{\boxspacing}
\begin{Verbatim}[commandchars=\\\{\}]
\PY{c+c1}{\PYZsh{} 给定 x\PYZus{}0,求 y\PYZus{}0 的预测区间}
\PY{k}{def} \PY{n+nf}{confidence\PYZus{}interval}\PY{p}{(}\PY{n}{x0}\PY{p}{)}\PY{p}{:}
    \PY{n}{x0} \PY{o}{=} \PY{n}{np}\PY{o}{.}\PY{n}{array}\PY{p}{(}\PY{n}{x0}\PY{p}{)}
    \PY{n}{Y0} \PY{o}{=} \PY{n}{np}\PY{o}{.}\PY{n}{dot}\PY{p}{(}\PY{n}{x0}\PY{o}{.}\PY{n}{T}\PY{p}{,} \PY{n}{beta}\PY{p}{)}
    \PY{n}{delta1} \PY{o}{=} \PY{n}{tVal} \PY{o}{*} \PY{n}{sigma} \PY{o}{*} \PY{n}{np}\PY{o}{.}\PY{n}{sqrt}\PY{p}{(}\PY{l+m+mi}{1} \PY{o}{+} \PY{n}{x0}\PY{o}{.}\PY{n}{T} \PY{o}{@} \PY{n}{C} \PY{o}{@} \PY{n}{x0}\PY{p}{)}
    \PY{n}{Y0\PYZus{}int} \PY{o}{=} \PY{p}{[}\PY{n}{Y0} \PY{o}{\PYZhy{}} \PY{n}{delta1}\PY{p}{,} \PY{n}{Y0} \PY{o}{+} \PY{n}{delta1}\PY{p}{]}
    \PY{k}{return} \PY{n}{Y0\PYZus{}int}

\PY{n}{x0\PYZus{}} \PY{o}{=} \PY{p}{[}\PY{l+m+mi}{1}\PY{p}{]}
\PY{k}{for} \PY{n}{i} \PY{o+ow}{in} \PY{n+nb}{range}\PY{p}{(}\PY{n}{p}\PY{p}{)}\PY{p}{:}
    \PY{n}{x0\PYZus{}}\PY{o}{.}\PY{n}{append}\PY{p}{(}\PY{n+nb}{int}\PY{p}{(}\PY{n+nb}{input}\PY{p}{(}\PY{p}{)}\PY{p}{)}\PY{p}{)}
\PY{n+nb}{print}\PY{p}{(}\PY{l+s+s1}{\PYZsq{}}\PY{l+s+s1}{给定 x = }\PY{l+s+s1}{\PYZsq{}}\PY{p}{,} \PY{n}{x0\PYZus{}}\PY{p}{,} \PY{l+s+s1}{\PYZsq{}}\PY{l+s+s1}{, y\PYZus{}0 的预测区间:}\PY{l+s+s1}{\PYZsq{}}\PY{p}{,} \PY{n}{np}\PY{o}{.}\PY{n}{round}\PY{p}{(}\PY{n}{confidence\PYZus{}interval}\PY{p}{(}\PY{n}{x0\PYZus{}}\PY{p}{)}\PY{p}{,} \PY{l+m+mi}{4}\PY{p}{)}\PY{p}{)}
\end{Verbatim}
\end{tcolorbox}

    \begin{Verbatim}[commandchars=\\\{\}]
3000
90
100
给定 x =  [1, 3000, 90, 100] , y\_0 的预测区间: [212.8622 257.1016]
    \end{Verbatim}

    由于计算公式不同,计算得到的\(y_0\)的预测区间要略宽于\(E[y_0]\)的置信区间。这里采用\(y_0\)的预测区间作为预测得到的制动马力,即\([212.86,\ 257.10]\)。


    % Add a bibliography block to the postdoc
    
    
    
\end{document}
