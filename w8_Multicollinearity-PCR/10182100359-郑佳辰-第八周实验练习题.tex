%!TEX program = xelatex
\documentclass[11pt]{ctexart}

    \usepackage[breakable]{tcolorbox}
    \usepackage{parskip} % Stop auto-indenting (to mimic markdown behaviour)
    
    \usepackage{iftex}
    \ifPDFTeX
    	\usepackage[T1]{fontenc}
    	\usepackage{mathpazo}
    \else
    	\usepackage{fontspec}
    \fi

    % Basic figure setup, for now with no caption control since it's done
    % automatically by Pandoc (which extracts ![](path) syntax from Markdown).
    \usepackage{graphicx}
    % Maintain compatibility with old templates. Remove in nbconvert 6.0
    \let\Oldincludegraphics\includegraphics
    % Ensure that by default, figures have no caption (until we provide a
    % proper Figure object with a Caption API and a way to capture that
    % in the conversion process - todo).
    \usepackage{caption}
    \DeclareCaptionFormat{nocaption}{}
    \captionsetup{format=nocaption,aboveskip=0pt,belowskip=0pt}

    \usepackage{float}
    \floatplacement{figure}{H} % forces figures to be placed at the correct location
    \usepackage{xcolor} % Allow colors to be defined
    \usepackage{enumerate} % Needed for markdown enumerations to work
    \usepackage{geometry} % Used to adjust the document margins
    \usepackage{amsmath} % Equations
    \usepackage{amssymb} % Equations
    \usepackage{textcomp} % defines textquotesingle
    % Hack from http://tex.stackexchange.com/a/47451/13684:
    \AtBeginDocument{%
        \def\PYZsq{\textquotesingle}% Upright quotes in Pygmentized code
    }
    \usepackage{upquote} % Upright quotes for verbatim code
    \usepackage{eurosym} % defines \euro
    \usepackage[mathletters]{ucs} % Extended unicode (utf-8) support
    \usepackage{fancyvrb} % verbatim replacement that allows latex
    \usepackage{grffile} % extends the file name processing of package graphics 
                         % to support a larger range
    \makeatletter % fix for old versions of grffile with XeLaTeX
    \@ifpackagelater{grffile}{2019/11/01}
    {
      % Do nothing on new versions
    }
    {
      \def\Gread@@xetex#1{%
        \IfFileExists{"\Gin@base".bb}%
        {\Gread@eps{\Gin@base.bb}}%
        {\Gread@@xetex@aux#1}%
      }
    }
    \makeatother
    \usepackage[Export]{adjustbox} % Used to constrain images to a maximum size
    \adjustboxset{max size={0.9\linewidth}{0.9\paperheight}}

    % The hyperref package gives us a pdf with properly built
    % internal navigation ('pdf bookmarks' for the table of contents,
    % internal cross-reference links, web links for URLs, etc.)
    \usepackage{hyperref}
    % The default LaTeX title has an obnoxious amount of whitespace. By default,
    % titling removes some of it. It also provides customization options.
    \usepackage{titling}
    \usepackage{longtable} % longtable support required by pandoc >1.10
    \usepackage{booktabs}  % table support for pandoc > 1.12.2
    \usepackage[inline]{enumitem} % IRkernel/repr support (it uses the enumerate* environment)
    \usepackage[normalem]{ulem} % ulem is needed to support strikethroughs (\sout)
                                % normalem makes italics be italics, not underlines
    \usepackage{mathrsfs}
    

    
    % Colors for the hyperref package
    \definecolor{urlcolor}{rgb}{0,.145,.698}
    \definecolor{linkcolor}{rgb}{.71,0.21,0.01}
    \definecolor{citecolor}{rgb}{.12,.54,.11}

    % ANSI colors
    \definecolor{ansi-black}{HTML}{3E424D}
    \definecolor{ansi-black-intense}{HTML}{282C36}
    \definecolor{ansi-red}{HTML}{E75C58}
    \definecolor{ansi-red-intense}{HTML}{B22B31}
    \definecolor{ansi-green}{HTML}{00A250}
    \definecolor{ansi-green-intense}{HTML}{007427}
    \definecolor{ansi-yellow}{HTML}{DDB62B}
    \definecolor{ansi-yellow-intense}{HTML}{B27D12}
    \definecolor{ansi-blue}{HTML}{208FFB}
    \definecolor{ansi-blue-intense}{HTML}{0065CA}
    \definecolor{ansi-magenta}{HTML}{D160C4}
    \definecolor{ansi-magenta-intense}{HTML}{A03196}
    \definecolor{ansi-cyan}{HTML}{60C6C8}
    \definecolor{ansi-cyan-intense}{HTML}{258F8F}
    \definecolor{ansi-white}{HTML}{C5C1B4}
    \definecolor{ansi-white-intense}{HTML}{A1A6B2}
    \definecolor{ansi-default-inverse-fg}{HTML}{FFFFFF}
    \definecolor{ansi-default-inverse-bg}{HTML}{000000}

    % common color for the border for error outputs.
    \definecolor{outerrorbackground}{HTML}{FFDFDF}

    % commands and environments needed by pandoc snippets
    % extracted from the output of `pandoc -s`
    \providecommand{\tightlist}{%
      \setlength{\itemsep}{0pt}\setlength{\parskip}{0pt}}
    \DefineVerbatimEnvironment{Highlighting}{Verbatim}{commandchars=\\\{\}}
    % Add ',fontsize=\small' for more characters per line
    \newenvironment{Shaded}{}{}
    \newcommand{\KeywordTok}[1]{\textcolor[rgb]{0.00,0.44,0.13}{\textbf{{#1}}}}
    \newcommand{\DataTypeTok}[1]{\textcolor[rgb]{0.56,0.13,0.00}{{#1}}}
    \newcommand{\DecValTok}[1]{\textcolor[rgb]{0.25,0.63,0.44}{{#1}}}
    \newcommand{\BaseNTok}[1]{\textcolor[rgb]{0.25,0.63,0.44}{{#1}}}
    \newcommand{\FloatTok}[1]{\textcolor[rgb]{0.25,0.63,0.44}{{#1}}}
    \newcommand{\CharTok}[1]{\textcolor[rgb]{0.25,0.44,0.63}{{#1}}}
    \newcommand{\StringTok}[1]{\textcolor[rgb]{0.25,0.44,0.63}{{#1}}}
    \newcommand{\CommentTok}[1]{\textcolor[rgb]{0.38,0.63,0.69}{\textit{{#1}}}}
    \newcommand{\OtherTok}[1]{\textcolor[rgb]{0.00,0.44,0.13}{{#1}}}
    \newcommand{\AlertTok}[1]{\textcolor[rgb]{1.00,0.00,0.00}{\textbf{{#1}}}}
    \newcommand{\FunctionTok}[1]{\textcolor[rgb]{0.02,0.16,0.49}{{#1}}}
    \newcommand{\RegionMarkerTok}[1]{{#1}}
    \newcommand{\ErrorTok}[1]{\textcolor[rgb]{1.00,0.00,0.00}{\textbf{{#1}}}}
    \newcommand{\NormalTok}[1]{{#1}}
    
    % Additional commands for more recent versions of Pandoc
    \newcommand{\ConstantTok}[1]{\textcolor[rgb]{0.53,0.00,0.00}{{#1}}}
    \newcommand{\SpecialCharTok}[1]{\textcolor[rgb]{0.25,0.44,0.63}{{#1}}}
    \newcommand{\VerbatimStringTok}[1]{\textcolor[rgb]{0.25,0.44,0.63}{{#1}}}
    \newcommand{\SpecialStringTok}[1]{\textcolor[rgb]{0.73,0.40,0.53}{{#1}}}
    \newcommand{\ImportTok}[1]{{#1}}
    \newcommand{\DocumentationTok}[1]{\textcolor[rgb]{0.73,0.13,0.13}{\textit{{#1}}}}
    \newcommand{\AnnotationTok}[1]{\textcolor[rgb]{0.38,0.63,0.69}{\textbf{\textit{{#1}}}}}
    \newcommand{\CommentVarTok}[1]{\textcolor[rgb]{0.38,0.63,0.69}{\textbf{\textit{{#1}}}}}
    \newcommand{\VariableTok}[1]{\textcolor[rgb]{0.10,0.09,0.49}{{#1}}}
    \newcommand{\ControlFlowTok}[1]{\textcolor[rgb]{0.00,0.44,0.13}{\textbf{{#1}}}}
    \newcommand{\OperatorTok}[1]{\textcolor[rgb]{0.40,0.40,0.40}{{#1}}}
    \newcommand{\BuiltInTok}[1]{{#1}}
    \newcommand{\ExtensionTok}[1]{{#1}}
    \newcommand{\PreprocessorTok}[1]{\textcolor[rgb]{0.74,0.48,0.00}{{#1}}}
    \newcommand{\AttributeTok}[1]{\textcolor[rgb]{0.49,0.56,0.16}{{#1}}}
    \newcommand{\InformationTok}[1]{\textcolor[rgb]{0.38,0.63,0.69}{\textbf{\textit{{#1}}}}}
    \newcommand{\WarningTok}[1]{\textcolor[rgb]{0.38,0.63,0.69}{\textbf{\textit{{#1}}}}}
    
    
    % Define a nice break command that doesn't care if a line doesn't already
    % exist.
    \def\br{\hspace*{\fill} \\* }
    % Math Jax compatibility definitions
    \def\gt{>}
    \def\lt{<}
    \let\Oldtex\TeX
    \let\Oldlatex\LaTeX
    \renewcommand{\TeX}{\textrm{\Oldtex}}
    \renewcommand{\LaTeX}{\textrm{\Oldlatex}}
    % Document parameters
    % Document title
    \title{10182100359-郑佳辰-第八周实验练习题}
    
    
    
    
    
% Pygments definitions
\makeatletter
\def\PY@reset{\let\PY@it=\relax \let\PY@bf=\relax%
    \let\PY@ul=\relax \let\PY@tc=\relax%
    \let\PY@bc=\relax \let\PY@ff=\relax}
\def\PY@tok#1{\csname PY@tok@#1\endcsname}
\def\PY@toks#1+{\ifx\relax#1\empty\else%
    \PY@tok{#1}\expandafter\PY@toks\fi}
\def\PY@do#1{\PY@bc{\PY@tc{\PY@ul{%
    \PY@it{\PY@bf{\PY@ff{#1}}}}}}}
\def\PY#1#2{\PY@reset\PY@toks#1+\relax+\PY@do{#2}}

\expandafter\def\csname PY@tok@w\endcsname{\def\PY@tc##1{\textcolor[rgb]{0.73,0.73,0.73}{##1}}}
\expandafter\def\csname PY@tok@c\endcsname{\let\PY@it=\textit\def\PY@tc##1{\textcolor[rgb]{0.25,0.50,0.50}{##1}}}
\expandafter\def\csname PY@tok@cp\endcsname{\def\PY@tc##1{\textcolor[rgb]{0.74,0.48,0.00}{##1}}}
\expandafter\def\csname PY@tok@k\endcsname{\let\PY@bf=\textbf\def\PY@tc##1{\textcolor[rgb]{0.00,0.50,0.00}{##1}}}
\expandafter\def\csname PY@tok@kp\endcsname{\def\PY@tc##1{\textcolor[rgb]{0.00,0.50,0.00}{##1}}}
\expandafter\def\csname PY@tok@kt\endcsname{\def\PY@tc##1{\textcolor[rgb]{0.69,0.00,0.25}{##1}}}
\expandafter\def\csname PY@tok@o\endcsname{\def\PY@tc##1{\textcolor[rgb]{0.40,0.40,0.40}{##1}}}
\expandafter\def\csname PY@tok@ow\endcsname{\let\PY@bf=\textbf\def\PY@tc##1{\textcolor[rgb]{0.67,0.13,1.00}{##1}}}
\expandafter\def\csname PY@tok@nb\endcsname{\def\PY@tc##1{\textcolor[rgb]{0.00,0.50,0.00}{##1}}}
\expandafter\def\csname PY@tok@nf\endcsname{\def\PY@tc##1{\textcolor[rgb]{0.00,0.00,1.00}{##1}}}
\expandafter\def\csname PY@tok@nc\endcsname{\let\PY@bf=\textbf\def\PY@tc##1{\textcolor[rgb]{0.00,0.00,1.00}{##1}}}
\expandafter\def\csname PY@tok@nn\endcsname{\let\PY@bf=\textbf\def\PY@tc##1{\textcolor[rgb]{0.00,0.00,1.00}{##1}}}
\expandafter\def\csname PY@tok@ne\endcsname{\let\PY@bf=\textbf\def\PY@tc##1{\textcolor[rgb]{0.82,0.25,0.23}{##1}}}
\expandafter\def\csname PY@tok@nv\endcsname{\def\PY@tc##1{\textcolor[rgb]{0.10,0.09,0.49}{##1}}}
\expandafter\def\csname PY@tok@no\endcsname{\def\PY@tc##1{\textcolor[rgb]{0.53,0.00,0.00}{##1}}}
\expandafter\def\csname PY@tok@nl\endcsname{\def\PY@tc##1{\textcolor[rgb]{0.63,0.63,0.00}{##1}}}
\expandafter\def\csname PY@tok@ni\endcsname{\let\PY@bf=\textbf\def\PY@tc##1{\textcolor[rgb]{0.60,0.60,0.60}{##1}}}
\expandafter\def\csname PY@tok@na\endcsname{\def\PY@tc##1{\textcolor[rgb]{0.49,0.56,0.16}{##1}}}
\expandafter\def\csname PY@tok@nt\endcsname{\let\PY@bf=\textbf\def\PY@tc##1{\textcolor[rgb]{0.00,0.50,0.00}{##1}}}
\expandafter\def\csname PY@tok@nd\endcsname{\def\PY@tc##1{\textcolor[rgb]{0.67,0.13,1.00}{##1}}}
\expandafter\def\csname PY@tok@s\endcsname{\def\PY@tc##1{\textcolor[rgb]{0.73,0.13,0.13}{##1}}}
\expandafter\def\csname PY@tok@sd\endcsname{\let\PY@it=\textit\def\PY@tc##1{\textcolor[rgb]{0.73,0.13,0.13}{##1}}}
\expandafter\def\csname PY@tok@si\endcsname{\let\PY@bf=\textbf\def\PY@tc##1{\textcolor[rgb]{0.73,0.40,0.53}{##1}}}
\expandafter\def\csname PY@tok@se\endcsname{\let\PY@bf=\textbf\def\PY@tc##1{\textcolor[rgb]{0.73,0.40,0.13}{##1}}}
\expandafter\def\csname PY@tok@sr\endcsname{\def\PY@tc##1{\textcolor[rgb]{0.73,0.40,0.53}{##1}}}
\expandafter\def\csname PY@tok@ss\endcsname{\def\PY@tc##1{\textcolor[rgb]{0.10,0.09,0.49}{##1}}}
\expandafter\def\csname PY@tok@sx\endcsname{\def\PY@tc##1{\textcolor[rgb]{0.00,0.50,0.00}{##1}}}
\expandafter\def\csname PY@tok@m\endcsname{\def\PY@tc##1{\textcolor[rgb]{0.40,0.40,0.40}{##1}}}
\expandafter\def\csname PY@tok@gh\endcsname{\let\PY@bf=\textbf\def\PY@tc##1{\textcolor[rgb]{0.00,0.00,0.50}{##1}}}
\expandafter\def\csname PY@tok@gu\endcsname{\let\PY@bf=\textbf\def\PY@tc##1{\textcolor[rgb]{0.50,0.00,0.50}{##1}}}
\expandafter\def\csname PY@tok@gd\endcsname{\def\PY@tc##1{\textcolor[rgb]{0.63,0.00,0.00}{##1}}}
\expandafter\def\csname PY@tok@gi\endcsname{\def\PY@tc##1{\textcolor[rgb]{0.00,0.63,0.00}{##1}}}
\expandafter\def\csname PY@tok@gr\endcsname{\def\PY@tc##1{\textcolor[rgb]{1.00,0.00,0.00}{##1}}}
\expandafter\def\csname PY@tok@ge\endcsname{\let\PY@it=\textit}
\expandafter\def\csname PY@tok@gs\endcsname{\let\PY@bf=\textbf}
\expandafter\def\csname PY@tok@gp\endcsname{\let\PY@bf=\textbf\def\PY@tc##1{\textcolor[rgb]{0.00,0.00,0.50}{##1}}}
\expandafter\def\csname PY@tok@go\endcsname{\def\PY@tc##1{\textcolor[rgb]{0.53,0.53,0.53}{##1}}}
\expandafter\def\csname PY@tok@gt\endcsname{\def\PY@tc##1{\textcolor[rgb]{0.00,0.27,0.87}{##1}}}
\expandafter\def\csname PY@tok@err\endcsname{\def\PY@bc##1{\setlength{\fboxsep}{0pt}\fcolorbox[rgb]{1.00,0.00,0.00}{1,1,1}{\strut ##1}}}
\expandafter\def\csname PY@tok@kc\endcsname{\let\PY@bf=\textbf\def\PY@tc##1{\textcolor[rgb]{0.00,0.50,0.00}{##1}}}
\expandafter\def\csname PY@tok@kd\endcsname{\let\PY@bf=\textbf\def\PY@tc##1{\textcolor[rgb]{0.00,0.50,0.00}{##1}}}
\expandafter\def\csname PY@tok@kn\endcsname{\let\PY@bf=\textbf\def\PY@tc##1{\textcolor[rgb]{0.00,0.50,0.00}{##1}}}
\expandafter\def\csname PY@tok@kr\endcsname{\let\PY@bf=\textbf\def\PY@tc##1{\textcolor[rgb]{0.00,0.50,0.00}{##1}}}
\expandafter\def\csname PY@tok@bp\endcsname{\def\PY@tc##1{\textcolor[rgb]{0.00,0.50,0.00}{##1}}}
\expandafter\def\csname PY@tok@fm\endcsname{\def\PY@tc##1{\textcolor[rgb]{0.00,0.00,1.00}{##1}}}
\expandafter\def\csname PY@tok@vc\endcsname{\def\PY@tc##1{\textcolor[rgb]{0.10,0.09,0.49}{##1}}}
\expandafter\def\csname PY@tok@vg\endcsname{\def\PY@tc##1{\textcolor[rgb]{0.10,0.09,0.49}{##1}}}
\expandafter\def\csname PY@tok@vi\endcsname{\def\PY@tc##1{\textcolor[rgb]{0.10,0.09,0.49}{##1}}}
\expandafter\def\csname PY@tok@vm\endcsname{\def\PY@tc##1{\textcolor[rgb]{0.10,0.09,0.49}{##1}}}
\expandafter\def\csname PY@tok@sa\endcsname{\def\PY@tc##1{\textcolor[rgb]{0.73,0.13,0.13}{##1}}}
\expandafter\def\csname PY@tok@sb\endcsname{\def\PY@tc##1{\textcolor[rgb]{0.73,0.13,0.13}{##1}}}
\expandafter\def\csname PY@tok@sc\endcsname{\def\PY@tc##1{\textcolor[rgb]{0.73,0.13,0.13}{##1}}}
\expandafter\def\csname PY@tok@dl\endcsname{\def\PY@tc##1{\textcolor[rgb]{0.73,0.13,0.13}{##1}}}
\expandafter\def\csname PY@tok@s2\endcsname{\def\PY@tc##1{\textcolor[rgb]{0.73,0.13,0.13}{##1}}}
\expandafter\def\csname PY@tok@sh\endcsname{\def\PY@tc##1{\textcolor[rgb]{0.73,0.13,0.13}{##1}}}
\expandafter\def\csname PY@tok@s1\endcsname{\def\PY@tc##1{\textcolor[rgb]{0.73,0.13,0.13}{##1}}}
\expandafter\def\csname PY@tok@mb\endcsname{\def\PY@tc##1{\textcolor[rgb]{0.40,0.40,0.40}{##1}}}
\expandafter\def\csname PY@tok@mf\endcsname{\def\PY@tc##1{\textcolor[rgb]{0.40,0.40,0.40}{##1}}}
\expandafter\def\csname PY@tok@mh\endcsname{\def\PY@tc##1{\textcolor[rgb]{0.40,0.40,0.40}{##1}}}
\expandafter\def\csname PY@tok@mi\endcsname{\def\PY@tc##1{\textcolor[rgb]{0.40,0.40,0.40}{##1}}}
\expandafter\def\csname PY@tok@il\endcsname{\def\PY@tc##1{\textcolor[rgb]{0.40,0.40,0.40}{##1}}}
\expandafter\def\csname PY@tok@mo\endcsname{\def\PY@tc##1{\textcolor[rgb]{0.40,0.40,0.40}{##1}}}
\expandafter\def\csname PY@tok@ch\endcsname{\let\PY@it=\textit\def\PY@tc##1{\textcolor[rgb]{0.25,0.50,0.50}{##1}}}
\expandafter\def\csname PY@tok@cm\endcsname{\let\PY@it=\textit\def\PY@tc##1{\textcolor[rgb]{0.25,0.50,0.50}{##1}}}
\expandafter\def\csname PY@tok@cpf\endcsname{\let\PY@it=\textit\def\PY@tc##1{\textcolor[rgb]{0.25,0.50,0.50}{##1}}}
\expandafter\def\csname PY@tok@c1\endcsname{\let\PY@it=\textit\def\PY@tc##1{\textcolor[rgb]{0.25,0.50,0.50}{##1}}}
\expandafter\def\csname PY@tok@cs\endcsname{\let\PY@it=\textit\def\PY@tc##1{\textcolor[rgb]{0.25,0.50,0.50}{##1}}}

\def\PYZbs{\char`\\}
\def\PYZus{\char`\_}
\def\PYZob{\char`\{}
\def\PYZcb{\char`\}}
\def\PYZca{\char`\^}
\def\PYZam{\char`\&}
\def\PYZlt{\char`\<}
\def\PYZgt{\char`\>}
\def\PYZsh{\char`\#}
\def\PYZpc{\char`\%}
\def\PYZdl{\char`\$}
\def\PYZhy{\char`\-}
\def\PYZsq{\char`\'}
\def\PYZdq{\char`\"}
\def\PYZti{\char`\~}
% for compatibility with earlier versions
\def\PYZat{@}
\def\PYZlb{[}
\def\PYZrb{]}
\makeatother


    % For linebreaks inside Verbatim environment from package fancyvrb. 
    \makeatletter
        \newbox\Wrappedcontinuationbox 
        \newbox\Wrappedvisiblespacebox 
        \newcommand*\Wrappedvisiblespace {\textcolor{red}{\textvisiblespace}} 
        \newcommand*\Wrappedcontinuationsymbol {\textcolor{red}{\llap{\tiny$\m@th\hookrightarrow$}}} 
        \newcommand*\Wrappedcontinuationindent {3ex } 
        \newcommand*\Wrappedafterbreak {\kern\Wrappedcontinuationindent\copy\Wrappedcontinuationbox} 
        % Take advantage of the already applied Pygments mark-up to insert 
        % potential linebreaks for TeX processing. 
        %        {, <, #, %, $, ' and ": go to next line. 
        %        _, }, ^, &, >, - and ~: stay at end of broken line. 
        % Use of \textquotesingle for straight quote. 
        \newcommand*\Wrappedbreaksatspecials {% 
            \def\PYGZus{\discretionary{\char`\_}{\Wrappedafterbreak}{\char`\_}}% 
            \def\PYGZob{\discretionary{}{\Wrappedafterbreak\char`\{}{\char`\{}}% 
            \def\PYGZcb{\discretionary{\char`\}}{\Wrappedafterbreak}{\char`\}}}% 
            \def\PYGZca{\discretionary{\char`\^}{\Wrappedafterbreak}{\char`\^}}% 
            \def\PYGZam{\discretionary{\char`\&}{\Wrappedafterbreak}{\char`\&}}% 
            \def\PYGZlt{\discretionary{}{\Wrappedafterbreak\char`\<}{\char`\<}}% 
            \def\PYGZgt{\discretionary{\char`\>}{\Wrappedafterbreak}{\char`\>}}% 
            \def\PYGZsh{\discretionary{}{\Wrappedafterbreak\char`\#}{\char`\#}}% 
            \def\PYGZpc{\discretionary{}{\Wrappedafterbreak\char`\%}{\char`\%}}% 
            \def\PYGZdl{\discretionary{}{\Wrappedafterbreak\char`\$}{\char`\$}}% 
            \def\PYGZhy{\discretionary{\char`\-}{\Wrappedafterbreak}{\char`\-}}% 
            \def\PYGZsq{\discretionary{}{\Wrappedafterbreak\textquotesingle}{\textquotesingle}}% 
            \def\PYGZdq{\discretionary{}{\Wrappedafterbreak\char`\"}{\char`\"}}% 
            \def\PYGZti{\discretionary{\char`\~}{\Wrappedafterbreak}{\char`\~}}% 
        } 
        % Some characters . , ; ? ! / are not pygmentized. 
        % This macro makes them "active" and they will insert potential linebreaks 
        \newcommand*\Wrappedbreaksatpunct {% 
            \lccode`\~`\.\lowercase{\def~}{\discretionary{\hbox{\char`\.}}{\Wrappedafterbreak}{\hbox{\char`\.}}}% 
            \lccode`\~`\,\lowercase{\def~}{\discretionary{\hbox{\char`\,}}{\Wrappedafterbreak}{\hbox{\char`\,}}}% 
            \lccode`\~`\;\lowercase{\def~}{\discretionary{\hbox{\char`\;}}{\Wrappedafterbreak}{\hbox{\char`\;}}}% 
            \lccode`\~`\:\lowercase{\def~}{\discretionary{\hbox{\char`\:}}{\Wrappedafterbreak}{\hbox{\char`\:}}}% 
            \lccode`\~`\?\lowercase{\def~}{\discretionary{\hbox{\char`\?}}{\Wrappedafterbreak}{\hbox{\char`\?}}}% 
            \lccode`\~`\!\lowercase{\def~}{\discretionary{\hbox{\char`\!}}{\Wrappedafterbreak}{\hbox{\char`\!}}}% 
            \lccode`\~`\/\lowercase{\def~}{\discretionary{\hbox{\char`\/}}{\Wrappedafterbreak}{\hbox{\char`\/}}}% 
            \catcode`\.\active
            \catcode`\,\active 
            \catcode`\;\active
            \catcode`\:\active
            \catcode`\?\active
            \catcode`\!\active
            \catcode`\/\active 
            \lccode`\~`\~ 	
        }
    \makeatother

    \let\OriginalVerbatim=\Verbatim
    \makeatletter
    \renewcommand{\Verbatim}[1][1]{%
        %\parskip\z@skip
        \sbox\Wrappedcontinuationbox {\Wrappedcontinuationsymbol}%
        \sbox\Wrappedvisiblespacebox {\FV@SetupFont\Wrappedvisiblespace}%
        \def\FancyVerbFormatLine ##1{\hsize\linewidth
            \vtop{\raggedright\hyphenpenalty\z@\exhyphenpenalty\z@
                \doublehyphendemerits\z@\finalhyphendemerits\z@
                \strut ##1\strut}%
        }%
        % If the linebreak is at a space, the latter will be displayed as visible
        % space at end of first line, and a continuation symbol starts next line.
        % Stretch/shrink are however usually zero for typewriter font.
        \def\FV@Space {%
            \nobreak\hskip\z@ plus\fontdimen3\font minus\fontdimen4\font
            \discretionary{\copy\Wrappedvisiblespacebox}{\Wrappedafterbreak}
            {\kern\fontdimen2\font}%
        }%
        
        % Allow breaks at special characters using \PYG... macros.
        \Wrappedbreaksatspecials
        % Breaks at punctuation characters . , ; ? ! and / need catcode=\active 	
        \OriginalVerbatim[#1,codes*=\Wrappedbreaksatpunct]%
    }
    \makeatother

    % Exact colors from NB
    \definecolor{incolor}{HTML}{303F9F}
    \definecolor{outcolor}{HTML}{D84315}
    \definecolor{cellborder}{HTML}{CFCFCF}
    \definecolor{cellbackground}{HTML}{F7F7F7}
    
    % prompt
    \makeatletter
    \newcommand{\boxspacing}{\kern\kvtcb@left@rule\kern\kvtcb@boxsep}
    \makeatother
    \newcommand{\prompt}[4]{
        {\ttfamily\llap{{\color{#2}[#3]:\hspace{3pt}#4}}\vspace{-\baselineskip}}
    }
    

    
    % Prevent overflowing lines due to hard-to-break entities
    \sloppy 
    % Setup hyperref package
    \hypersetup{
      breaklinks=true,  % so long urls are correctly broken across lines
      colorlinks=true,
      urlcolor=urlcolor,
      linkcolor=linkcolor,
      citecolor=citecolor,
      }
    % Slightly bigger margins than the latex defaults
    
    \geometry{verbose,tmargin=1in,bmargin=1in,lmargin=1in,rmargin=1in}
    
    

\begin{document}
    
    \maketitle
    
    

    
    \hypertarget{week8-multicollinearity-pcr}{%
\section{Week8
Multicollinearity-PCR}\label{week8-multicollinearity-pcr}}

\hypertarget{ux80ccux666fux63cfux8ff0}{%
\subsection{背景描述}\label{ux80ccux666fux63cfux8ff0}}

数据集来源:Longley's(1967)

我们构造了 16 个观测的 6 个自变量,具体请见下表:

\hypertarget{ux6570ux636eux63cfux8ff0}{%
\subsection{数据描述}\label{ux6570ux636eux63cfux8ff0}}

\begin{longtable}[]{llll}
\toprule
变量名 & 变量含义 & 变量类型 & 变量取值范围 \\
\midrule
\endhead
(自变量)X1 & 国民生产总值隐含价格评价指数(1954=100) & continuous
variable & \(\mathbb{R}^+\) \\
(自变量)X2 & 国民生产总值 & continuous variable & \(\mathbb{R}^+\) \\
(自变量)X3 & 失业人数 & continuous variable & \(\mathbb{R}^+\) \\
(自变量)X4 & 武装力量的规模 & continuous variable &
\(\mathbb{R}^+\) \\
(自变量)X5 & 14 岁及以上的非机构人口 & continuous variable &
\(\mathbb{R}^+\) \\
(自变量)X6 & 时间(年份) & continuous variable & \(\mathbb{R}^+\) \\
(因变量)Y & 总就业人数 & continuous variable & \(\mathbb{R}^+\) \\
\bottomrule
\end{longtable}

    \hypertarget{ux95eeux9898}{%
\subsection{问题}\label{ux95eeux9898}}

注:这里使用 \(\alpha=0.05\) 的显著性水平

\begin{enumerate}
\def\labelenumi{\arabic{enumi}.}
\tightlist
\item
  判断所给数据是否具有多重共线性.
\item
  若具有多重共线性, 选择适当的主成分.
\item
  对降维后的数据进行回归分析.
\end{enumerate}

\hypertarget{ux89e3ux51b3ux65b9ux6848}{%
\subsection{解决方案}\label{ux89e3ux51b3ux65b9ux6848}}

    \textbf{Q1:}

多重共线性是指自变量\(x_1, x_2, ... ,x_p\)之间不完全线性相关但是相关性很高的情况。此时,虽然可以得到最小二乘估计,但是精度很低。随着自变量之间相关性增加,最小二乘估计结果的方差会增大。

    \begin{tcolorbox}[breakable, size=fbox, boxrule=1pt, pad at break*=1mm,colback=cellbackground, colframe=cellborder]
\prompt{In}{incolor}{1}{\boxspacing}
\begin{Verbatim}[commandchars=\\\{\}]
\PY{c+c1}{\PYZsh{} Import standard packages}
\PY{k+kn}{import} \PY{n+nn}{numpy} \PY{k}{as} \PY{n+nn}{np}
\PY{k+kn}{import} \PY{n+nn}{pandas} \PY{k}{as} \PY{n+nn}{pd}
\PY{k+kn}{import} \PY{n+nn}{scipy}\PY{n+nn}{.}\PY{n+nn}{stats} \PY{k}{as} \PY{n+nn}{stats}
\PY{k+kn}{import} \PY{n+nn}{matplotlib}\PY{n+nn}{.}\PY{n+nn}{pyplot} \PY{k}{as} \PY{n+nn}{plt}
\PY{k+kn}{import} \PY{n+nn}{math}

\PY{c+c1}{\PYZsh{} Import additional packages}
\PY{k+kn}{from} \PY{n+nn}{itertools} \PY{k+kn}{import} \PY{n}{combinations}
\PY{k+kn}{import} \PY{n+nn}{statsmodels}\PY{n+nn}{.}\PY{n+nn}{api} \PY{k}{as} \PY{n+nn}{sm}
\PY{k+kn}{from} \PY{n+nn}{statsmodels}\PY{n+nn}{.}\PY{n+nn}{stats}\PY{n+nn}{.}\PY{n+nn}{outliers\PYZus{}influence} \PY{k+kn}{import} \PY{n}{variance\PYZus{}inflation\PYZus{}factor}
\PY{k+kn}{from} \PY{n+nn}{sklearn}\PY{n+nn}{.}\PY{n+nn}{decomposition} \PY{k+kn}{import} \PY{n}{PCA}  \PY{c+c1}{\PYZsh{} 进行主成分分析}

\PY{n}{alpha} \PY{o}{=} \PY{l+m+mf}{0.05}
\PY{n}{p} \PY{o}{=} \PY{l+m+mi}{6}
\PY{n}{n} \PY{o}{=} \PY{l+m+mi}{16}

\PY{n}{x} \PY{o}{=} \PY{n}{pd}\PY{o}{.}\PY{n}{read\PYZus{}csv}\PY{p}{(}\PY{l+s+s1}{\PYZsq{}}\PY{l+s+s1}{Project8.csv}\PY{l+s+s1}{\PYZsq{}}\PY{p}{)}
\PY{n}{x}\PY{o}{.}\PY{n}{insert}\PY{p}{(}\PY{l+m+mi}{0}\PY{p}{,} \PY{l+s+s1}{\PYZsq{}}\PY{l+s+s1}{intercept}\PY{l+s+s1}{\PYZsq{}}\PY{p}{,} \PY{n}{np}\PY{o}{.}\PY{n}{ones}\PY{p}{(}\PY{n+nb}{len}\PY{p}{(}\PY{n}{x}\PY{p}{)}\PY{p}{)}\PY{p}{)} 
\PY{n}{data} \PY{o}{=} \PY{n}{x}\PY{o}{.}\PY{n}{values} \PY{o}{*} \PY{l+m+mf}{1.0}
\PY{n}{df} \PY{o}{=} \PY{n}{pd}\PY{o}{.}\PY{n}{DataFrame}\PY{p}{(}\PY{n}{data}\PY{p}{)}
\PY{n+nb}{print}\PY{p}{(}\PY{n}{df}\PY{o}{.}\PY{n}{head}\PY{p}{(}\PY{p}{)}\PY{p}{)}

\PY{c+c1}{\PYZsh{} 对数据进行分割}
\PY{n}{X} \PY{o}{=} \PY{n}{data}\PY{p}{[}\PY{p}{:}\PY{p}{,}\PY{l+m+mi}{0}\PY{p}{:}\PY{n}{p}\PY{o}{+}\PY{l+m+mi}{1}\PY{p}{]}
\PY{n}{Y} \PY{o}{=} \PY{n}{data}\PY{p}{[}\PY{p}{:}\PY{p}{,}\PY{o}{\PYZhy{}}\PY{l+m+mi}{1}\PY{p}{]}
\end{Verbatim}
\end{tcolorbox}

    \begin{Verbatim}[commandchars=\\\{\}]
     0      1         2       3       4         5       6        7
0  1.0  830.0  234289.0  2356.0  1590.0  107608.0  1947.0  60323.0
1  1.0  885.0  259426.0  2325.0  1456.0  108632.0  1948.0  61122.0
2  1.0  882.0  258054.0  3682.0  1616.0  109773.0  1949.0  60171.0
3  1.0  895.0  284599.0  3351.0  1650.0  110929.0  1950.0  61187.0
4  1.0  962.0  328975.0  2099.0  3099.0  112075.0  1951.0  63221.0
    \end{Verbatim}

    \begin{tcolorbox}[breakable, size=fbox, boxrule=1pt, pad at break*=1mm,colback=cellbackground, colframe=cellborder]
\prompt{In}{incolor}{2}{\boxspacing}
\begin{Verbatim}[commandchars=\\\{\}]
\PY{c+c1}{\PYZsh{} Do the multiple linear regression——对原始数据}
\PY{c+c1}{\PYZsh{} OLS(endog,exog=None,missing=\PYZsq{}none\PYZsq{},hasconst=None) (endog:因变量,exog=自变量)}
\PY{n}{model} \PY{o}{=} \PY{n}{sm}\PY{o}{.}\PY{n}{OLS}\PY{p}{(}\PY{n}{Y}\PY{p}{,} \PY{n}{X}\PY{p}{)}\PY{o}{.}\PY{n}{fit}\PY{p}{(}\PY{p}{)}
\PY{n}{beta} \PY{o}{=} \PY{n}{model}\PY{o}{.}\PY{n}{params}
\PY{n}{model}\PY{o}{.}\PY{n}{summary}\PY{p}{(}\PY{p}{)}
\end{Verbatim}
\end{tcolorbox}

    \begin{Verbatim}[commandchars=\\\{\}]
/Library/Frameworks/Python.framework/Versions/3.6/lib/python3.6/site-
packages/scipy/stats/stats.py:1604: UserWarning: kurtosistest only valid for
n>=20 {\ldots} continuing anyway, n=16
  "anyway, n=\%i" \% int(n))
    \end{Verbatim}

            \begin{tcolorbox}[breakable, size=fbox, boxrule=.5pt, pad at break*=1mm, opacityfill=0]
\prompt{Out}{outcolor}{2}{\boxspacing}
\begin{Verbatim}[commandchars=\\\{\}]
<class 'statsmodels.iolib.summary.Summary'>
"""
                            OLS Regression Results
==============================================================================
Dep. Variable:                      y   R-squared:                       0.995
Model:                            OLS   Adj. R-squared:                  0.992
Method:                 Least Squares   F-statistic:                     330.3
Date:                Tue, 20 Apr 2021   Prob (F-statistic):           4.98e-10
Time:                        22:19:01   Log-Likelihood:                -109.62
No. Observations:                  16   AIC:                             233.2
Df Residuals:                       9   BIC:                             238.6
Df Model:                           6
Covariance Type:            nonrobust
==============================================================================
                 coef    std err          t      P>|t|      [0.025      0.975]
------------------------------------------------------------------------------
const      -3.482e+06    8.9e+05     -3.911      0.004    -5.5e+06   -1.47e+06
x1             1.5062      8.491      0.177      0.863     -17.703      20.715
x2            -0.0358      0.033     -1.070      0.313      -0.112       0.040
x3            -2.0202      0.488     -4.136      0.003      -3.125      -0.915
x4            -1.0332      0.214     -4.822      0.001      -1.518      -0.549
x5            -0.0511      0.226     -0.226      0.826      -0.563       0.460
x6          1829.1515    455.478      4.016      0.003     798.788    2859.515
==============================================================================
Omnibus:                        0.749   Durbin-Watson:                   2.559
Prob(Omnibus):                  0.688   Jarque-Bera (JB):                0.684
Skew:                           0.420   Prob(JB):                        0.710
Kurtosis:                       2.434   Cond. No.                     4.86e+09
==============================================================================

Notes:
[1] Standard Errors assume that the covariance matrix of the errors is correctly
specified.
[2] The condition number is large, 4.86e+09. This might indicate that there are
strong multicollinearity or other numerical problems.
"""
\end{Verbatim}
\end{tcolorbox}
        
    \textbf{数据预处理:}

    \begin{tcolorbox}[breakable, size=fbox, boxrule=1pt, pad at break*=1mm,colback=cellbackground, colframe=cellborder]
\prompt{In}{incolor}{3}{\boxspacing}
\begin{Verbatim}[commandchars=\\\{\}]
\PY{c+c1}{\PYZsh{} 对数据进行标准化}
\PY{c+c1}{\PYZsh{} 自变量 X 的均值}
\PY{n}{X\PYZus{}mean} \PY{o}{=} \PY{p}{[}\PY{p}{]}
\PY{k}{for} \PY{n}{i} \PY{o+ow}{in} \PY{n+nb}{range}\PY{p}{(}\PY{n}{p}\PY{p}{)}\PY{p}{:}
    \PY{n}{X\PYZus{}mean}\PY{o}{.}\PY{n}{append}\PY{p}{(}\PY{n}{np}\PY{o}{.}\PY{n}{mean}\PY{p}{(}\PY{n}{X}\PY{p}{[}\PY{p}{:}\PY{p}{,} \PY{n}{i}\PY{o}{+}\PY{l+m+mi}{1}\PY{p}{]}\PY{p}{)}\PY{p}{)} 

\PY{c+c1}{\PYZsh{} 自变量 X 的标准差}
\PY{n}{X\PYZus{}L} \PY{o}{=} \PY{p}{[}\PY{p}{]}
\PY{k}{for} \PY{n}{i} \PY{o+ow}{in} \PY{n+nb}{range}\PY{p}{(}\PY{n}{p}\PY{p}{)}\PY{p}{:}
    \PY{n}{X\PYZus{}L}\PY{o}{.}\PY{n}{append}\PY{p}{(}\PY{n+nb}{sum}\PY{p}{(}\PY{p}{(}\PY{n}{X}\PY{p}{[}\PY{p}{:}\PY{p}{,} \PY{n}{i}\PY{o}{+}\PY{l+m+mi}{1}\PY{p}{]} \PY{o}{\PYZhy{}} \PY{n}{X\PYZus{}mean}\PY{p}{[}\PY{n}{i}\PY{p}{]}\PY{p}{)} \PY{o}{*}\PY{o}{*} \PY{l+m+mi}{2}\PY{p}{)}\PY{p}{)}  

\PY{c+c1}{\PYZsh{} 对自变量 X 标准化(截距项不用标准化)}
\PY{n}{X\PYZus{}std} \PY{o}{=} \PY{n}{X} \PY{o}{*} \PY{l+m+mf}{1.0}
\PY{n}{X\PYZus{}std}\PY{p}{[}\PY{p}{:}\PY{p}{,}\PY{l+m+mi}{1}\PY{p}{:}\PY{n}{p}\PY{o}{+}\PY{l+m+mi}{1}\PY{p}{]} \PY{o}{=} \PY{p}{(}\PY{n}{X}\PY{p}{[}\PY{p}{:}\PY{p}{,}\PY{l+m+mi}{1}\PY{p}{:}\PY{n}{p}\PY{o}{+}\PY{l+m+mi}{1}\PY{p}{]} \PY{o}{\PYZhy{}} \PY{n}{X\PYZus{}mean}\PY{p}{)} \PY{o}{/} \PY{n}{np}\PY{o}{.}\PY{n}{sqrt}\PY{p}{(}\PY{n}{X\PYZus{}L}\PY{p}{)}

\PY{c+c1}{\PYZsh{} 对因变量 Y 标准化}
\PY{n}{Y\PYZus{}std} \PY{o}{=} \PY{p}{(}\PY{n}{Y} \PY{o}{\PYZhy{}} \PY{n}{np}\PY{o}{.}\PY{n}{mean}\PY{p}{(}\PY{n}{Y}\PY{p}{)}\PY{p}{)} \PY{o}{/} \PY{n}{np}\PY{o}{.}\PY{n}{sqrt}\PY{p}{(}\PY{n+nb}{sum}\PY{p}{(}\PY{p}{(}\PY{n}{Y} \PY{o}{\PYZhy{}} \PY{n}{np}\PY{o}{.}\PY{n}{mean}\PY{p}{(}\PY{n}{Y}\PY{p}{)}\PY{p}{)}\PY{o}{*}\PY{o}{*}\PY{l+m+mi}{2}\PY{p}{)}\PY{p}{)}

\PY{n}{df\PYZus{}std} \PY{o}{=} \PY{n}{pd}\PY{o}{.}\PY{n}{DataFrame}\PY{p}{(}\PY{n}{X\PYZus{}std}\PY{p}{)}
\PY{n}{df\PYZus{}std}\PY{p}{[}\PY{l+s+s1}{\PYZsq{}}\PY{l+s+s1}{Y}\PY{l+s+s1}{\PYZsq{}}\PY{p}{]} \PY{o}{=} \PY{n}{Y\PYZus{}std}
\PY{n+nb}{print}\PY{p}{(}\PY{n}{df\PYZus{}std}\PY{p}{)}
\end{Verbatim}
\end{tcolorbox}

    \begin{Verbatim}[commandchars=\\\{\}]
      0         1         2         3         4         5         6         Y
0   1.0 -0.446968 -0.398513 -0.231355 -0.377210 -0.364354 -0.406745 -0.367157
1   1.0 -0.315375 -0.333214 -0.239921 -0.426926 -0.326344 -0.352512 -0.308415
2   1.0 -0.322553 -0.336778  0.135028 -0.367563 -0.283992 -0.298279 -0.378332
3   1.0 -0.291449 -0.267822  0.043570 -0.354949 -0.241084 -0.244047 -0.303636
4   1.0 -0.131144 -0.152546 -0.302366  0.182657 -0.198546 -0.189814 -0.154097
5   1.0 -0.085685 -0.105725 -0.348509  0.366311 -0.154190 -0.135582 -0.123366
6   1.0 -0.064152 -0.057964 -0.365640  0.348873 -0.086486 -0.081349 -0.024114
7   1.0 -0.040226 -0.063868  0.106292  0.275783 -0.044728 -0.027116 -0.114397
8   1.0 -0.011514  0.025381 -0.079939  0.163735 -0.001336  0.027116  0.051611
9   1.0  0.069834  0.081780 -0.102596  0.092871  0.048625  0.081349  0.186740
10  1.0  0.160753  0.143057 -0.071097  0.070980  0.112134  0.135582  0.209678
11  1.0  0.218175  0.147673  0.411058  0.011246  0.167998  0.189814  0.087930
12  1.0  0.261242  0.246797  0.171224 -0.020290  0.220557  0.244047  0.245409
13  1.0  0.299524  0.298483  0.203828 -0.034389  0.294868  0.298279  0.312238
14  1.0  0.335413  0.338935  0.445597 -0.012870  0.387070  0.352512  0.295108
15  1.0  0.364124  0.434325  0.224827  0.081740  0.469807  0.406745  0.384802
    \end{Verbatim}

    \textbf{做多元线性回归分析:}

先后对未经标准化和已标准化的数据进行回归分析,得到的\(\hat{\beta}\)分别如表所示。

    \begin{tcolorbox}[breakable, size=fbox, boxrule=1pt, pad at break*=1mm,colback=cellbackground, colframe=cellborder]
\prompt{In}{incolor}{4}{\boxspacing}
\begin{Verbatim}[commandchars=\\\{\}]
\PY{c+c1}{\PYZsh{} Do the multiple linear regression——对标准化后的数据}
\PY{n}{model\PYZus{}std} \PY{o}{=} \PY{n}{sm}\PY{o}{.}\PY{n}{OLS}\PY{p}{(}\PY{n}{Y\PYZus{}std}\PY{p}{,} \PY{n}{X\PYZus{}std}\PY{p}{)}\PY{o}{.}\PY{n}{fit}\PY{p}{(}\PY{p}{)}
\PY{n}{beta\PYZus{}std} \PY{o}{=} \PY{n}{model\PYZus{}std}\PY{o}{.}\PY{n}{params}
\PY{n}{model\PYZus{}std}\PY{o}{.}\PY{n}{summary}\PY{p}{(}\PY{p}{)}
\end{Verbatim}
\end{tcolorbox}

    \begin{Verbatim}[commandchars=\\\{\}]
/Library/Frameworks/Python.framework/Versions/3.6/lib/python3.6/site-
packages/scipy/stats/stats.py:1604: UserWarning: kurtosistest only valid for
n>=20 {\ldots} continuing anyway, n=16
  "anyway, n=\%i" \% int(n))
    \end{Verbatim}

            \begin{tcolorbox}[breakable, size=fbox, boxrule=.5pt, pad at break*=1mm, opacityfill=0]
\prompt{Out}{outcolor}{4}{\boxspacing}
\begin{Verbatim}[commandchars=\\\{\}]
<class 'statsmodels.iolib.summary.Summary'>
"""
                            OLS Regression Results
==============================================================================
Dep. Variable:                      y   R-squared:                       0.995
Model:                            OLS   Adj. R-squared:                  0.992
Method:                 Least Squares   F-statistic:                     330.3
Date:                Tue, 20 Apr 2021   Prob (F-statistic):           4.98e-10
Time:                        22:19:01   Log-Likelihood:                 42.670
No. Observations:                  16   AIC:                            -71.34
Df Residuals:                       9   BIC:                            -65.93
Df Model:                           6
Covariance Type:            nonrobust
==============================================================================
                 coef    std err          t      P>|t|      [0.025      0.975]
------------------------------------------------------------------------------
const      -1.128e-17      0.006  -2.01e-15      1.000      -0.013       0.013
x1             0.0463      0.261      0.177      0.863      -0.544       0.637
x2            -1.0137      0.948     -1.070      0.313      -3.158       1.130
x3            -0.5375      0.130     -4.136      0.003      -0.832      -0.244
x4            -0.2047      0.042     -4.822      0.001      -0.301      -0.109
x5            -0.1012      0.448     -0.226      0.826      -1.114       0.912
x6             2.4797      0.617      4.016      0.003       1.083       3.876
==============================================================================
Omnibus:                        0.749   Durbin-Watson:                   2.559
Prob(Omnibus):                  0.688   Jarque-Bera (JB):                0.684
Skew:                           0.420   Prob(JB):                        0.710
Kurtosis:                       2.434   Cond. No.                         206.
==============================================================================

Notes:
[1] Standard Errors assume that the covariance matrix of the errors is correctly
specified.
"""
\end{Verbatim}
\end{tcolorbox}
        
    \textbf{求 \((X^*)^{'}(X^*)\) 矩阵的特征值和特征向量:}

如果求得的特征值至少一个非常接近0,则可以认为存在多重共线性。

    \begin{tcolorbox}[breakable, size=fbox, boxrule=1pt, pad at break*=1mm,colback=cellbackground, colframe=cellborder]
\prompt{In}{incolor}{5}{\boxspacing}
\begin{Verbatim}[commandchars=\\\{\}]
\PY{c+c1}{\PYZsh{} (X*)\PYZsq{}(X*) 矩阵等价于原始矩阵 X 样本相关矩阵}
\PY{n}{R} \PY{o}{=} \PY{n}{df}\PY{o}{.}\PY{n}{corr}\PY{p}{(}\PY{p}{)}
\PY{n}{R} \PY{o}{=} \PY{n}{R}\PY{o}{.}\PY{n}{iloc}\PY{p}{[}\PY{l+m+mi}{1}\PY{p}{:}\PY{o}{\PYZhy{}}\PY{l+m+mi}{1}\PY{p}{,}\PY{l+m+mi}{1}\PY{p}{:}\PY{o}{\PYZhy{}}\PY{l+m+mi}{1}\PY{p}{]}

\PY{c+c1}{\PYZsh{} 求 (X*)\PYZsq{}(X*) 矩阵,结果与样本相关矩阵一致}
\PY{c+c1}{\PYZsh{} R1 = np.dot(X\PYZus{}std.T,X\PYZus{}std) }
\PY{c+c1}{\PYZsh{} R1 = pd.DataFrame(R1[1:,1:])}

\PY{n}{R}
\end{Verbatim}
\end{tcolorbox}

            \begin{tcolorbox}[breakable, size=fbox, boxrule=.5pt, pad at break*=1mm, opacityfill=0]
\prompt{Out}{outcolor}{5}{\boxspacing}
\begin{Verbatim}[commandchars=\\\{\}]
          1         2         3         4         5         6
1  1.000000  0.991589  0.620633  0.464744  0.979163  0.991149
2  0.991589  1.000000  0.604261  0.446437  0.991090  0.995273
3  0.620633  0.604261  1.000000 -0.177421  0.686552  0.668257
4  0.464744  0.446437 -0.177421  1.000000  0.364416  0.417245
5  0.979163  0.991090  0.686552  0.364416  1.000000  0.993953
6  0.991149  0.995273  0.668257  0.417245  0.993953  1.000000
\end{Verbatim}
\end{tcolorbox}
        
    \begin{tcolorbox}[breakable, size=fbox, boxrule=1pt, pad at break*=1mm,colback=cellbackground, colframe=cellborder]
\prompt{In}{incolor}{6}{\boxspacing}
\begin{Verbatim}[commandchars=\\\{\}]
\PY{c+c1}{\PYZsh{} 求特征值 \PYZam{} 特征向量}
\PY{n}{W}\PY{p}{,} \PY{n}{V} \PY{o}{=} \PY{n}{np}\PY{o}{.}\PY{n}{linalg}\PY{o}{.}\PY{n}{eig}\PY{p}{(}\PY{n}{R}\PY{p}{)}
\PY{n}{W\PYZus{}diag} \PY{o}{=} \PY{n}{np}\PY{o}{.}\PY{n}{diag}\PY{p}{(}\PY{n}{W}\PY{p}{)}
\PY{n}{V} \PY{o}{=} \PY{n}{V}\PY{o}{.}\PY{n}{T} \PY{c+c1}{\PYZsh{} 这里需要转置}

\PY{n+nb}{print}\PY{p}{(}\PY{l+s+s1}{\PYZsq{}}\PY{l+s+s1}{特征值:}\PY{l+s+s1}{\PYZsq{}}\PY{p}{,} \PY{n}{W}\PY{p}{)}
\PY{c+c1}{\PYZsh{} print(W\PYZus{}diag)}
\PY{c+c1}{\PYZsh{} print(sum(W)) \PYZsh{} 验证特征值求和值为 p}
\PY{c+c1}{\PYZsh{} VV = np.dot(V,V.T)}
\PY{c+c1}{\PYZsh{} VV = pd.DataFrame(VV)}
\PY{c+c1}{\PYZsh{} print(VV) \PYZsh{} 验证矩阵 V\PYZsq{}V 结果为单位阵}
\end{Verbatim}
\end{tcolorbox}

    \begin{Verbatim}[commandchars=\\\{\}]
特征值: [4.60337710e+00 1.17534050e+00 2.03425372e-01 1.49282587e-02
 2.55206576e-03 3.76708133e-04]
    \end{Verbatim}

    \textbf{判断 X 矩阵是否具有多重共线性:}

    \begin{tcolorbox}[breakable, size=fbox, boxrule=1pt, pad at break*=1mm,colback=cellbackground, colframe=cellborder]
\prompt{In}{incolor}{7}{\boxspacing}
\begin{Verbatim}[commandchars=\\\{\}]
\PY{c+c1}{\PYZsh{} 定义\PYZdq{}判断多重共线性\PYZdq{}的函数}
\PY{c+c1}{\PYZsh{} 参数: (X\PYZus{}list: 设计矩阵 X, thres\PYZus{}vif: VIF 方法判断多重共线性的阈值, thres\PYZus{}kappa: 特征值方法判断多重共线性的阈值)}
\PY{k}{def} \PY{n+nf}{judge\PYZus{}col}\PY{p}{(}\PY{n}{X\PYZus{}list}\PY{p}{,} \PY{n}{thres\PYZus{}vif}\PY{p}{,} \PY{n}{thres\PYZus{}kappa}\PY{p}{)}\PY{p}{:} 
    \PY{n}{var\PYZus{}num} \PY{o}{=} \PY{n}{X\PYZus{}list}\PY{o}{.}\PY{n}{shape}\PY{p}{[}\PY{l+m+mi}{1}\PY{p}{]}
    \PY{n+nb}{print}\PY{p}{(}\PY{l+s+s1}{\PYZsq{}}\PY{l+s+s1}{VIF方法判断结果(阈值为 }\PY{l+s+si}{\PYZpc{}d}\PY{l+s+s1}{): }\PY{l+s+s1}{\PYZsq{}}\PY{o}{\PYZpc{}} \PY{n}{thres\PYZus{}vif}\PY{p}{)}
    \PY{n}{vif} \PY{o}{=} \PY{p}{[}\PY{n}{variance\PYZus{}inflation\PYZus{}factor}\PY{p}{(}\PY{n}{X\PYZus{}list}\PY{p}{,} \PY{n}{i}\PY{p}{)} \PY{k}{for} \PY{n}{i} \PY{o+ow}{in} \PY{n+nb}{range}\PY{p}{(}\PY{n}{var\PYZus{}num}\PY{p}{)}\PY{p}{]}
    \PY{k}{for} \PY{n}{i} \PY{o+ow}{in} \PY{n+nb}{range}\PY{p}{(}\PY{n}{var\PYZus{}num}\PY{p}{)}\PY{p}{:}
        \PY{k}{if} \PY{n}{vif}\PY{p}{[}\PY{n}{i}\PY{p}{]} \PY{o}{\PYZgt{}}\PY{o}{=} \PY{n}{thres\PYZus{}vif}\PY{p}{:}
            \PY{n+nb}{print}\PY{p}{(}\PY{l+s+s1}{\PYZsq{}}\PY{l+s+s1}{设计矩阵 X 存在多重共线性.}\PY{l+s+s1}{\PYZsq{}}\PY{p}{)}
            \PY{k}{break}
        \PY{k}{elif} \PY{n}{i} \PY{o}{==} \PY{n}{var\PYZus{}num}\PY{o}{\PYZhy{}}\PY{l+m+mi}{1}\PY{p}{:}
            \PY{n+nb}{print}\PY{p}{(}\PY{l+s+s1}{\PYZsq{}}\PY{l+s+s1}{设计矩阵 X 不存在多重共线性.}\PY{l+s+s1}{\PYZsq{}}\PY{p}{)}

    \PY{n+nb}{print}\PY{p}{(}\PY{l+s+s1}{\PYZsq{}}\PY{l+s+se}{\PYZbs{}n}\PY{l+s+s1}{特征值判定法判断结果(阈值为 }\PY{l+s+si}{\PYZpc{}d}\PY{l+s+s1}{): }\PY{l+s+s1}{\PYZsq{}}\PY{o}{\PYZpc{}} \PY{n}{thres\PYZus{}kappa}\PY{p}{)}
    \PY{n}{kappa} \PY{o}{=} \PY{p}{[}\PY{p}{]}
    \PY{k}{for} \PY{n}{i} \PY{o+ow}{in} \PY{n+nb}{range}\PY{p}{(}\PY{n}{var\PYZus{}num}\PY{p}{)}\PY{p}{:}
        \PY{n}{kappa}\PY{o}{.}\PY{n}{append}\PY{p}{(}\PY{n}{np}\PY{o}{.}\PY{n}{sqrt}\PY{p}{(}\PY{n+nb}{max}\PY{p}{(}\PY{n}{W}\PY{p}{)} \PY{o}{/} \PY{n}{W}\PY{p}{[}\PY{n}{i}\PY{p}{]}\PY{p}{)}\PY{p}{)}
    \PY{k}{if} \PY{n}{np}\PY{o}{.}\PY{n}{max}\PY{p}{(}\PY{n}{kappa}\PY{p}{)} \PY{o}{\PYZgt{}}\PY{o}{=} \PY{n}{thres\PYZus{}kappa}\PY{p}{:}
        \PY{n+nb}{print}\PY{p}{(}\PY{l+s+s1}{\PYZsq{}}\PY{l+s+s1}{设计矩阵 X 存在多重共线性,其中kappa值为:}\PY{l+s+si}{\PYZpc{}.4f}\PY{l+s+s1}{\PYZsq{}}\PY{o}{\PYZpc{}} \PY{n}{np}\PY{o}{.}\PY{n}{max}\PY{p}{(}\PY{n}{kappa}\PY{p}{)}\PY{p}{)}
    \PY{k}{else}\PY{p}{:}
        \PY{n+nb}{print}\PY{p}{(}\PY{l+s+s1}{\PYZsq{}}\PY{l+s+s1}{设计矩阵 X 不存在多重共线性,其中kappa值为:}\PY{l+s+si}{\PYZpc{}.4f}\PY{l+s+s1}{\PYZsq{}}\PY{o}{\PYZpc{}} \PY{n}{np}\PY{o}{.}\PY{n}{max}\PY{p}{(}\PY{n}{kappa}\PY{p}{)}\PY{p}{)}

\PY{c+c1}{\PYZsh{} 判断多重共线性}
\PY{n}{judge\PYZus{}col}\PY{p}{(}\PY{n}{X\PYZus{}std}\PY{p}{[}\PY{p}{:}\PY{p}{,}\PY{l+m+mi}{1}\PY{p}{:}\PY{n}{p}\PY{o}{+}\PY{l+m+mi}{1}\PY{p}{]}\PY{p}{,} \PY{l+m+mi}{5}\PY{p}{,} \PY{l+m+mi}{10}\PY{p}{)}
\end{Verbatim}
\end{tcolorbox}

    \begin{Verbatim}[commandchars=\\\{\}]
VIF方法判断结果(阈值为 5):
设计矩阵 X 存在多重共线性.

特征值判定法判断结果(阈值为 10):
设计矩阵 X 存在多重共线性,其中kappa值为:110.5442
    \end{Verbatim}

    \textbf{Q2:}

本题需要构造出主成分矩阵,然后利用累计贡献率和特征值选择出适当的要保留的主成分的数量,并选择出需要保留的主成分。

    \textbf{构造主成分矩阵 Z:}

    \begin{tcolorbox}[breakable, size=fbox, boxrule=1pt, pad at break*=1mm,colback=cellbackground, colframe=cellborder]
\prompt{In}{incolor}{8}{\boxspacing}
\begin{Verbatim}[commandchars=\\\{\}]
\PY{c+c1}{\PYZsh{} 构造主成分矩阵 Z}
\PY{n}{Z} \PY{o}{=} \PY{n}{np}\PY{o}{.}\PY{n}{dot}\PY{p}{(}\PY{n}{X\PYZus{}std}\PY{p}{[}\PY{p}{:}\PY{p}{,}\PY{l+m+mi}{1}\PY{p}{:}\PY{n}{p}\PY{o}{+}\PY{l+m+mi}{1}\PY{p}{]}\PY{p}{,}\PY{n}{V}\PY{o}{.}\PY{n}{T}\PY{p}{)}
\PY{c+c1}{\PYZsh{} ZZ = np.dot(Z.T,Z)}
\PY{c+c1}{\PYZsh{} ZZ = pd.DataFrame(ZZ)}
\PY{c+c1}{\PYZsh{} print(ZZ)  \PYZsh{} 验证主成分矩阵 Z 各列之间正交,主对角线元素对应的是特征值}
\end{Verbatim}
\end{tcolorbox}

    \begin{tcolorbox}[breakable, size=fbox, boxrule=1pt, pad at break*=1mm,colback=cellbackground, colframe=cellborder]
\prompt{In}{incolor}{9}{\boxspacing}
\begin{Verbatim}[commandchars=\\\{\}]
\PY{n}{D} \PY{o}{=} \PY{n}{np}\PY{o}{.}\PY{n}{linalg}\PY{o}{.}\PY{n}{det}\PY{p}{(}\PY{n}{R}\PY{p}{)}
\PY{n+nb}{print}\PY{p}{(}\PY{l+s+s1}{\PYZsq{}}\PY{l+s+s1}{(X*)}\PY{l+s+se}{\PYZbs{}\PYZsq{}}\PY{l+s+s1}{X*的行列式: }\PY{l+s+s1}{\PYZsq{}}\PY{p}{,} \PY{n}{D}\PY{p}{)}
\end{Verbatim}
\end{tcolorbox}

    \begin{Verbatim}[commandchars=\\\{\}]
(X*)'X*的行列式:  1.5796154862477436e-08
    \end{Verbatim}

    由于 \(|(X^*)^{'}(X^*)| \approx 0\), 则存在一个 k, 使得
\(\lambda_{k+1},\cdots,\lambda_p\) 均近似为 0. 因此
\(\mathbf{z}_{k+1},\cdots,\mathbf{z}_p\) 近似为 \(\mathbf{0}\)

    \textbf{选主成分:}

    \begin{tcolorbox}[breakable, size=fbox, boxrule=1pt, pad at break*=1mm,colback=cellbackground, colframe=cellborder]
\prompt{In}{incolor}{10}{\boxspacing}
\begin{Verbatim}[commandchars=\\\{\}]
\PY{c+c1}{\PYZsh{} 对特征值按降序排序}
\PY{n}{W\PYZus{}srt} \PY{o}{=} \PY{n}{W}\PY{o}{.}\PY{n}{tolist}\PY{p}{(}\PY{p}{)}
\PY{n}{W\PYZus{}srt}\PY{o}{.}\PY{n}{sort}\PY{p}{(}\PY{n}{reverse}\PY{o}{=}\PY{k+kc}{True}\PY{p}{)}
\PY{n}{W\PYZus{}idx} \PY{o}{=} \PY{n}{np}\PY{o}{.}\PY{n}{argsort}\PY{p}{(}\PY{o}{\PYZhy{}}\PY{n}{W}\PY{p}{)} \PY{c+c1}{\PYZsh{} 返回的是元素值降序排序后的索引值的数组}
\PY{n+nb}{print}\PY{p}{(}\PY{l+s+s1}{\PYZsq{}}\PY{l+s+s1}{特征值为: }\PY{l+s+s1}{\PYZsq{}}\PY{p}{,} \PY{n}{W\PYZus{}srt}\PY{p}{)}
\PY{n+nb}{print}\PY{p}{(}\PY{l+s+s1}{\PYZsq{}}\PY{l+s+s1}{排序后特征值对应的原索引值: }\PY{l+s+s1}{\PYZsq{}}\PY{p}{,} \PY{n}{W\PYZus{}idx}\PY{p}{)}
\end{Verbatim}
\end{tcolorbox}

    \begin{Verbatim}[commandchars=\\\{\}]
特征值为:  [4.603377095768392, 1.1753404992571477, 0.20342537240143493,
0.014928258677276958, 0.002552065763074821, 0.0003767081326775469]
排序后特征值对应的原索引值:  [0 1 2 3 4 5]
    \end{Verbatim}

    \begin{tcolorbox}[breakable, size=fbox, boxrule=1pt, pad at break*=1mm,colback=cellbackground, colframe=cellborder]
\prompt{In}{incolor}{11}{\boxspacing}
\begin{Verbatim}[commandchars=\\\{\}]
\PY{c+c1}{\PYZsh{} 绘制主成分的累计贡献率(响应变量中解释的方差百分比)与组件数量的碎石图}
\PY{n}{comp} \PY{o}{=} \PY{n+nb}{range}\PY{p}{(}\PY{l+m+mi}{0}\PY{p}{,} \PY{n}{p}\PY{o}{+}\PY{l+m+mi}{1}\PY{p}{)}
\PY{c+c1}{\PYZsh{} 主成分的累计贡献率(计算方差百分比)}
\PY{n}{summ} \PY{o}{=} \PY{l+m+mi}{0}
\PY{n}{W\PYZus{}sum} \PY{o}{=} \PY{p}{[}\PY{l+m+mi}{0}\PY{p}{]}
\PY{k}{for} \PY{n}{i} \PY{o+ow}{in} \PY{n+nb}{range}\PY{p}{(}\PY{n}{p}\PY{p}{)}\PY{p}{:}
    \PY{n}{summ} \PY{o}{+}\PY{o}{=} \PY{n}{W\PYZus{}srt}\PY{p}{[}\PY{n}{i}\PY{p}{]}
    \PY{n}{W\PYZus{}sum}\PY{o}{.}\PY{n}{append}\PY{p}{(}\PY{n}{summ} \PY{o}{/} \PY{n}{p}\PY{p}{)}
\PY{n}{plt}\PY{o}{.}\PY{n}{plot}\PY{p}{(}\PY{n}{comp}\PY{p}{,} \PY{n}{W\PYZus{}sum}\PY{p}{)}
\PY{n}{plt}\PY{o}{.}\PY{n}{xlabel}\PY{p}{(}\PY{l+s+s1}{\PYZsq{}}\PY{l+s+s1}{Number of PCR components}\PY{l+s+s1}{\PYZsq{}}\PY{p}{)}
\PY{n}{plt}\PY{o}{.}\PY{n}{ylabel}\PY{p}{(}\PY{l+s+s1}{\PYZsq{}}\PY{l+s+s1}{Percent Variance Explained in Y}\PY{l+s+s1}{\PYZsq{}}\PY{p}{)}
\PY{k}{for} \PY{n}{i}\PY{p}{,}\PY{n}{j} \PY{o+ow}{in} \PY{n+nb}{zip}\PY{p}{(}\PY{n}{comp}\PY{p}{,} \PY{n}{W\PYZus{}sum}\PY{p}{)}\PY{p}{:}
    \PY{n}{plt}\PY{o}{.}\PY{n}{text}\PY{p}{(}\PY{n}{i}\PY{p}{,} \PY{n}{j}\PY{p}{,} \PY{l+s+s1}{\PYZsq{}}\PY{l+s+si}{\PYZpc{}.4f}\PY{l+s+s1}{\PYZsq{}} \PY{o}{\PYZpc{}} \PY{n+nb}{float}\PY{p}{(}\PY{n}{j}\PY{p}{)}\PY{p}{)}
\end{Verbatim}
\end{tcolorbox}

    \begin{center}
    \adjustimage{max size={0.9\linewidth}{0.9\paperheight}}{output_21_0.png}
    \end{center}
    { \hspace*{\fill} \\}
    
    \begin{tcolorbox}[breakable, size=fbox, boxrule=1pt, pad at break*=1mm,colback=cellbackground, colframe=cellborder]
\prompt{In}{incolor}{12}{\boxspacing}
\begin{Verbatim}[commandchars=\\\{\}]
\PY{c+c1}{\PYZsh{} 保留累计贡献率比重大的主成分}
\PY{n}{c\PYZus{}pc} \PY{o}{=} \PY{l+m+mf}{0.8}
\PY{n}{cnt} \PY{o}{=} \PY{k+kc}{True}
\PY{n}{thres} \PY{o}{=} \PY{n}{p} \PY{o}{*} \PY{n}{c\PYZus{}pc}
\PY{k}{while} \PY{n}{cnt}\PY{p}{:}
    \PY{n}{W\PYZus{}sum} \PY{o}{=} \PY{l+m+mi}{0}
    \PY{n}{W\PYZus{}summ} \PY{o}{=} \PY{n}{W\PYZus{}srt}\PY{p}{[}\PY{l+m+mi}{0}\PY{p}{]} 
    \PY{k}{for} \PY{n}{i} \PY{o+ow}{in} \PY{n+nb}{range}\PY{p}{(}\PY{n}{p}\PY{o}{\PYZhy{}}\PY{l+m+mi}{1}\PY{p}{)}\PY{p}{:}
        \PY{n}{k1} \PY{o}{=} \PY{n}{i} \PY{o}{+} \PY{l+m+mi}{1} 
        \PY{n}{W\PYZus{}sum} \PY{o}{+}\PY{o}{=} \PY{n}{W\PYZus{}srt}\PY{p}{[}\PY{n}{i}\PY{p}{]}
        \PY{n}{W\PYZus{}summ} \PY{o}{+}\PY{o}{=} \PY{n}{W\PYZus{}srt}\PY{p}{[}\PY{n}{i}\PY{o}{+}\PY{l+m+mi}{1}\PY{p}{]}
        \PY{c+c1}{\PYZsh{} print(i,W\PYZus{}sum,W\PYZus{}summ,thres)}
        \PY{k}{if} \PY{p}{(}\PY{n}{W\PYZus{}sum} \PY{o}{\PYZlt{}} \PY{n}{thres}\PY{p}{)} \PY{o}{\PYZam{}} \PY{p}{(}\PY{n}{W\PYZus{}summ} \PY{o}{\PYZgt{}}\PY{o}{=} \PY{n}{thres}\PY{p}{)}\PY{p}{:}
            \PY{n}{cnt} \PY{o}{=} \PY{k+kc}{False}
            \PY{k}{break}
        \PY{k}{elif} \PY{n}{i} \PY{o}{==} \PY{n}{p} \PY{o}{\PYZhy{}} \PY{l+m+mi}{2}\PY{p}{:}
            \PY{n}{cnt} \PY{o}{=} \PY{k+kc}{False}
            \PY{n}{k1} \PY{o}{=} \PY{n}{i} \PY{o}{+} \PY{l+m+mi}{1}
            \PY{k}{break}
\PY{n}{k1} \PY{o}{=} \PY{n}{k1} \PY{o}{+} \PY{l+m+mi}{1}
\PY{n+nb}{print}\PY{p}{(}\PY{l+s+s1}{\PYZsq{}}\PY{l+s+s1}{保留变量个数为: }\PY{l+s+s1}{\PYZsq{}}\PY{p}{,} \PY{n}{k1}\PY{p}{)} 
\end{Verbatim}
\end{tcolorbox}

    \begin{Verbatim}[commandchars=\\\{\}]
保留变量个数为:  2
    \end{Verbatim}

    \begin{tcolorbox}[breakable, size=fbox, boxrule=1pt, pad at break*=1mm,colback=cellbackground, colframe=cellborder]
\prompt{In}{incolor}{13}{\boxspacing}
\begin{Verbatim}[commandchars=\\\{\}]
\PY{c+c1}{\PYZsh{} 删除特征值接近于零的主成分}
\PY{k}{for} \PY{n}{i} \PY{o+ow}{in} \PY{n+nb}{range}\PY{p}{(}\PY{n}{p}\PY{p}{)}\PY{p}{:}
    \PY{k}{if} \PY{n}{W\PYZus{}srt}\PY{p}{[}\PY{n}{i}\PY{p}{]} \PY{o}{\PYZlt{}} \PY{l+m+mi}{1}\PY{p}{:}
        \PY{n}{k2} \PY{o}{=} \PY{n}{i}
        \PY{k}{break}
\PY{n+nb}{print}\PY{p}{(}\PY{l+s+s1}{\PYZsq{}}\PY{l+s+s1}{保留变量个数为: }\PY{l+s+s1}{\PYZsq{}}\PY{p}{,} \PY{n}{k2}\PY{p}{)}   
\end{Verbatim}
\end{tcolorbox}

    \begin{Verbatim}[commandchars=\\\{\}]
保留变量个数为:  2
    \end{Verbatim}

    \begin{tcolorbox}[breakable, size=fbox, boxrule=1pt, pad at break*=1mm,colback=cellbackground, colframe=cellborder]
\prompt{In}{incolor}{14}{\boxspacing}
\begin{Verbatim}[commandchars=\\\{\}]
\PY{c+c1}{\PYZsh{} 均方误差确定 k}
\PY{n}{mse} \PY{o}{=} \PY{l+m+mi}{0}
\PY{k}{for} \PY{n}{i} \PY{o+ow}{in} \PY{n+nb}{range}\PY{p}{(}\PY{n}{p}\PY{p}{)}\PY{p}{:}
    \PY{n}{k3} \PY{o}{=} \PY{n}{p}
    \PY{n}{mse} \PY{o}{+}\PY{o}{=} \PY{l+m+mi}{1} \PY{o}{/} \PY{n}{W\PYZus{}srt}\PY{p}{[}\PY{n}{i}\PY{p}{]}
    \PY{n+nb}{print}\PY{p}{(}\PY{l+m+mi}{5} \PY{o}{*} \PY{p}{(}\PY{n}{i} \PY{o}{+} \PY{l+m+mi}{1}\PY{p}{)}\PY{p}{,} \PY{n}{mse}\PY{p}{)}
    \PY{k}{if} \PY{n}{mse} \PY{o}{\PYZgt{}} \PY{l+m+mi}{5} \PY{o}{*} \PY{p}{(}\PY{n}{i} \PY{o}{+} \PY{l+m+mi}{1}\PY{p}{)}\PY{p}{:}
        \PY{n}{k3} \PY{o}{=} \PY{n}{i}
        \PY{k}{break}
    \PY{k}{elif} \PY{n}{i} \PY{o}{==} \PY{n}{p}\PY{o}{\PYZhy{}}\PY{l+m+mi}{1}\PY{p}{:}
        \PY{n}{k3} \PY{o}{=} \PY{n}{p}
        \PY{k}{break}
\PY{n+nb}{print}\PY{p}{(}\PY{l+s+s1}{\PYZsq{}}\PY{l+s+s1}{保留变量个数 \PYZlt{}=}\PY{l+s+s1}{\PYZsq{}}\PY{p}{,} \PY{n}{k3}\PY{p}{)} 
\end{Verbatim}
\end{tcolorbox}

    \begin{Verbatim}[commandchars=\\\{\}]
5 0.21723182333231833
10 1.0680490976728447
15 5.983856738553664
20 72.97090604003378
保留变量个数 <= 3
    \end{Verbatim}

    根据选择的累计贡献率,特征值,均方误差三个指标,综上,我们选择保留变量的个数为
2.

    \textbf{Q3:}

在设计矩阵X具有多重共线性时,选择合适的k,可以降低回归系数的均方误差。在进行主成分回归时,代码有两种编写方法:一为利用矩阵拆分的原理进行主成分回归,二为直接调用PCA库进行主成分分析。以下依次是两种方法的代码。

    \begin{tcolorbox}[breakable, size=fbox, boxrule=1pt, pad at break*=1mm,colback=cellbackground, colframe=cellborder]
\prompt{In}{incolor}{15}{\boxspacing}
\begin{Verbatim}[commandchars=\\\{\}]
\PY{c+c1}{\PYZsh{} 矩阵拆分}
\PY{n}{k} \PY{o}{=} \PY{n}{k1}
\PY{n}{list\PYZus{}var1} \PY{o}{=} \PY{n}{W\PYZus{}idx}\PY{p}{[}\PY{l+m+mi}{0}\PY{p}{:}\PY{n}{k}\PY{p}{]} \PY{c+c1}{\PYZsh{} 记录降序排序后的前 k 个主成分}
\PY{n}{list\PYZus{}var2} \PY{o}{=} \PY{n}{W\PYZus{}idx}\PY{p}{[}\PY{n}{k}\PY{p}{:}\PY{p}{]}
\PY{c+c1}{\PYZsh{} list\PYZus{}var1 = [0,2] }
\PY{c+c1}{\PYZsh{} list\PYZus{}var2 = [1,3]}

\PY{n}{Z\PYZus{}1} \PY{o}{=} \PY{n}{Z}\PY{p}{[}\PY{p}{:}\PY{p}{,}\PY{n}{list\PYZus{}var1}\PY{p}{]}
\PY{n}{Z\PYZus{}2} \PY{o}{=} \PY{n}{Z}\PY{p}{[}\PY{p}{:}\PY{p}{,}\PY{n}{list\PYZus{}var2}\PY{p}{]}

\PY{n}{W\PYZus{}diag\PYZus{}1} \PY{o}{=} \PY{n}{np}\PY{o}{.}\PY{n}{diag}\PY{p}{(}\PY{n}{W\PYZus{}diag}\PY{p}{[}\PY{n}{list\PYZus{}var1}\PY{p}{,}\PY{n}{list\PYZus{}var1}\PY{p}{]}\PY{p}{)}
\PY{n}{W\PYZus{}diag\PYZus{}2} \PY{o}{=} \PY{n}{np}\PY{o}{.}\PY{n}{diag}\PY{p}{(}\PY{n}{W\PYZus{}diag}\PY{p}{[}\PY{n}{list\PYZus{}var2}\PY{p}{,}\PY{n}{list\PYZus{}var2}\PY{p}{]}\PY{p}{)}

\PY{c+c1}{\PYZsh{} 按行进行拆分}
\PY{n}{V\PYZus{}1} \PY{o}{=} \PY{n}{V}\PY{p}{[}\PY{n}{list\PYZus{}var1}\PY{p}{,}\PY{p}{:}\PY{p}{]}
\PY{n}{V\PYZus{}2} \PY{o}{=} \PY{n}{V}\PY{p}{[}\PY{n}{list\PYZus{}var2}\PY{p}{,}\PY{p}{:}\PY{p}{]}

\PY{c+c1}{\PYZsh{} α的估计}
\PY{c+c1}{\PYZsh{} alpha\PYZus{}hat = np.linalg.inv(W\PYZus{}diag) @ Z.T @ Y\PYZus{}std}
\PY{n}{alpha1\PYZus{}hat} \PY{o}{=} \PY{n}{np}\PY{o}{.}\PY{n}{linalg}\PY{o}{.}\PY{n}{inv}\PY{p}{(}\PY{n}{W\PYZus{}diag\PYZus{}1}\PY{p}{)} \PY{o}{@} \PY{n}{Z\PYZus{}1}\PY{o}{.}\PY{n}{T} \PY{o}{@} \PY{n}{Y\PYZus{}std}
\PY{n+nb}{print}\PY{p}{(}\PY{l+s+s1}{\PYZsq{}}\PY{l+s+s1}{系数:}\PY{l+s+s1}{\PYZsq{}}\PY{p}{,} \PY{n}{alpha1\PYZus{}hat}\PY{p}{)}

\PY{c+c1}{\PYZsh{} 主成分估计}
\PY{c+c1}{\PYZsh{} beta\PYZus{}pc = np.dot(V\PYZus{}1.T,alpha1\PYZus{}hat)}
\PY{c+c1}{\PYZsh{} print(beta\PYZus{}pc)}
\PY{c+c1}{\PYZsh{} print(V\PYZus{}1.T @ V\PYZus{}1 @ beta\PYZus{}std[1:]) \PYZsh{} 验证PPT 99页的性质1}
\end{Verbatim}
\end{tcolorbox}

    \begin{Verbatim}[commandchars=\\\{\}]
系数: [0.44565109 0.11156928]
    \end{Verbatim}

    \begin{tcolorbox}[breakable, size=fbox, boxrule=1pt, pad at break*=1mm,colback=cellbackground, colframe=cellborder]
\prompt{In}{incolor}{16}{\boxspacing}
\begin{Verbatim}[commandchars=\\\{\}]
\PY{c+c1}{\PYZsh{} 使用拆分后的数据用线性回归模型进行建模}
\PY{n}{X\PYZus{}pc} \PY{o}{=} \PY{n}{Z\PYZus{}1}
\PY{n}{model\PYZus{}pc} \PY{o}{=} \PY{n}{sm}\PY{o}{.}\PY{n}{OLS}\PY{p}{(}\PY{n}{Y\PYZus{}std}\PY{p}{,} \PY{n}{X\PYZus{}pc}\PY{p}{)}\PY{o}{.}\PY{n}{fit}\PY{p}{(}\PY{p}{)}
\PY{n}{model\PYZus{}pc}\PY{o}{.}\PY{n}{summary}\PY{p}{(}\PY{p}{)}
\end{Verbatim}
\end{tcolorbox}

    \begin{Verbatim}[commandchars=\\\{\}]
/Library/Frameworks/Python.framework/Versions/3.6/lib/python3.6/site-
packages/scipy/stats/stats.py:1604: UserWarning: kurtosistest only valid for
n>=20 {\ldots} continuing anyway, n=16
  "anyway, n=\%i" \% int(n))
    \end{Verbatim}

            \begin{tcolorbox}[breakable, size=fbox, boxrule=.5pt, pad at break*=1mm, opacityfill=0]
\prompt{Out}{outcolor}{16}{\boxspacing}
\begin{Verbatim}[commandchars=\\\{\}]
<class 'statsmodels.iolib.summary.Summary'>
"""
                                 OLS Regression Results
================================================================================
=======
Dep. Variable:                      y   R-squared (uncentered):
0.929
Model:                            OLS   Adj. R-squared (uncentered):
0.919
Method:                 Least Squares   F-statistic:
91.43
Date:                Tue, 20 Apr 2021   Prob (F-statistic):
9.20e-09
Time:                        22:19:03   Log-Likelihood:
20.625
No. Observations:                  16   AIC:
-37.25
Df Residuals:                      14   BIC:
-35.71
Df Model:                           2
Covariance Type:            nonrobust
==============================================================================
                 coef    std err          t      P>|t|      [0.025      0.975]
------------------------------------------------------------------------------
x1             0.4457      0.033     13.416      0.000       0.374       0.517
x2             0.1116      0.066      1.697      0.112      -0.029       0.253
==============================================================================
Omnibus:                        0.210   Durbin-Watson:                   1.919
Prob(Omnibus):                  0.900   Jarque-Bera (JB):                0.371
Skew:                          -0.201   Prob(JB):                        0.831
Kurtosis:                       2.371   Cond. No.                         1.98
==============================================================================

Notes:
[1] R² is computed without centering (uncentered) since the model does not
contain a constant.
[2] Standard Errors assume that the covariance matrix of the errors is correctly
specified.
"""
\end{Verbatim}
\end{tcolorbox}
        
    \begin{tcolorbox}[breakable, size=fbox, boxrule=1pt, pad at break*=1mm,colback=cellbackground, colframe=cellborder]
\prompt{In}{incolor}{17}{\boxspacing}
\begin{Verbatim}[commandchars=\\\{\}]
\PY{c+c1}{\PYZsh{} 判断多重共线性【k \PYZgt{} 1 时才可能存在多重共线性问题】}
\PY{n}{judge\PYZus{}col}\PY{p}{(}\PY{n}{X\PYZus{}pc}\PY{p}{,} \PY{n}{thres\PYZus{}vif}\PY{o}{=}\PY{l+m+mi}{5}\PY{p}{,} \PY{n}{thres\PYZus{}kappa}\PY{o}{=}\PY{l+m+mi}{10}\PY{p}{)}
\end{Verbatim}
\end{tcolorbox}

    \begin{Verbatim}[commandchars=\\\{\}]
VIF方法判断结果(阈值为 5):
设计矩阵 X 不存在多重共线性.

特征值判定法判断结果(阈值为 10):
设计矩阵 X 不存在多重共线性,其中kappa值为:1.9790
    \end{Verbatim}

    \begin{tcolorbox}[breakable, size=fbox, boxrule=1pt, pad at break*=1mm,colback=cellbackground, colframe=cellborder]
\prompt{In}{incolor}{18}{\boxspacing}
\begin{Verbatim}[commandchars=\\\{\}]
\PY{c+c1}{\PYZsh{} 创建pca模型}
\PY{n}{pca} \PY{o}{=} \PY{n}{PCA}\PY{p}{(}\PY{n}{n\PYZus{}components}\PY{o}{=}\PY{l+m+mi}{2}\PY{p}{)}

\PY{c+c1}{\PYZsh{} 对模型进行训练}
\PY{n}{X\PYZus{}pc\PYZus{}} \PY{o}{=} \PY{n}{X\PYZus{}std} \PY{o}{*} \PY{l+m+mf}{1.0}
\PY{n}{pca}\PY{o}{.}\PY{n}{fit}\PY{p}{(}\PY{n}{X\PYZus{}pc\PYZus{}}\PY{p}{)}

\PY{c+c1}{\PYZsh{} 返回降维后数据}
\PY{n}{X\PYZus{}pc\PYZus{}} \PY{o}{=} \PY{n}{pca}\PY{o}{.}\PY{n}{transform}\PY{p}{(}\PY{n}{X\PYZus{}pc\PYZus{}}\PY{p}{)}

\PY{c+c1}{\PYZsh{} 使用返回后的数据用线性回归模型进行建模}
\PY{n}{model\PYZus{}pc\PYZus{}} \PY{o}{=} \PY{n}{sm}\PY{o}{.}\PY{n}{OLS}\PY{p}{(}\PY{n}{Y\PYZus{}std}\PY{p}{,} \PY{n}{X\PYZus{}pc\PYZus{}}\PY{p}{)}\PY{o}{.}\PY{n}{fit}\PY{p}{(}\PY{p}{)}
\PY{n}{model\PYZus{}pc\PYZus{}}\PY{o}{.}\PY{n}{summary}\PY{p}{(}\PY{p}{)}
\end{Verbatim}
\end{tcolorbox}

    \begin{Verbatim}[commandchars=\\\{\}]
/Library/Frameworks/Python.framework/Versions/3.6/lib/python3.6/site-
packages/scipy/stats/stats.py:1604: UserWarning: kurtosistest only valid for
n>=20 {\ldots} continuing anyway, n=16
  "anyway, n=\%i" \% int(n))
    \end{Verbatim}

            \begin{tcolorbox}[breakable, size=fbox, boxrule=.5pt, pad at break*=1mm, opacityfill=0]
\prompt{Out}{outcolor}{18}{\boxspacing}
\begin{Verbatim}[commandchars=\\\{\}]
<class 'statsmodels.iolib.summary.Summary'>
"""
                                 OLS Regression Results
================================================================================
=======
Dep. Variable:                      y   R-squared (uncentered):
0.929
Model:                            OLS   Adj. R-squared (uncentered):
0.919
Method:                 Least Squares   F-statistic:
91.43
Date:                Tue, 20 Apr 2021   Prob (F-statistic):
9.20e-09
Time:                        22:19:03   Log-Likelihood:
20.625
No. Observations:                  16   AIC:
-37.25
Df Residuals:                      14   BIC:
-35.71
Df Model:                           2
Covariance Type:            nonrobust
==============================================================================
                 coef    std err          t      P>|t|      [0.025      0.975]
------------------------------------------------------------------------------
x1            -0.4457      0.033    -13.416      0.000      -0.517      -0.374
x2             0.1116      0.066      1.697      0.112      -0.029       0.253
==============================================================================
Omnibus:                        0.210   Durbin-Watson:                   1.919
Prob(Omnibus):                  0.900   Jarque-Bera (JB):                0.371
Skew:                          -0.201   Prob(JB):                        0.831
Kurtosis:                       2.371   Cond. No.                         1.98
==============================================================================

Notes:
[1] R² is computed without centering (uncentered) since the model does not
contain a constant.
[2] Standard Errors assume that the covariance matrix of the errors is correctly
specified.
"""
\end{Verbatim}
\end{tcolorbox}
        
    \begin{tcolorbox}[breakable, size=fbox, boxrule=1pt, pad at break*=1mm,colback=cellbackground, colframe=cellborder]
\prompt{In}{incolor}{19}{\boxspacing}
\begin{Verbatim}[commandchars=\\\{\}]
\PY{c+c1}{\PYZsh{} 判断多重共线性【k 值大于 1 时才可能存在多重共线性问题】}
\PY{n}{judge\PYZus{}col}\PY{p}{(}\PY{n}{X\PYZus{}pc\PYZus{}}\PY{p}{,} \PY{n}{thres\PYZus{}vif}\PY{o}{=}\PY{l+m+mi}{5}\PY{p}{,} \PY{n}{thres\PYZus{}kappa}\PY{o}{=}\PY{l+m+mi}{10}\PY{p}{)}
\end{Verbatim}
\end{tcolorbox}

    \begin{Verbatim}[commandchars=\\\{\}]
VIF方法判断结果(阈值为 5):
设计矩阵 X 不存在多重共线性.

特征值判定法判断结果(阈值为 10):
设计矩阵 X 不存在多重共线性,其中kappa值为:1.9790
    \end{Verbatim}

    可以看出,选择合适的k用于进行主成分回归,可以使得到的设计矩阵X不再具有多重共线性。故主成分回归可用于数据具有多重共线性的情形。


    % Add a bibliography block to the postdoc
    
    
    
\end{document}
